% !TEX root = ../FundationsDataScience.tex

\chapter{Shannon Theory}
\label{sec-shannon}

The main reference is~\cite{mallat2008wavelet}.

%%%%%%%%%%%%%%%%%%%%%%%%%%%%%%%%%%%%%%%%%%%%%%%%%%%%%%%%%%%%%%%%%%%%
%%%%%%%%%%%%%%%%%%%%%%%%%%%%%%%%%%%%%%%%%%%%%%%%%%%%%%%%%%%%%%%%%%%%
%%%%%%%%%%%%%%%%%%%%%%%%%%%%%%%%%%%%%%%%%%%%%%%%%%%%%%%%%%%%%%%%%%%%
\section{Analog vs. Discrete Signals}

To develop numerical tools and analyze their performances, the mathematical modeling is usually done over a continuous setting. 
An analog signal is a 1D function $f_0 \in \Ldeux([0,1])$ where $[0,1]$ denotes the domain of acquisition, which might for instance be time. An analog image is a 2D function $f_0 \in \Ldeux([0,1]^2)$ where the unit square $[0,1]^2$ is the image domain.

Although these notes are focussed on the processing of sounds and natural images, most of the methods extend to multi-dimensional datasets, which are higher dimensional mappings
\eq{
	f_0 : [0,1]^d \rightarrow [0,1]^s
}
where $d$ is the dimensionality of the input space ($d=1$ for sound and $d=2$ for images) whereas $s$ is the dimensionality of the feature space. For instance, gray scale images corresponds to $(d=2,s=1)$, 
videos to $(d=3, s=1)$, color images to $(d=2, s=3)$ where one has three channels $(R,G,B)$.
One can even consider multi-spectral images where $(d=2, s \gg 3)$ that is made of a large number of channels for different light wavelengths. Figures \ref{fig-examples-1} and \ref{fig-examples-2} show examples of such data.


\myfigure{
	\image{orthobases}{.35}{example-sound}
	\image{orthobases}{.28}{example-image}
	\image{orthobases}{.3}{example-video}
}{%
	Examples of sounds ($d=1$), image ($d=2$) and videos ($d=3$). %	
}{fig-examples-1}

\myfigure{
	\image{orthobases}{.4}{example-color}
	\image{orthobases}{.5}{example-multispectral}
}{%
	Example of color image $s=3$ and multispectral image ($s=32$). %	
}{fig-examples-2}


%%%%%%%%%%%%%%%%%%%%%%%%%%%%%%%%%%%%%%%%%%%
\subsection{Acquisition and Sampling}

Signal acquisition is a low dimensional projection of the continuous signal performed by some hardware device. This is for instance the case for a microphone that acquires 1D samples or a digital camera that acquires 2D pixel samples.
The sampling operation thus corresponds to mapping from the set of continuous functions to a discrete finite dimensional vector with $N$ entries.
\eq{
	f_0 \in \Ldeux([0,1]^d) \mapsto f \in \CC^N
}

\myfigure{
	\image{orthobases}{.4}{discretization-image}
	\image{orthobases}{.5}{discretization-sound}
}{%
	Image and sound discretization. %	
}{fig-discretization}

Figure \ref{fig-discretization} shows examples of discretized signals.

%%%%%%%%%%%%%%%%%%%%%%%%%%%%%%%%%%%%%%%%%%%
\subsection{Linear Translation Invariant Sampler}

A translation invariant sampler performs the acquisition as an inner product between the continuous signal and a constant impulse response $h$ translated at the sample location
\eql{\label{eq-linear-sampling}
	f[n] = \int_{-S/2}^{S/2} f_0(x) h(n/N - x) \d x= f_0 \star h(n/N).
}
The precise shape of $h(x)$ depends on the sampling device, and is usually a smooth low pass function that is maximal around $x=0$. The size $S$ of the sampler determines the precision of the sampling device, and is usually of the order of $1/N$ to avoid blurring (if $S$ is too large) or aliasing (if $S$ is too small).

Section \ref{sec-sampling} details how to reverse the sampling operation in the case where the function is smooth.


%%%%%%%%%%%%%%%%%%%%%%%%%%%%%%%%%%%%%%%%%%%%%%%%%%%%%%%%%%%%%%%%%%%%
%%%%%%%%%%%%%%%%%%%%%%%%%%%%%%%%%%%%%%%%%%%%%%%%%%%%%%%%%%%%%%%%%%%%
%%%%%%%%%%%%%%%%%%%%%%%%%%%%%%%%%%%%%%%%%%%%%%%%%%%%%%%%%%%%%%%%%%%%
\section{Shannon Sampling Theorem}
\label{subsec-sampling}

%%%
\paragraph{Reminders about Fourier transform.}

For $f \in L^1(\RR)$, its Fourier transform is defined as
\eql{\label{eq-fourier-transform}
	\foralls \om \in \RR, \quad
	\hat f(\om) \eqdef \int_\RR f(x) e^{-\imath x \om} \d x.
}
One has $\norm{\hat f}^2 = (2\pi)^{-1} \norm{f}^2$, so that $f \mapsto \hat f$ can be extended by continuity to $L^2(\RR)$, which corresponds to computing $\hat f$ as a limit when $T \rightarrow +\infty$ of $\int_{-T}^T f(x) e^{-\imath x \om} \d x$.
%
When $\hat f \in L^1(\RR)$, one can invert the Fourier transform so that
\eql{\label{eq-i-ft}
	f(x) = \int_\RR \hat f(\om) e^{\imath x \om} \d \om, 
}
which shows in particular that $f$ is continuous with vanishing limits at $\pm\infty$. 

The Fourier transform $\Ff : f \mapsto \hat f$ exchanges regularity and decay. For instance, if $f \in C^p(\RR)$ with an integrable Fourier transform, then $\Ff(f^{(p)})(\om) = (\imath \om)^{-p} \hat f(\om)$ so that $|\hat f(\om)|=O(1/|\om|^p)$. 
%
Conversely, 
\eql{\label{eq-fourier-regul}
	\int_\RR (1+|\om|)^{-p} |\hat f(\om)| \d \om<+\infty
	\qarrq f \in C^p(\RR).
}

%%%
\paragraph{Reminders about Fourier series.}

We denote $\TT=\RR/2\pi\ZZ$ the torus.
%
A function $f \in L^2(\TT)$ is $2\pi$-periodic, and can be viewed as a function $f \in L^2([0,1])$ (beware that this means that the boundary points are glued together), and its Fourier coefficients are
\eq{
	\foralls n \in \ZZ, \quad 
	\hat f_n \eqdef \frac{1}{2\pi}\int_0^{2\pi} f(x) e^{-\imath x n} \d x.
}
This formula is equivalent to the computation of an inner-product $\hat f_n = \dotp{f}{e_n}$ for the inner-product $\dotp{f}{g} \eqdef \frac{1}{2\pi} \int_\TT f(x) \bar g(x) \d x$. 
%
For this inner product, $(e_n)_n$ is orthonormal and is actually an Hilbert basis, meaning that one reconstruct with the following converging series 
\eql{\label{eq-fourier-series}
	f = \sum_{n \in \ZZ} \dotp{f}{e_n} e_n
}
which means $\norm{f-\sum_{n=-N}^N \dotp{f}{e_n} e_n}_{L^2(\TT)} \rightarrow 0$ for $N \rightarrow +\infty$.
%
The pointwise convergence of~\eqref{eq-fourier-series}, and is ensured (and there is normal convergence) when for instance $f \in C^3(\TT)$. 


%%%
\paragraph{Poisson formula.}

The poisson formula connects the Fourier transform and the Fourier series to sampling and periodization operators.
%
For some function $\hat f(\om)$ defined on $\RR$, its periodization reads
\eql{\label{eq-periodizing}
	\hat f_P(\om) \eqdef \sum_n f(\om-2\pi n).
} 
This formula makes sense if $\hat f \in L^1(\RR)$, and in this case $\norm{\hat f_P}_{L^1(\TT)} \leq \norm{\hat f}_1$
%
The Poisson formula, state in Proposition~\ref{prop-poisson}�bellow, corresponds to proving that the following diagram
\eq{
	\begin{array}{rcccl}
						& f(x)  &  \overset{\Ff}{\longrightarrow} &  \hat f(\om) &\\
		\text{sampling}& \downarrow & & \downarrow &\text{periodization} \\
						& (f(n))_n  &  \overset{\text{Fourier serie}}{\longrightarrow} &  \sum_n f(n) e^{-\imath \om n} &\\
	\end{array}
}
is actually commutative.

\begin{prop}[Poisson formula]\label{prop-poisson}
Assume that $\hat f$ has compact support and that $|f(x)| \leq C(1+|x|)^{-3}$ for some $C$. Then one has 
\eql{\label{eq-poisson-formula}
	\foralls \om \in \RR, \quad
	\sum_n f(n) e^{-\imath \om n} = \hat f_P(\om).
}
\end{prop}
\begin{proof}
	Since $\hat f$ is compactly supported, $\hat f_P$ is well defined (it involves only a finite sum) and since $f$ has fast decay, using~\eqref{eq-fourier-regul}, $\hat f_P$ is $C^1$. It is thus the sum of its Fourier transform
	\eql{\label{eq-poisson-formula}
		\hat f_P(\om) = \sum_k c_k e^{\imath k \om},
	} 
	where
	\begin{align*}
		c_k = \frac{1}{2\pi} \int_0^{2\pi} \hat f_P(\om) e^{-\imath k \om} \d \om = 
		\frac{1}{2\pi} \int_0^{2\pi} \sum_n f(x-2\pi n) e^{-\imath k \om}  \d \om .
	\end{align*}
	One has 
	\eq{
		\int_0^{2\pi} \sum_n |f(x-2\pi n) e^{-\imath k \om}|  \d \om = \int_\RR |f| 
	}
	which is bounded because $\hat f \in L^1(\RR)$ (it has a compact support and is $C^1$), so one can exchange the sum and integral
	\eq{
		c_k = \sum_n \frac{1}{2\pi} \int_0^{2\pi} f(x-2\pi  n) e^{-\imath k \om}  \d \om
		= \frac{1}{2\pi} \int_{\RR} f(x) e^{-\imath k \om}  \d \om
		= f(-k)
	}
	where we used the inverse Fourier transform formula~\eqref{eq-i-ft}, which is legit because $\hat f \in L^1(\RR)$.
\end{proof}


%%%
\paragraph{Shannon theorem.}

Shannon sampling theorem state a sufficient condition ensuring that the sampling operator $f \mapsto (f(ns))_n$ is invertible for some sampling step size $s>0$. 
%
It require that $\supp(\hat f) \subset [-\pi/s,\pi/s]$, which, thanks to formula~\eqref{eq-i-ft}, implies that $\hat f$ is $C^\infty$ (in fact it is even analytic). 

\begin{thm} \label{thm-shannon-sampling}
	If $|f(x)| \leq C(1+|x|)^{-3}$ for some $C$ and $\supp(\hat f) \subset [-\pi/s,\pi/s]$, then one has
	\eql{\label{eq-shannong-interp}
		\foralls x \in \RR, \quad 
		f(x) = \sum_n f(n s) \sinc(x/s-n) \qwhereq
		\sinc(u) = \frac{\sin(\pi u)}{\pi u}
	}
	with uniform convergence.
\end{thm}

\begin{proof} 
	The change of variable $g=f(s \cdot)$ results in $\hat g=s \hat f(s\cdot)$ so that we can restrict our attention to $s=1$.
	%
	The compact support hypothesis implies $\hat f(\om) = 1_{[-\pi,\pi]}(\om) \hat f_P(\om)$.  
	Combining the inversion formula~\eqref{eq-i-ft} with Poisson formula~\eqref{eq-poisson-formula}
	\eq{
		f(x) = \frac{1}{2\pi} \int_{-\pi}^\pi \hat f_P(\om) e^{\imath \om x} \d \om
		= \frac{1}{2\pi} \int_{-\pi}^\pi \sum_n f(n) e^{\imath \om (x-n)} \d \om.
	} 
	Since $f$ has fast decay, $\int_{-\pi}^\pi \sum_n |f(n) e^{\imath \om (x-n)}| \d \om = \sum_n |f(n)| < +\infty$, so that one can exchange summation and integration and obtain
	\eq{
		f(x) = \sum_n f(n)  \frac{1}{2\pi} \int_{-\pi}^\pi e^{\imath \om (x-n)} \d \om = \sum_n f(n) \sinc(x-n).
	}
\end{proof}

%%%%%%%%%%%%%%%%%%%%%%%%%%%%%%%%%%%%%%%%%%%%%%%%%%%%%%%%%%%%%%%%%%%%
\section{Shannon Source Coding Theorem}
\label{sec-shannon-source}

We consider an alphabet $(x_1,\ldots,x_K)$ of $K$ symbols, and assume at our disposal some probability distribution over this alphabet, which is just an histogram $p=(p_1,\ldots,p_K) \in \RR_+^K$ in the simplex, i.e. $\sum_k p_k=1$. 

The entropy of such an histogram is 
\eq{
	H(p) \eqdef -\sum_k p_k \log_2(p_k)
}
with the convention $O\log_2(0)=0$.

\begin{lem} One has
\eq{
	0 \leq H(p) \leq \log_2(K).
}
\end{lem}
\begin{proof}
	We consider the following constrained optimization problem
	\eq{
		\min_p \enscond{ f(p) }{ g(p) = \sum_k p_k =1 }
	}
	where $f=-H$. According to the linked extrema theorem, at an optimum $p^\star$, $\nabla f(p^\star)=\la \nabla g(p^\star)$ for some $\la \in \RR$, so that here $\log(p_k^\star)+1 = \la$, i.e. $p_k^\star=c$ is constant, and since $\sum_k p_k^\star=1$, one has $p_k^*=1/K$ and thus $H(p)=\log_2(K)$.
\end{proof}

A code $c_k=c(x_k)$ associate to each symbol $x_k$ a code word $c_k \in \{0,1\}^\NN$ with a varying length $|c_k| \in \NN^*$. We denote the average length associated to this code as
\eq{
	L(c) \eqdef \sum_k p_k |c_k|. 
}

A prefix code $c_k=c(x_k)$ is such that no word $c_k$ is the beginning of another word $c_k'$. This is equivalent to be able to embed the $(c_k)_k$ as leaves of a binary tree $T$, with the code being output of a traversal from root to leaves (with a convention that going to a left (resp. right) child output a 0 (resp. a 1). 
%
We denote $c=\text{Leaves}(T)$ such prefix property. The following fundamental lemma describes the set of prefix code using an inequality.

\begin{lem}[Kraft inequality]\label{lem-kraft}
	(i) For a code $c$, if there exists a tree $T$ such that $c=\text{Leaves}(T)$ then
	\eql{\label{eq-kraft-ineg}
		\sum_k 2^{-|c_k|} \leq 1.
	}
	(ii) Conversely, if $(\ell_k)_k$ are such that
	\eql{\label{eq-kraft-ineg-2}
		\sum_k 2^{-\ell_k} \leq 1
	}
	then there exists a code $c=\text{Leaves}(T)$ such that $|c_k|=\ell_k$.
\end{lem} 

\begin{proof}
	$\Rightarrow$ We suppose $c=\text{Leaves}(T)$. We denote $m=\max_k |c_k|$ and consider the full binary tree.
	%
	Bellow each $c_k$, one has a sub-tree of height $m-|c_k|$. This sub-tree has $2^{m-|c_k|}$ leaves. Since all these sub-trees do not overlap, the total number of leaf do not exceed the total number of leaves $2^m$ of the full binary tree, hence
	\eq{
		\sum_k 2^{m-|c_k|} \leq 2^m, 
	}
	hence~\eqref{eq-kraft-ineg}.�
	
	$\Leftarrow$ Conversely, we assume~\eqref{eq-kraft-ineg} holds. Without loss of generality, we assume that $|c_1| \leq \ldots \leq |c_K|$. We start by putting a sub-tree of height $2^{m-|c_1|}$. Since the second tree is smaller, one can put it immediately aside, and continue this way. Since  $\sum_k 2^{m-|c_k|} \leq 2^m$, this ensure that we can stack side-by-side all these sub-tree, and this defines a proper sub-tree of the full binary tree. 
\end{proof}

We now are ready to state and prove Shannon theory for entropic coding.

\begin{thm}
	(i) If $c=\text{Leaves}(T)$ for some tree $T$, then 
	\eq{
		L(c) \geq H(p).
	}
	(ii) Conversely, there exists a code $c$ with $c=\text{Leaves}(T)$ such that 
	\eq{
		L(c) \leq H(p)+1.
	}
\end{thm} 

\begin{proof}
	First, we consider the following optimization problem
	\eql{\label{eq-proof-shannon}
		\umin{ \ell = (\ell_k)_k } \enscond{ f(\ell) \eqdef \sum_k \ell_k p_k }{ g(\ell) \eqdef  \sum_k 2^{-\ell_k} \leq 1 }.
	}
	We fist show that at an optimal $\ell^\star$, the constraint is saturated, i.g. $g(\ell^\star)=1$. Indeed, if $g(\ell^\star)=2^{-u} < 1$, with $u>0$, we define $\ell_k' \eqdef \ell_k^\star-u$, which satisfies $g(\ell') = 1$ and also $f(\ell')=\sum_k (\ell_k-u) p_k < f(\ell^\star)$, which is a contradiction.
	So we can restrict in~\eqref{eq-proof-shannon} the constraint to $g(\ell)=1$ and apply the linked extra theorem, which shows that necessarily, there exists $\la \in \RR$ with $\nabla f(\ell^\star)=\nabla g(\ell^\star)$, i.e.  $(p_k)_k = -\la \ln(2) (2^{-\ell_k^\star})_k$. Since $\sum_k p_k=\sum_k 2^{-\ell_k^\star}=1$, we deduce that $\ell^\star_k = -\log(p_k)$. 
	
	(i) If $c=\text{Leave}(T)$, the by Kraft inequality~\eqref{eq-kraft-ineg}, necessarily $\ell_k=|c_k|$ satisfy the constraints of~\eqref{eq-proof-shannon}, and thus $H(p) = f(\ell^\star) \leq f(\ell) = L(\ell)$.
	
	(ii) We define $\ell_k \eqdef \lceil -\log_2(p_k) \rceil \in \NN^*$. Then $\sum_k 2^{-\ell_k} \leq \sum_k 2^{\log_2(p_k)} = 1$, so that these lengths satisfy~\eqref{eq-kraft-ineg-2}. Thanks to Proposition~\ref{lem-kraft} (ii), there thus exists a prefix code $c$ with $|c_k|=\lceil -\log_2(p_k) \rceil$. Furthermore
	\eq{
		L(c) = \sum_k p_k \lceil -\log_2(p_k) \rceil \leq \sum_k p_k ( -\log_2(p_k) +1) = H(p)+1.
	}
\end{proof}

