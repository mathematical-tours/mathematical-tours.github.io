\documentclass{article}
\usepackage{amsmath,amsfonts,amssymb}

\author{Gabriel Peyré}
\date{\today}
\title{Derivation of the EM Algorithm}

\begin{document}

\maketitle

\section{Variational reformulation of $-\log \sum$}

For any vector $u$ and for any probability vector $p$, one has thanks to Jensen inequality, since $-\log$ is convex
$$
	-\log(\sum_k u_k) = -\log(\sum_k p_k \frac{u_k}{p_k})  \leq - \sum_k p_k \log(\frac{u_k}{p_k}).
$$
But actually, if one used the best $p=p^\star(u)$, one has an equality 
$$
	-\log(\sum_k u_k) = \min_{p\geq 0, \sum_k p_k=1}  - \sum_k p_k \log(\frac{u_k}{p_k}) = \text{KL}(p|u).
$$
Indeed, this optimal $p^\star(u)$ is
$$
	p^\star(u) = \frac{u}{\sum_k u_k}.
$$

\section{MLE of mixtures reformulation}

MLE problem minimizes the negative log-likelihood of a mixture 
\begin{equation}
    \min_{\theta,\pi} \mathcal{L}(\theta,\pi) := \sum_{i=1}^n -\log \left( \sum_{k=1}^K \pi_k f(x_i|\theta_k) \right)
\end{equation}
We introduce probability weights $P_{i,\cdot}$ for each $i$, and using the variational formulation of $-\log \sum$ to obtain
$$
	\mathcal{L}(\theta,\pi) = \min_{P} \mathcal{G}(\theta,\pi,P) := - \sum_{i,k} P_{i,k} \log\left( \frac{\pi_k}{P_{i,k}}f(x_i|\theta_k)  \right)
	 = \text{KL}(P|\tilde P), 
$$
$$
	 \text{where}\quad
	 \tilde P_{i,k} := \pi_k f(x_i|\theta_k).
$$
The EM algorithm is an alternate minimization on the variables of the problem
$$
	\min_{P,\theta,\pi} \mathcal{G}(\theta,\pi,P) 
$$
This guarantees that $\mathcal{L}(\theta)$ is decaying through the iterations and if $f$ is smooth and the functional is coercive (which is problematic for Gaussians!) then converging sub-sequences are guaranteed to converge to a stationary point. 

\paragraph{E step.}

The E steps correspond, given the previous iterate $\theta$, to minimizing with respect to $P$ 
$$
	\min_{ P \in \mathbb{R}^{n \times K}_+ } \{ \mathcal{G}(\theta,\pi,P)  = \text{KL}(P|\tilde P) : \sum_k P_{i,k} = 1\}
	\quad\text{where}\quad
	\tilde P_{i,k} := \pi_k f(x_i|\theta_k),
$$
which solution reads
$$
	P_{i,k} = \frac{\tilde P_{i,k}}{ \sum_k \tilde P_{i,k} }.
$$

\paragraph{M step.}

Then the M step corresponds to minimizing 
$$
	\min_{\theta,\pi} \mathcal{G}(\theta,\pi,P) 
$$
For $\pi$, one solves 
$$
	\min_\pi \{  \sum_k \sum_{i=1}^n P_{i,k} \log(\pi_k / P_{i,k})  : \sum_k \pi_k = 1\}
$$
which solution is
$$
	\pi_k = \frac{\sum_i P_{i,k}}{ \sum_{i,\ell} P_{i,\ell} }
$$
For $\theta$, this splits independently over each $k$ as a usual (non-mixtures) MLE where the points are weights by $P_{i,k}$
$$
	\min_{\theta_k} -\sum_k P_{i,k} \log( f(x_i|\theta_k) ).
$$

\paragraph{Gaussian case.}

In the Gaussian case, where
$$
	f(x|\Sigma,m) := \frac{1}{\sqrt{2\pi\det(\Sigma)}} \exp\left(-\frac{ \langle \Sigma^{-1}(x-m),x-m \rangle }{2} \right)
$$
one has
$$
	m_k = \sum_i P_{i,k} x_i  \in \mathbb{R}^{d}
		\quad\text{and}\quad 
	\Sigma_k = \sum_i P_{i,k} (x_i-m_k)^\top  (x_i-m_k) \in \mathbb{R}^{d \times d}. 
$$



\end{document}