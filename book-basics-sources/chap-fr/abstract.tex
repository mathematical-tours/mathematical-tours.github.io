% !TEX root = ../IntroImaging-FR.tex

\chapter*{Présentation}

Les quatres chapitres de ce texte sont indépendants et présentent des introductions en douceur à quelques fondements mathématiques importants des sciences des données :
\begin{itemize}
	\item Le chapitre~\ref{chap-shannon} présente la théorie de Shannon sur la compression et insiste en particulier sur l'entropie liée au codage de l'information.
	\item Le chapitre~\ref{chap-images} présente les bases du traitement d'images, en particulier des traitements importants (quantification, débruitage, couleurs).
	\item Le chapitre~\ref{chap-sparsity} présente la théorie de l'échantillonnage, allant de l'échantillonnage classique de Shannon à l'échantillonnage comprimé. Il constitue également une introduction à la régularisation des problèmes inverses.
	\item Le chapitre 4 présente le transport optimal et ses applications.
\end{itemize}
Le niveau d'exposition pour les deux premiers chapitres est élémentaire. Le dernier chapitre présente des concepts et résultats mathématiques plus avancés.

\tableofcontents