\documentclass[11pt]{article}


% Be sure to use PDF Latex
\pdfoutput=1


% links
\usepackage[bookmarks,bookmarksdepth=2, colorlinks=true, linkcolor=blue,citecolor=red, urlcolor=blue]{hyperref}


\usepackage{fullpage}

\usepackage[utf8]{inputenc}
% \usepackage[latin1]{inputenc}
\usepackage[french]{babel}

\usepackage{mystyle}
\renewcommand{\guill}[1]{«~#1~»} % french

\usepackage{url}

\usepackage[T1]{fontenc}

%\usepackage{vmargin} \setpapersize{A4}
%\newcommand{\mypage}{30mm}
% \setmarginsrb{\mypage{}}{\mypage{}}{\mypage{}}{\mypage{}}{0mm}{0mm}{0mm}{0mm}

\graphicspath{{./figures/}}
\newcommand{\otarticle}

\title{Le transport optimal numérique \\ et ses applications} 

\author{%
\begin{tabular}{c}
	Gabriel Peyr{\'e} \\ CNRS \& DMA \\
	 \'Ecole Normale Sup\'erieure \\
	 \url{gabriel.peyre@ens.fr}
\end{tabular}
}
\date{}

\begin{document}

\maketitle

\begin{abstract}
Le transport optimal est un problème très ancien, formulé par Monge au 18$^e$ siècle. Il a cependant fallu plusieurs révolutions mathématiques pour qu'il devienne un outil incontournable à la fois en théorie et en pratique. Cet article retrace ces révolutions, initiées par Leonid Kantorovitch pendant la seconde guerre mondiale. La formulation qu'il a proposée se prête à une analyse mathématique poussée ainsi qu'à son application à de nombreux problèmes. Elle fait du transport optimal un outil de choix pour aborder l'explosion récente de la science des données. 
\end{abstract}

% !TEX root = ../IntroImaging-FR.tex

\newcommand{\iC}{{\color{red}i}}
\newcommand{\jC}{{\color{blue}j}}

\chapter{Le Transport Optimal et ses Applications}
\label{chap-ot}

Livres sur le transport optimal~\cite{Villani03}

%%%%%%%%%%%%%%%%%%%%%%%%%%%%%%%%%%%%%%%%%%%%%%
\section{Le Transport Optimal de Monge}

Gaspard Monge, en plus d'�tre un grand math�maticien, a particip� activement � la r�volution Fran�aise, et a cr�er l'\'Ecole Polytechnique ainsi que l'\'Ecole Normale Sup�rieure. Motiv� par des applications militaire, il a formul� en 1781 le probl�me du transport optimal~\cite{Monge1781}. Il s'est pos� la question du calcul de la fa�on la plus �conomique de transporter de la terre entre deux endroits pour faire des remblais. Dans son texte original, il a fait l'hypoth�se que le co�t du d�placement d'une unit� de masse est �gal � la distance parcourue, mais l'on peut utiliser n'importe quel co�t adapt� au probl�me � r�soudre. 

Pour illustrer le probl�me et sa formulation math�matique, int�ressons-nous � la fa�on optimale de distribuer les croissantes depuis les boulangeries vers le caf�, le matin dans Paris. Pour simplifier, nous allons supposer qu'il y a uniquement 6 boulangeries et caf�s, que l'on peut voir � la figure~\ref{fig:image-cafe} (les boulangeries sont en rouge et les caf�s en bleu). Le co�t � minimis� est le temps de trajets, et l'on note $C_{\iC,\jC}$ le temps entre la boulangerie $\iC \in \{1,\ldots,6\}$  et le caf� $\jC \in \{1,\ldots,6\}$. Par exemple, on a $C_{{\color{red}3},{\color{blue}4}}=10$, ce qui signifie qu'il y a 10 minutes de trajet entre la boulangerie num�ro ${\color{red}3}$ et le caf� num�ro ${\color{blue}4}$. 

\begin{figure}\centering
    \begin{tabular}{@{}c@{\hspace{1mm}}c@{\hspace{4mm}}c@{\hspace{1mm}}c@{}}
        \includegraphics[width=.2\linewidth]{transport/cafe-paris/map-paris-0-couts} 
        \includegraphics[width=.24\linewidth]{transport/cafe-paris/map-paris-0} 
        \includegraphics[width=.2\linewidth]{transport/cafe-paris/map-paris-1-couts} 
        \includegraphics[width=.24\linewidth]{transport/cafe-paris/map-paris-1} 
    \end{tabular}
    \caption{\label{fig:image-cafe} Matrice de co�t et connexions associ�es. Gauche: une ligne de la matrice co�t. Droite: un choix de permutation valide. } 
\end{figure}

Afin de satisfaire la contrainte d'approvisionnement (que l'on appelle aussi la conservation de la masse), il faut que chaque boulangerie soit connect�e � un et un seul caf�. Comme il y a le m�me nombre de boulangeries que de caf�, ceci implique que chaque caf� est �galement connect� � une et une seule boulangerie. On va noter 
\eq{    
    \si : \iC \in \{1,\ldots,6\} \longmapsto \jC \in \{1,\ldots,6\}
}
un tel choix de connexion. La figure~\ref{fig:image-cafe} illustre au centre et � droite l'exemple
\eql{\label{eq-bijection-exmp}
    \si({\color{red}1})={\color{blue}5}, \;
    \si({\color{red}2})={\color{blue}2}, \;
    \si({\color{red}3})={\color{blue}6}, \;
    \si({\color{red}4})={\color{blue}1}, \;
    \si({\color{red}5})={\color{blue}3}, \;
    \si({\color{red}6})={\color{blue}4}.
}  
La contrainte de conservation de masse signifie que $\si$ est une bijection de l'ensemble $\{1,\ldots,6\}$ dans lui-m�me. On dit ainsi que $\si$ est une permutation. 

Le co�t de transport associ� � une telle bijection est la somme des co�t $C_{\iC,\si(\iC)}$  s�lectionn�s par la permutation $\si$, c'est-�-dire 
\eq{
    \text{Co�t}(\si) \eqdef 
        C_{{\color{red}1},\si({\color{red}1})} + 
        C_{{\color{red}2},\si({\color{red}2})} + 
        C_{{\color{red}3},\si({\color{red}3})} + 
        C_{{\color{red}4},\si({\color{red}4})} + 
        C_{{\color{red}5},\si({\color{red}5})} + 
        C_{{\color{red}6},\si({\color{red}6})}. 
}
Par exemple, pour la bijection~\eqref{eq-bijection-exmp} montr�e � la figure~\ref{fig:image-cafe}, on obtient comme co�t
\eq{
    C_{{\color{red}1},{\color{blue}5}} + 
    C_{{\color{red}2},{\color{blue}2}} + 
    C_{{\color{red}3},{\color{blue}6}} + 
    C_{{\color{red}4},{\color{blue}1}} + 
    C_{{\color{red}5},{\color{blue}3}} + 
    C_{{\color{red}6},{\color{blue}4}} = 
    10 + 7 + 15 + 10 + 14 + 9 = 65. 
}

Le probl�me de Monge consiste � chercher la permutation $\si$ qui a le co�t minimum, c'est-�-dire r�soudre le probl�me d'optimisation
\eq{
    \umin{\si} \enscond{
        \text{Co�t}(\si)
    }{
        \si \in \text{Perm}(\{1,\ldots,6\})
    }
}
o� l'on a not� $\text{Perm}(\{1,\ldots,6\})$ l'ensemble des permutations de l'ensemble  $\{1,\ldots,6\})$.


\begin{figure}\centering
    \begin{tabular}{@{}c@{\hspace{1mm}}c@{\hspace{1mm}}c@{\hspace{1mm}}c@{}}
        \includegraphics[width=.22\linewidth]{transport/cafe-paris/ordre-croissant-64}&
        \includegraphics[width=.22\linewidth]{transport/cafe-paris/ordre-croissant-65}&
        \includegraphics[width=.22\linewidth]{transport/cafe-paris/ordre-croissant-66}&
        \includegraphics[width=.22\linewidth]{transport/cafe-paris/ordre-croissant-152}\\
        Co�t=64 &  
        Co�t=65 &  
        Co�t=66 &  
        Co�t=152
    \end{tabular}
    \caption{\label{fig:ordre-croissant} Choix de permutation avec diff�rent co�ts. } 
\end{figure}

La figure~\ref{fig:ordre-croissant} montre que la permutation~\eqref{eq-bijection-exmp} n'est pas la meilleure : il existe par exemple une autre permutation qui a un co�t de 64. Mais est-ce la meilleure ? Il se trouve que oui, on peut en effet tester sur un ordinateur toutes les permutations de  $\{1,\ldots,6\}$ et calculer leur co�t. Combien y a-t-il de permutation au total ? Il s'agit  


\begin{figure}\centering
    \includegraphics[width=.7\linewidth]{transport/metro/plan-metro}
    \caption{\label{fig:metro} Le transport optimal en 1D. } 
\end{figure}


\begin{figure}\centering
    \begin{tabular}{@{}c@{\hspace{1mm}}c@{\hspace{1mm}}c@{\hspace{1mm}}c@{}}
        \includegraphics[width=.22\linewidth]{transport/monge-2d/article-monge}&
        \includegraphics[width=.22\linewidth]{transport/monge-2d/decroisement}&
        \includegraphics[width=.22\linewidth]{transport/monge-2d/example-10}&
        \includegraphics[width=.22\linewidth]{transport/monge-2d/example-70}
    \end{tabular}
    \caption{\label{fig:ot2d} Le transport optimal en 2D. Gauche: extrait de l'article de Monge~\cite{Monge1781}. } 
\end{figure}


%%%%%%%%%%%%%%%%%%%%%%%%%%%%%%%%%%%%%%%%%%%%%%
\section{Le Transport Optimal de Kantorovitch}



\begin{figure}\centering
    \begin{tabular}{@{}c@{\hspace{1mm}}c@{\hspace{1mm}}c@{\hspace{1mm}}c@{}}
        \includegraphics[width=.22\linewidth]{transport/kantorovitch/coupling-array}&
        \includegraphics[width=.22\linewidth]{transport/kantorovitch/coupling-squares}&
        \includegraphics[width=.22\linewidth]{transport/kantorovitch/coupling-map}&
        \includegraphics[width=.22\linewidth]{transport/kantorovitch/coupling-bipartite}
    \end{tabular}
    \caption{\label{fig:coupling-visu} Diff�rentes fa�ons de repr�senter une matrice de couplage. } 
\end{figure}





\cite{Kantorovich42}

\cite{Dantzig51}

Th�or�me de birkhoff-von Neumann. 

\cite{birkhoff,von1953certain}

%%%%%%%%%%%%%%%%%%%%%%%%%%%%%%%%%%%%%%%%%%%%%%
\section{Les Applications}


\begin{figure}\centering
\begin{tabular}{@{}c@{\hspace{1mm}}c@{\hspace{1mm}}c@{}}
    \includegraphics[width=.24\linewidth]{transport/applis/painting-1} &
    \includegraphics[width=.24\linewidth]{transport/applis/painting-2} &
    \includegraphics[width=.24\linewidth]{transport/applis/painting-2-equalized} \\
    \includegraphics[width=.24\linewidth]{transport/applis/painting-1-histo}&
    \includegraphics[width=.24\linewidth]{transport/applis/painting-2-histo}&
    \includegraphics[width=.24\linewidth]{transport/applis/painting-1-histo}\\
    Toto & tata & tutu 
\end{tabular}
\caption{\label{fig:image-eq} Exemple de transfert d'histogrammes � l'aide du transport optimal. }
\end{figure}



\begin{figure}\centering
        \includegraphics[width=.6\linewidth]{transport/applis/shapes-3d}
    \caption{\label{fig:barycenters} Exemple d'interpolation barycentrique entre des formes 3D.  }
\end{figure}


\begin{figure}\centering
    \includegraphics[width=.35\linewidth]{transport/applis/bag-word-1}
    \qquad
    \includegraphics[width=.35\linewidth]{transport/applis/bag-word-2}
\caption{\label{fig:bagwords} Exemple d'interpolation barycentrique entre des formes 3D.  }
\end{figure}


\bibliographystyle{plain}
\bibliography{biblio-ot}

\end{document}

