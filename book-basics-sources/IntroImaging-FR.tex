\documentclass[11pt]{book}


% Be sure to use PDF Latex
\pdfoutput=1


% links
\usepackage[bookmarks,bookmarksdepth=2, colorlinks=true, linkcolor=blue,citecolor=red, urlcolor=blue]{hyperref}


\usepackage{fullpage}

% \usepackage[utf8]{inputenc}
%\usepackage[french]{babel}
\usepackage[latin1]{inputenc}

\usepackage{mystyle}
\renewcommand{\guill}[1]{«~#1~»} % french

\usepackage{url}

\usepackage[T1]{fontenc}

%\usepackage{vmargin} \setpapersize{A4}
%\newcommand{\mypage}{30mm}
% \setmarginsrb{\mypage{}}{\mypage{}}{\mypage{}}{\mypage{}}{0mm}{0mm}{0mm}{0mm}

\graphicspath{{./figures/},{./figures/sparsity/},{./figures/images/},{./figures/shannon/}}



\title{\Huge \sf Une introduction aux sciences des données\vspace{1cm}
\includegraphics[width=.9\linewidth]{wave-simple}} 

\author{%
\begin{tabular}{c}
	Gabriel Peyr{\'e} \\ CNRS \& DMA \\
	 \'Ecole Normale Sup\'erieure \\
	 \url{gabriel.peyre@ens.fr}\\
	 \url{https://mathematical-tours.github.io}
\end{tabular}
}


\date{\today}

\begin{document}

\maketitle

% !TEX root = ../IntroImaging-FR.tex

\chapter*{Pr�sentation}

Les trois chapitres de ce texte sont ind�pendants et pr�sentent des introductions en douceur � quelques fondements math�matiques importants des sciences de l'imagerie :
\begin{itemize}
	\item Le chapitre~\ref{chap-shannon} pr�sente la th�orie de Shannon sur la compression et insiste en particulier sur l'entropie li�e au codage de l'information.
	\item Le chapitre~\ref{chap-images} pr�sente les bases du traitement d'images, en particulier des traitements importants (quantification, d�bruitage, couleurs).
	\item Le chapitre~\ref{chap-sparsity} pr�sente la th�orie de l'�chantillonnage, allant de l'�chantillonnage classique de Shannon � l'�chantillonnage comprim�. Il constitue �galement une introduction � la r�gularisation des probl�mes inverses.
\end{itemize}
Le niveau d'exposition pour les deux premiers chapitres est �l�mentaire. Le dernier chapitre pr�sente des concepts et r�sultats math�matiques plus avanc�s.

\tableofcontents
% !TEX root = ../FundationsDataScience.tex

\chapter{Shannon Sampling Theory}
\label{sec-shannon}

%
Shannon's theory of information, published in 1948/1949, is made of three parts:
\begin{enumerate}
	\item Sampling: it studies conditions under which sampling a continuous function to obtain a discrete vector is invertible. The discrete real values representing the signal are then typically quantized to a finite precision to obtain a set of symbols in a finite alphabet.  
	\item Source coding: it studies optimal ways to represent (code) such a set of symbols as a binary sequence. It leverages the statistical distributions to obtain the most possible compact code.  
	\item Channel coding (not studied here): it studies adding some redundancy to the coded sequence to gain robustness to errors or attacks during transmission (flip of certain bits with some probability). It is often named ``error correcting codes theory''. 
\end{enumerate}
%
This chapter is focussed on the sampling theory and Chapter~\ref{sec-shannon-source} is dedicated to the source coding theory. We do not cover channel coding. 
%
The main reference for this chapter is~\cite{mallat2008wavelet}.


%%%%%%%%%%%%%%%%%%%%%%%%%%%%%%%%%%%%%%%%%%%%%%%%%%%%%%%%%%%%%%%%%%%%
%%%%%%%%%%%%%%%%%%%%%%%%%%%%%%%%%%%%%%%%%%%%%%%%%%%%%%%%%%%%%%%%%%%%
%%%%%%%%%%%%%%%%%%%%%%%%%%%%%%%%%%%%%%%%%%%%%%%%%%%%%%%%%%%%%%%%%%%%
\section{Analog vs. Discrete Signals}

To develop numerical tools and analyze their performances, the mathematical modeling is usually done over a continuous setting (so-called ``analog signals''). 
%
Such a continuous setting also aims at representing the signal in the physical world, which are inputs to sensor hardware such as microphones, digital cameras, or medical imaging devices. 
%
An analog signal is a 1-D function $f_0 \in \Ldeux([0,1])$ where $[0,1]$ denotes the domain of acquisition, which might for instance be time. An analog image is a 2D function $f_0 \in \Ldeux([0,1]^2)$ where the unit square $[0,1]^2$ is the image domain.

Although these notes are focused on the processing of sounds and natural images, most of the methods extend to multi-dimensional datasets, which are higher-dimensional mappings
\eq{
	f_0 : [0,1]^d \rightarrow [0,1]^s
}
where $d$ is the dimensionality of the input space ($d=1$ for sound and $d=2$ for images) whereas $s$ is the dimensionality of the feature space. For instance, grayscale images correspond to $(d=2,s=1)$, 
videos to $(d=3, s=1)$, color images to $(d=2, s=3)$ where one has three channels $(R,G,B)$.
One can even consider multi-spectral images where $(d=2, s \gg 3)$ is made of many channels for different light wavelengths. Figures \ref{fig-examples-1} and \ref{fig-examples-2} show examples of such data.


\myfigure{
	\image{orthobases}{.35}{example-sound}
	\image{orthobases}{.28}{example-image}
	\image{orthobases}{.3}{example-video}
}{%
	Examples of sounds ($d=1$), image ($d=2$) and videos ($d=3$). %	
}{fig-examples-1}

\myfigure{
	\image{orthobases}{.4}{example-color}
	\image{orthobases}{.5}{example-multispectral}
}{%
	Example of color image $s=3$ and multispectral image ($s=32$). %	
}{fig-examples-2}


%%%%%%%%%%%%%%%%%%%%%%%%%%%%%%%%%%%%%%%%%%%
\subsection{Acquisition and Sampling}

Signal acquisition is a low-dimensional projection of the continuous signal performed by some hardware device. This is for instance the case for a microphone that acquires 1D samples or a digital camera that acquires 2D pixel samples.
The sampling operation thus corresponds to mapping from the set of continuous functions to a discrete finite dimensional vector with $N$ entries.
\eq{
	f_0 \in \Ldeux([0,1]^d) \mapsto f \in \CC^N
}

\myfigure{
	\image{orthobases}{.4}{discretization-image}
	\image{orthobases}{.5}{discretization-sound}
}{%
	Image and sound discretization. %	
}{fig-discretization}

Figure \ref{fig-discretization} shows examples of discretized signals.

%%%%%%%%%%%%%%%%%%%%%%%%%%%%%%%%%%%%%%%%%%%
\subsection{Linear Translation Invariant Sampler}

A translation-invariant sampler performs the acquisition as an inner product between the continuous signal and a constant impulse response $h$ translated at the sample location
\eql{\label{eq-linear-sampling}
	f_n = \int_{-S/2}^{S/2} f_0(x) h(n/N - x) \d x= f_0 \star h(n/N).
}
The precise shape of $h(x)$ depends on the sampling device and is usually a smooth low-pass function that is maximal around $x=0$. The size $S$ of the sampler determines the precision of the sampling device and is usually of the order of $1/N$ to avoid blurring (if $S$ is too large) or aliasing (if $S$ is too small).

Section \ref{sec-sampling} details how to reverse the sampling operation in the case where the function is smooth.


%%%%%%%%%%%%%%%%%%%%%%%%%%%%%%%%%%%%%%%%%%%%%%%%%%%%%%%%%%%%%%%%%%%%
%%%%%%%%%%%%%%%%%%%%%%%%%%%%%%%%%%%%%%%%%%%%%%%%%%%%%%%%%%%%%%%%%%%%
%%%%%%%%%%%%%%%%%%%%%%%%%%%%%%%%%%%%%%%%%%%%%%%%%%%%%%%%%%%%%%%%%%%%
\section{Shannon Sampling Theorem}
\label{subsec-sampling}

%%%
\paragraph{Reminders about Fourier transform.}

For $f \in L^1(\RR)$, its Fourier transform is defined as
\eql{\label{eq-fourier-transform}
	\foralls \om \in \RR, \quad
	\hat f(\om) \eqdef \int_\RR f(x) e^{-\imath x \om} \d x.
}
One has $\norm{\hat f}^2 = (2\pi)^{-1} \norm{f}^2$, so that $f \mapsto \hat f$ can be extended by continuity to $L^2(\RR)$, which corresponds to computing $\hat f$ as a limit when $T \rightarrow +\infty$ of $\int_{-T}^T f(x) e^{-\imath x \om} \d x$.
%
When $\hat f \in L^1(\RR)$, one can invert the Fourier transform so that
\eql{\label{eq-i-ft}
	f(x) = \frac{1}{2\pi} \int_\RR \hat f(\om) e^{\imath x \om} \d \om, 
}
which shows in particular that $f$ is continuous with vanishing limits at $\pm\infty$. 

The Fourier transform $\Ff : f \mapsto \hat f$ exchanges regularity and decay. For instance, if $f \in C^p(\RR)$ with an integrable Fourier transform, then $\Ff(f^{(p)})(\om) = (\imath \om)^{p} \hat f(\om)$ so that $|\hat f(\om)|=O(1/|\om|^p)$. 
%
Conversely, 
\eql{\label{eq-fourier-regul}
	\int_\RR (1+|\om|)^{p} |\hat f(\om)| \d \om<+\infty
	\qarrq f \in C^p(\RR).
}
For instance, if $\hat f(\om)=O(1/|\om|^{p+1+\epsilon})$ for $\epsilon>0$, one obtains that $f \in C^p(\RR)$. 

A related, but different way to impose smoothness is by using the Sobolev $H^1$ norm $\int_\RR (1+|\om|^2) |\hat f(\om)|^2 \d \om$, which, when $f$ is smooth, is equal to $\|f\|_2^2 + \|f'\|_2^2$. It is a fundamental space to define weak solutions (non smooth) to PDE's, and also to control compression errors in image process. We will not pursue this here.

%%%
\paragraph{Reminders about Fourier series.}

We denote $\TT=\RR/2\pi\ZZ$ the torus.
%
A function $f \in L^2(\TT)$ is $2\pi$-periodic, and can be viewed as a function $f \in L^2([0,2\pi])$ (beware that this means that the boundary points are glued together), and its Fourier coefficients are
\eq{
	\foralls k \in \ZZ, \quad 
	\hat f_k \eqdef \frac{1}{2\pi}\int_0^{2\pi} f(x) e^{-\imath x k} \d x.
}
This formula is equivalent to the computation of an inner-product $\hat f_k = \dotp{f}{e_k}$ for the inner-product $\dotp{f}{g} \eqdef \frac{1}{2\pi} \int_\TT f(x) \bar g(x) \d x$ and for $e_k(x) \eqdef e^{\imath x k}$. 
%
For this inner product, $(e_k)_k$ is orthonormal and is a Hilbert basis, meaning that one reconstructs with the following converging series 
\eql{\label{eq-fourier-series}
	f = \sum_{n \in \ZZ} \dotp{f}{e_k} e_k
}
which means $\norm{f-\sum_{k=-N}^N \dotp{f}{e_k} e_k}_{L^2(\TT)} \rightarrow 0$ for $N \rightarrow +\infty$.
%
The pointwise convergence of~\eqref{eq-fourier-series} at some $x \in \TT$ is ensured if, for instance, $f$ is differentiable. The series is normally convergent (and hence uniform) if for instance $f$ if of class $C^2$ on $\TT$ since in this case, $\hat f_k = O(1/n^2)$. 
%
If there is a step discontinuity, then there are Gibbs oscillations preventing uniform convergence, but the series still converges to half of the left and right limits.


%%%
\paragraph{Poisson formula.}

The Poisson formula connects the Fourier transform and the Fourier series to sampling and periodization operators.
%
For some function $h(\om)$ defined on $\RR$ (typically the goal is to apply this to $h=\hat f$), its periodization reads
\eql{\label{eq-periodizing}
	h_P(\om) \eqdef \sum_n h(\om-2\pi n).
} 
This formula makes sense if $h \in L^1(\RR)$, and in this case $\norm{h_P}_{L^1(\TT)} \leq \norm{h}_{L^1(\RR)}$ (and there is equality for positive functions). Indeed, one has
$$
	|h_P(x)| \leq (|h|_P)(x), \quad\Longrightarrow\quad
	\norm{h_P}_{L^1} \leq  \norm{|h|_P}_{L^1} = \norm{h}_{L^1}. 
$$
%
The Poisson formula, stated in Proposition~\ref{prop-poisson} below, corresponds to proving that the following diagram
\eq{
	\begin{array}{rcccl}
						& f(x)  &  \overset{\Ff}{\longrightarrow} &  \hat f(\om) &\\
		\text{sampling}& \downarrow & & \downarrow &\text{periodization} \\
						& (f(n))_n  &  \overset{\text{Fourier serie}}{\longrightarrow} &  \sum_n f(n) e^{-\imath \om n} &\\
	\end{array}
}
is commutative. Beware that $\sum_n f(n) e^{-\imath \om n}$ is actually a reverse Fourier series (there is a + sign in the reconstruction formula for Fourier series). 

\begin{prop}[Poisson formula]\label{prop-poisson}
Assume that $\hat f$ has compact support and that $|f(x)| \leq C(1+|x|)^{-3}$ for some $C$. Then one has 
\eql{\label{eq-poisson-formula}
	\foralls \om \in \RR, \quad
	\sum_n f(n) e^{-\imath \om n} = \hat f_P(\om).
}
\end{prop}
\begin{proof}
	Since $\hat f$ is compactly supported, $\hat f_P$ is well defined (it involves only a finite sum) and since $f$ has fast decay, using~\eqref{eq-fourier-regul}, $(\hat f)_P$ is $C^1$. It is thus the sum of its Fourier series
	\eql{\label{eq-poisson-formula}
		(\hat f)_P(\om) = \sum_k c_k e^{\imath k \om},
	} 
	where
	\begin{align*}
		c_k = \frac{1}{2\pi} \int_0^{2\pi} (\hat f)_P(\om) e^{-\imath k \om} \d \om = 
		\frac{1}{2\pi} \int_0^{2\pi} \sum_n \hat f(\om-2\pi n) e^{-\imath k \om}  \d \om .
	\end{align*}
	One has 
	\eq{
		\int_0^{2\pi} \sum_n |\hat f(\om-2\pi n) e^{-\imath k \om}|  \d \om = \int_\RR |\hat f| 
	}
	which is bounded because $\hat f \in L^1(\RR)$ (it has a compact support and is $C^1$), so one can exchange the sum and integral
	\eq{
		c_k = \sum_n \frac{1}{2\pi} \int_0^{2\pi} \hat f(\om-2\pi  n) e^{-\imath k \om}  \d \om
		= \frac{1}{2\pi} \int_{\RR} \hat f(\om) e^{-\imath k \om}  \d \om
		= f(-k)
	}
	where we used the inverse Fourier transform formula~\eqref{eq-i-ft}, which is legit because $\hat f \in L^1(\RR)$.
\end{proof}

%%%
\paragraph{Shannon theorem.}

Shannon's sampling theorem states a sufficient condition ensuring that the sampling operator $f \mapsto (f(ns))_n$ is invertible for some sampling step size $s>0$. 
%
It require that $\supp(\hat f) \subset [-\pi/s,\pi/s]$, which, thanks to formula~\eqref{eq-i-ft}, implies that $\hat f$ is $C^\infty$ (in fact it is even analytic). 
%
This theorem was first proved by Whittaker in 1915. It was re-proved and put in perspective in electrical engineering by Nyquist in 1928. It became famous after the paper of Shannon in 1949, which put forward its importance in numerical communications.
%
Figure~\ref{fig-sampling-aliasing} gives some insight on how the proof works (left) and more importantly, on what happens when the compact support hypothesis fails (in which case aliasing occurs, see also Figure~\ref{fig-aliasing}). 

\myfigure{
	\image{1-shannon}{.8}{sampling-aliasing}
}{%
	Schematic view for the proof of Theorem~\ref{thm-shannon-sampling}. %	
}{fig-sampling-aliasing}





\begin{thm} \label{thm-shannon-sampling}
	If $|f(x)| \leq C(1+|x|)^{-3}$ for some $C$ and $\supp(\hat f) \subset [-\pi/s,\pi/s]$, then one has
	\eql{\label{eq-shannong-interp}
		\foralls x \in \RR, \quad 
		f(x) = \sum_n f(n s) \sinc(x/s-n) \qwhereq
		\sinc(u) = \frac{\sin(\pi u)}{\pi u}
	}
	with uniform convergence.
\end{thm}

\begin{proof} 
	The change of variable $g \eqdef f(s \cdot)$ results in $\hat g=1/s \hat f(\cdot/s)$, indeed, denoting $z=s x$
	\eq{
		\hat g(\om) = \int f(s x) e^{-\imath \om x} \d x = \frac{1}{s} \int f(z) e^{-\imath (\om/s) z} \d z = \hat f(\om/s)/s, 
	} 
	so that we can restrict our attention to $s=1$. With this change of variable, we thus need to prove that
	$$
		g(x) = \sum_n g(n) \sinc(x-n), 
	$$
	(and in the following, we keep using the notation $f=g$).
	%
	The compact support hypothesis implies $\hat f(\om) = 1_{[-\pi,\pi]}(\om) \hat f_P(\om)$.  
	Combining the inversion formula~\eqref{eq-i-ft} with Poisson formula~\eqref{eq-poisson-formula}
	\eq{
		f(x) = \frac{1}{2\pi} \int_{-\pi}^\pi \hat f_P(\om) e^{\imath \om x} \d \om
		= \frac{1}{2\pi} \int_{-\pi}^\pi \sum_n f(n) e^{\imath \om (x-n)} \d \om.
	} 
	Since $f$ has fast decay, $\int_{-\pi}^\pi \sum_n |f(n) e^{\imath \om (x-n)}| \d \om = \sum_n |f(n)| < +\infty$, so that one can exchange summation and integration and obtain
	\eq{
		f(x) = \sum_n f(n)  \frac{1}{2\pi} \int_{-\pi}^\pi e^{\imath \om (x-n)} \d \om = \sum_n f(n) \sinc(x-n).
	}
\end{proof}

\wrapf{1-shannon/sinc}{sinc kernel}
One issue with this reconstruction formula is that it uses slowly decaying and very oscillating $\sinc$ kernels. In practice, one rarely uses such a kernel for interpolation, and one prefers a smoother and more localized kernel. If $\supp(\hat f) \subset [-\pi/s',\pi/s']$ with $s'>s$ (i.e. have a more compact spectrum), one can re-do the proof of the theorem, and one gains some degree of freedom to design the reconstruction kernel, which now can be chosen smoother in Fourier and hence have exponential decay in time. 

%
Spline interpolation are defined by considering $\phi_0=1_{[-1/2,1/2]}$ and $\phi_k = \phi_{k-1} \star \phi_0$ which is a piecewise polynomial of degree $k$ and has bounded derivative of order $k$ (and is of class $C^{k-1}$) with compact support on $[-(k+1)/2,(k+1)/2]$. The reconstruction formula reads $f \approx \tilde f \eqdef \sum_n a_n \phi(\cdot-n)$ where $(a_n)_n$ is computed from the $(f(n))_n$ by solving a linear system (associated to the interpolation property $\tilde f(n)=f(n)$). It is only in the cases $k \in \{0,1\}$ (piecewise constant and affine interpolations) that one has $a_n=f(n)$.
%
In practice, one typically use the cubic spline interpolation, which corresponds to $k=3$.

\texttt{Associated code: test\_sampling.m}

\myfigure{
	\image{1-shannon}{.6}{spline}
}{%
	Cardinal splines as basis functions for interpolation. %	
}{fig-aliasing}



This theorem also explains what happens if $\hat f$ is not supported in $[-\pi/s,\pi/s]$. This leads to aliasing, and high frequency outside this interval leads to low frequency artifacts often referred to as ``aliasing''. If the input signal is not bandlimited, it is thus very important to pre-filter it (smooth it) before sampling to avoid these phenomena (of course this kills the high frequencies, which are lost), see Figure~\ref{fig-aliasing}. 

\myfigure{
	\image{1-shannon}{.6}{aliasing}
}{%
	Aliasing in the simple case of a sine wave (beware however that this function does not have compact support). %	
}{fig-aliasing}

%%%
\paragraph{Quantization.}

Once the signal have been sampled to obtain a discrete vector, in order to store it and transmit it, it is necessary to quantize the value to some finite precision. 
% 
Section~\ref{sec-transform-coding} presents transform coding, which is an efficient family of compression schemes which operates the quantization over some transformed domain (which corresponds to applying a linear transform, usually orthogonal, to the sampled values). This is useful to enhance the performance of the source coding scheme. It is however common to operate directly the quantization over the sampled value. 

Considering for instance a step size $s=1/N$, one samples $(u_n \eqdef f(n/N))_{n=1}^{N} \in \RR^N$ to obtain a finite dimensional data vector of length $N$. Note that dealing with finite data corresponds to restricting the function $f$ to some compact domain (here $[0,1]$) and is contradictory with the Shannon sampling theorem since a function $f$ cannot have compact support in both space and frequency (so perfect reconstruction never holds when using finite storage).

\wrapf{1-shannon/quantizer}{}

Choosing a quantization step $T$, quantization $v_n = Q_T(u_n) \in \ZZ$ rounds to the nearest multiple of $T$, i.e. 
\eq{
	v = Q_T(u) \quad\Leftrightarrow\quad
	v-\frac{1}{2} \leq u/T < v+\frac{1}{2},
}
see Fig.~\ref{fig-quantizer}. De-quantization is needed to restore a signal, and the best reconstruction (in average or in worse case) is defined by setting $D_T(v) \eqdef T v$. Quantizing and then de-quantizing introduce an error bounded by $T/2$, since $|D_T(Q_T(u))-u| \leq T/2$. 
%
Up to machine precision, quantization is the only source of error (often called ``lossy compression'') in Shannon's standard pipeline.


\myfigure{
\includegraphics[width=.8\linewidth]{1-shannon/quantize/quantization}
%\tabquatre{
%\includegraphics[width=.2\linewidth]{1-shannon/quantize/quantize-2}&
%\includegraphics[width=.2\linewidth]{1-shannon/quantize/quantize-3}&
%\includegraphics[width=.2\linewidth]{1-shannon/quantize/quantize-4}&
%\includegraphics[width=.2\linewidth]{1-shannon/quantize/quantize-16}\\
%$2$ graylevels &
%$3$ graylevels &
%$4$ graylevels &
%$16$ graylevels 
%}
}{Quantizing an image using a decaying $T=1/K$ where $K \in \{2,3,4,16\}$ is the number of graylevels and the original image is normalized so that $0 \leq f_0 < 1$. 
}{fig-section3-quantize}





% !TEX root = ../IntroImaging.tex

\chapter{Image Processing}
\label{chap-images}

\newcommand{\FootLink}[1]{\footnote{\url{#1}}}
\newcommand{\myparagraph}[1]{\vspace{2mm}\noindent\textbf{#1}}
\newcommand{\lien}[2]{#2}

%\newcommand{\bs}{\_}

\newcommand{\BetweenImageSpace}{\hspace{3mm}}

\newcommand{\image}[1]{\includegraphics[width=0.32\linewidth]{#1}}
\newcommand{\imageM}[1]{\includegraphics[width=0.4\linewidth]{#1}}
\newcommand{\imageL}[1]{\includegraphics[width=0.6\linewidth]{#1}}
\newcommand{\imageQuad}[8]{
\begin{tabular}{@{}c@{\BetweenImageSpace{}}c@{}}
\image{#1} & \image{#2}\\
#5 & #6 \\
\image{#3} & \image{#4}\\
#5 & #6 
\end{tabular}
}
\newcommand{\imageTri}[6]{
\begin{tabular}{@{}c@{\BetweenImageSpace{}}c@{\BetweenImageSpace{}}c@{}}
\image{#1} & \image{#2} & \image{#3}\\
#4 & #5 & #6 \\
\end{tabular}
}
\newcommand{\imageDuo}[4]{
\begin{tabular}{@{}c@{\BetweenImageSpace{}}c@{}}
\image{#1} & \image{#2} \\
#3 & #4  \\
\end{tabular}
}

\newcommand{\myfigure}[3]{
    \begin{figure}[ht]
    \begin{center}
        #1                          % usual include graphics
    \end{center}
    \vspace{-4mm}
        \caption{\textit{#2}}       % caption
    \label{#3}          % label
    \end{figure}
}



Digital cameras take precise photographs of the world around us. The user wants to be able to store his photos on his hard drive with minimum memory requirement. He also wishes to be able to reprocess them in order to improve their quality. This article presents the mathematical and computer tools used to perform these different tasks. 

%%%%%%%%%%%%%%%%%%%%%%%%%%%%%%%%%%%%%%%%%%%%%%%%%%%%%%%%%%%%%%
%%%%%%%%%%%%%%%%%%%%%%%%%%%%%%%%%%%%%%%%%%%%%%%%%%%%%%%%%%%%%%
%%%%%%%%%%%%%%%%%%%%%%%%%%%%%%%%%%%%%%%%%%%%%%%%%%%%%%%%%%%%%%
\section{The pixels of an image}

A \lien{https://en.wikipedia.org/wiki/Digital_image}{digital image}
in gray levels is an array of values. Each
box of this table, which stores a value, is called a
\lien{https://en.wikipedia.org/wiki/Pixel}{pixel}.
By noting $n$ the number of rows and $p$ the number of columns in the image,
we manipulate an array of $n \times p$ pixels. Figure \ref{fig-section1-original-zoom}, left, shows a visualization of a square table with $n = p = 240$, which represents $240\times 240$ = 57600 pixels. The
\lien{https://en.wikipedia.org/wiki/Digital_camera}{digital cameras}
can record much larger images,
with several millions of pixels.

The values of the pixels are stored in
\lien{http://en.wikipedia.org/wiki/Computer}{a computer} or
a digital camera in the form of
\lien{https://en.wikipedia.org/wiki/Integer}{relative integers} between entre 0 et $255=2^8-1$,
making 256 possible values for each pixel. The value 0 is black, and the value 255 is white. The intermediate values correspond to
\lien{https://en.wikipedia.org/wiki/Grayscale}{gray levels}
ranging from black to white. Figure \ref{fig-section1-original-zoom} shows a subset of $6 \times 6$ pixels taken from the previous image. You can see both the values that make up the table and the gray levels that allow you to display the image on the screen.

\myfigure{
\imageL{section1-original-zoom}
}{Sub-image of size $5 \times 5$}{fig-section1-original-zoom}

%%%%%%%%%%%%%%%%%%%%%%%%%%%%%%%%%%%%%%%%%%%%%%%%%% %%%%%%%%%%%%
%%%%%%%%%%%%%%%%%%%%%%%%%%%%%%%%%%%%%%%%%%%%%%%%%% %%%%%%%%%%%%
\section{Image Storage}

%%%%%%%%%%%%%%%%%%%%%%%%%%%%%%%%%%%%%%%%%%%%%%%%%% %%%%%%%%%%%%
\subsection{Binary Codes}

Storing large images on the
\lien{https://en.wikipedia.org/wiki/Hard_disk_drive}{hard drive}
of a computer takes
a significant amount of places. Integer numbers are stored
in \lien{https://en.wikipedia.org/wiki/Binary_number}{binary},
in the form of a succession
of 0 and 1. Each 0 and each 1 corresponds to an elementary unit
of information, called bit.
To obtain the binary expression of a pixel having the value 179,
it is necessary to decompose this value as a sum of powers of two.
We thus obtain
\[ 
	179=2^7+2^5+2^4+2+1, 
\]
where care has been taken to order the powers of two in decaying order. In order to make the binary more explicit,
we add ``$1 \times$" before each power that appears in the expression,
and ``$0\times$" before the powers that do not appear
\[ 179=1 \times 2^7 + 0 \times 2^6 + 1 \times 2^5 + 1 \times 2^4 + 
  0 \times 2^3 + 0 \times 2^2 + 1 \times 2^1 + 1 \times 2^0. \]
Using such a decomposition, the value of each pixel, which is a number between 0 and 255, requires
$\log_2(256) = 8\text{bits}$. 
The binary writing of the value 179 of the pixel is thus $(1,0,1,1,0,0,1,1)$.
Any value between 0 and 255 can be written in this way, which requires the use of 8 bits. Indeed,
there are 256 possible values, and $256 = 2^8$. To store the complete image, it is therefore necessary to use $n \times p \times 8 \text{bits}$.
For the image shown in the previous figures, it is thus necessary to use
\[
	256 \times 256 \times 8 = 524288 \text{bits}. 
\]
Equivalently, this image requires 57.6kb (kilobytes), since a kilobyte is equal to $8$ bits.


%%%%%%%%%%%%%%%%%%%%%%%%%%%%%%%%%%%%%%%%%%%%%%%%%% %%%%%%%%%%%%
\subsection{Sub-sampling an Image}


\myfigure{
\imageTri
%{section2-subsample-2}
{section2-subsample-4}
{section2-subsample-8}
{section2-subsample-16}
%{Une ligne/colonne sur 2}
{One row / column of 4}
{One row / column out of 8}
{One row / column of 16}
}{Subsampling of an image}{fig-section2-subsample}

In order to reduce the required storage space of an image, the number of pixels can be reduced.
The easiest way to do this is to delete rows and columns in the original image. Figure \ref{fig-section2-subsample}, at the top left, shows what is obtained if one row is kept out of 4 and one column out of 4. We thus have divided by $4 \times 4 = 16$ the number of pixels of the image, and thus also divided by 16 the number of bits required to store the image on
a hard disc. In fig. \ref{fig-section2-subsample}, one can see the results obtained by removing more and more rows and columns. Of course, the quality of the image degrades quickly.

%%%%%%%%%%%%%%%%%%%%%%%%%%%%%%%%%%%%%%%%%%%%%%%%%% %%%%%%%%%%%%
\subsection{Quantizing an image}

Another way to reduce the memory space required for storage
is to use fewer integers for each value. For example, we can use only integers between 0 and 3, which will give an image with only 4 levels of gray. One can convert the original image to an image with 4 levels of values by performing the replacements:
\begin{rs}
\item the values in $0,1,\ldots, 63$ are replaced by the value 0 (black),
\item the values in $64,1,\ldots, 127$ are replaced by the value 1 (light gray),
The values in $128.1,\ldots, 191$ are replaced by the value 2 (dark gray),
\item the values in $192,\ldots, 255$ are replaced by the value 3 (white).
\end{rs}
Such an operation is called \lien{https://en.wikipedia.org/wiki/Quantization}{quantization}. Figure~\ref{fig-section3-quantize}, in the center, shows the resulting image with 4 grayscale levels.

We have already seen that we can represent any value between 0 and 255 using 8 bits using binary coding. In the same way,
one can check that any value between 0 and 3 can be represented using 2 bits.
This thus results in a reduction of a factor 8/2=4 of the memory footprint necessary
for storing the image on a hard disk. Figure \ref{fig-section3-quantize} shows the results obtained using less and less gray levels.

\myfigure{
\imageDuo
%{section3-quantize-2}
%{section3-quantize-3}
{section3-quantize-4}
{section3-quantize-16}
%{16 niveaux de gris}
%{3 niveaux de gris}
{4 niveaux de gris}
{16 niveaux de gris}
}{Quantizing an image
}{fig-section3-quantize}
 
As with the reduction of the number of pixels, the reduction of the number
gray levels greatly affects the quality of the image.
In order to minimize the size of an image without changing its quality,
more complex methods of
\lien{https://en.wikipedia.org/wiki/Image_compression}{image compression}.
The most effective method is JPEG-2000. It uses the theory of wavelets.




%%%%%%%%%%%%%%%%%%%%%%%%%%%%%%%%%%%%%%%%%%%%%%%%%% %%%%%%%%%%%%
%%%%%%%%%%%%%%%%%%%%%%%%%%%%%%%%%%%%%%%%%%%%%%%%%% %%%%%%%%%%%%
%%%%%%%%%%%%%%%%%%%%%%%%%%%%%%%%%%%%%%%%%%%%%%%%%% %%%%%%%%%%%%
\section{Noise Removal}

%%%%%%%%%%%%%%%%%%%%%%%%%%%%%%%%%%%%%%%%%%%%%%%%%% %%%%%%%%%%%%
\subsection{Local Averaging}

Images are sometimes of poor quality. A typical example of a defect
is the noise which appears when a picture is under-exposed,
that is if there is not enough light. This noise corresponds to small
random fluctuations of the gray levels. Figure~\ref{fig-section5-moyenne}, on the left, shows such 
a noisy image.


\myfigure{
\imageM{section5-moyenne-numbers}
}{Pixels neighborhood.
}{fig-section5-moyenne-numbers}

In order to remove the noise in the images, it is necessary to
modify the pixel values.
The simplest operation is to replace the value
$a$ of each pixel by the average of
$a$ and the values $b, c, d, e, f, g, h, i$ of the 8 pixels that surround $a$. Figure \ref{fig-section5-moyenne-numbers} shows an example of a neighborhood of 9 pixels. A modified image is thus obtained by replacing $a$ by
\[\frac{a + b + c + d + e + f + g + h + i}{9}, \]
since the average of 9 values is averaged.
%
In our example, this average is
\[ \frac{190+192+79+54+47+153+203+189+166}{9} \approx 141. \]
By performing this operation for each pixel, a large part of 
the noise is removed, because this noise is made up of random fluctuations, which are
decreased by averaging. Figure \ref{fig-section5-moyenne}, top left, shows the effect of such a calculation. All the noise was not removed by this operation. In order to remove more
of noise, one can average more values around each pixel.
%
Figure \ref{fig-section5-moyenne} shows the result obtained by increasing the average
of values.


\myfigure{
\imageTri
{section5-moyenne-1}
{section5-moyenne-2}
{section5-moyenne-3}
%{section5-moyenne-4}
{Mean on 9 pixels}
{Mean on 25 pixels}
{Mean on 49 pixels}
%{Moyenne sur 81 pixels}
}{Mean with increasing width
}{fig-section5-moyenne}

Pixel averaging is very effective in removing noise in
images, unfortunately it also destroys much of the
the information of the image. It can indeed be seen that the images
obtained by averaging are blurry.
This is particularly visible near contours, which are not sharp.

%%%%%%%%%%%%%%%%%%%%%%%%%%%%%%%%%%%%%%%%%%%%%%%%%% %%%%%%%%%%%%
\subsection{Local Median}

To reduce this blur, the mean can be replaced by the median. In the example of the neighborhood of 9 pixels used in the previous section, the 9 sorted values are:
\[47, 54, 79, 153, 166, 189, 190, 192, 203. \]
The median of these nine values is 166. In order to remove more noise, it is enough to calculate the median over a larger number of neighboring pixels, as shown in Figure \ref{fig-section6-mediane}.
One can observe that this method is more efficient than the mean calculation 
because the resulting images are less blurry. However, just as
with the calculation of averages, if we take neighborhoods that are too large, we lose
also information of the image, especially the edges of the objects are degraded.


\myfigure{
\imageTri
{section6-mediane-1}
{section6-mediane-2}
{section6-mediane-3}
%{section6-mediane-4}
{Median on 9 pixels}
{Median on 25 pixels}
{Median on 49 pixels}
%{M�diane sur 81 pixels}
}{Median filtering with increasing windowing size.
}{fig-section6-mediane}




%%%%%%%%%%%%%%%%%%%%%%%%%%%%%%%%%%%%%%%%%%%%%%%%%% %%%%%%%%%%%%
%%%%%%%%%%%%%%%%%%%%%%%%%%%%%%%%%%%%%%%%%%%%%%%%%% %%%%%%%%%%%%
%%%%%%%%%%%%%%%%%%%%%%%%%%%%%%%%%%%%%%%%%%%%%%%%%% %%%%%%%%%%%%
\section{Detecting Edges of Objects}

In order to locate objects in the images, it is necessary to
detect their edges. These edges correspond to
areas of the image where pixel values change rapidly. It is the case
for example when passing from the boat (which is dark,
ie. with small values) to the sea (which is clear, therefore with
large values).

In order to know if a pixel with a value $a$ is along an edge of an object, the values $b, c, d, e$ of its four neighbors are taken into account, which have a common side with it (Figure \ref{fig-section7-contours-numbers}). This allows the edges of objects to be detected as accurately as possible.


\myfigure{
\imageM{section7-contours-numbers}
}{Example of a neighborhood of 5 pixels.
}{fig-section7-contours-numbers}


A value $\ell$ is calculated according to the formula
\[ \ell = \sqrt{ (b-d)^2 + (c-e)^2 }.  \]
In our example, we thus obtain
\[ \ell= \sqrt{�(192 - 153)^2 + (189 - 54)^2 } = \sqrt{19746} \approx 141. \]
It may be noted that if $\ell = 0$, then $b = c$
and $d = e$. On the contrary, if
$\ell$ is large, this means that the neighboring pixels have very high values
different, the pixel considered is therefore probably on the edge of an object.

Figure \ref{fig-section7-contours} shows an image whose pixel value is $\min(\ell, 255)$.
It is necessary to take the minium with 255, because the value of $\ell$ can exceed the maximum displayable value (255, which corresponds to white). These values are displayed with black when $\ell = 0$, in white when $\ell$ is high, and gray levels are used for the intermediate values. It can be seen that in the image on the right, the outlines of the objects appear white, as they correspond to  large values of $\ell$.

\myfigure{
\imageDuo
{section7-image}
{section7-contours}
{Original Image}
{Contour map $\ell$}
}{Edge detection.
}{fig-section7-contours}

%%%%%%%%%%%%%%%%%%%%%%%%%%%%%%%%%%%%%%%%%%%%%%%%%%%%%%%%%%%%%%
%%%%%%%%%%%%%%%%%%%%%%%%%%%%%%%%%%%%%%%%%%%%%%%%%%%%%%%%%%%%%%
%%%%%%%%%%%%%%%%%%%%%%%%%%%%%%%%%%%%%%%%%%%%%%%%%%%%%%%%%%%%%%
\section{Color Images}


%%%%%%%%%%%%%%%%%%%%%%%%%%%%%%%%%%%%%%%%%%%%%%%%%% %%%%%%%%%%%%
\subsection{RGB Space}

A
\lien{https://en.wikipedia.org/wiki/Color_space}{color image}
is actually composed of three independent images,
in order to represent the
\lien{https://en.wikipedia.org/wiki/RGB_color_model}{red, green, and blue}.
Each of these three
images is called a
\lien{https://en.wikipedia.org/wiki/Channel_(digital_image)}{color channel}.
This representation in red, green and blue mimics the
human visual system.
Figure~\ref{fig-canaux-coul} shows the three constituent channels of the image shown on the left of Figure~\ref{fig-luminance}.

\myfigure{
\imageDuo
{section8-image}
{section8-luminance}
{Original image}
{Luminance}
}{Color image.
}{fig-luminance}


Each pixel of the color image thus contains three numbers $(r, v, b)$,
each being an integer between 0 and 255.
If the pixel is equal to $(r, v, b) = (255,0,0)$, it contains only information
red, and is displayed as red.
Similarly, the pixels of $(0,255,0)$ and $(0,0,255)$ are
respectively displayed green and blue.


\myfigure{
\imageTri
{section8-rouge}
{section8-vert}
{section8-bleu}
{Red canal }
{Green canal}
{Blue canal}
}{Color channels
}{fig-canaux-coul}


A color image can be displayed on the screen.
from its three channels $(r, v, b)$ using the rules of
\lien{https://en.wikipedia.org/wiki/Additive_color}{additive color synthesis}. These rules correspond to the way in which light rays combine, hence the qualifier \guill{additive}.
Figure \ref{fig-section8-synthese}, left, shows the composition rules
this additive synthesis of colors.
For example, a pixel with the values
$(r, v, b) = (255,0,255)$ is a mixture of red and green,
displayed as yellow.


\myfigure{
\imageDuo
{section8-synthese-additive}
{section8-synthese-soustractive}
{Additive synthesis}
{Subtractive synthesis}
}{Color Synthesis
}{fig-section8-synthese}


%%%%%%%%%%%%%%%%%%%%%%%%%%%%%%%%%%%%%%%%%%%%%%%%%% %%%%%%%%%%%%
\subsection{CMJ Space}

Another common representation for color images uses
as background colors cyan, magenta and yellow. It is calculated
the three numbers $(c, m, j)$ corresponding to each of these three channels
from the red, green and blue channels $(r, v, b)$ as follows
\[c = 255-r, \quad m = 255-v, \quad j = 255-b. \]
For example, a pixel of pure blue
$(r, v, b) = (0,0,255)$ will become
$(c, m, j) = (255,255,0)$. Figure~\ref{fig-canaux-cmj} shows the three channels
$(c, m, j)$ of a color image.

\myfigure{
\imageTri
{section8-cyan}
{section8-magenta}
{section8-jaune}
{Cyan canal}
{Magenta canal}
{Yellow canal}
}{Channels CMJ
}{fig-canaux-cmj}


In order to display a color image on the screen from the three channels $(c, m, j)$, the \lien{https://en.wikipedia.org/wiki/Subtractive_color}{subtractive color synthesis} must be used. Figure \ref{fig-section8-synthese}, right, shows the composition rules of this subtractive synthesis. They correspond in painting to the absorption of light by colored pigments, hence the qualifier \guill{subtractive}. Cyan, magenta and yellow are called primary colors.

It is thus possible to store on a hard disk a color image by storing the
three channels, corresponding to the $(r, g, b)$ or $(c, m, j)$.
%
One can change color images in a similar way as graylevel image, by changing each channel.


%%%%%%%%%%%%%%%%%%%%%%%%%%%%%%%%%%%%%%%%%%%%%%%%%%%%%%%%%%%%%%
%%%%%%%%%%%%%%%%%%%%%%%%%%%%%%%%%%%%%%%%%%%%%%%%%%%%%%%%%%%%%%
%%%%%%%%%%%%%%%%%%%%%%%%%%%%%%%%%%%%%%%%%%%%%%%%%%%%%%%%%%%%%%
\section{Changing the Contrast of an Image}

%%%%%%%%%%%%%%%%%%%%%%%%%%%%%%%%%%%%%%%%%%%%%%%%%% %%%%%%%%%%%%
\subsection{Luminance}

One calculates a grayscale image from a color image
as the mean of the three channels. Thus, for each pixel, a value
\[a = \frac{r + v + b}{3} \]
is computed which is called luminance
of the color. Figure \ref{fig-luminance} shows the transition from a color image to a luminance image in grayscales. 


%%%%%%%%%%%%%%%%%%%%%%%%%%%%%%%%%%%%%%%%%%%%%%%%%% %%%%%%%%%%%%
\subsection{Grayscale contrast manipulations}

It is possible to make various changes to the image in order to
modify his contrast. We consider here a grayscale image.
A simple manipulation consists in replacing each value $a$ of a pixel of an image by $255-a$, which corresponds to the opposite gray intensity. The white becomes black and vice-et-versa, giving a similar effect to that
of the negative of film for cameras, see figure \ref{fig-section4-square}, left.

The image is lightened or darkened using an increasing function from $[0,255]$ to itself, which is applied to the $a$ values of the pixels. One can darken the image by using the square function. More precisely, we define the new value of a pixel of the image as $a^2/255$ (see figure \ref{fig-section4-square} in the center). Since the result is not generally an integer, it is rounded to the nearest integer. Similarly, for lightening the image, the value $a$ of each pixel is replaced by the rounding of $\sqrt{255 a} $. Figure \ref{fig-section4-square}, on the right, shows the obtained result. It will be noted that these two operations (square lightening and square root darkening) are inverse to one another.

% It will be noted that these two operations (square lightening and square root darkening) are inverse to one another.


\myfigure{
\imageTri{section4-negatif}
{section4-square}
{section4-sqrt}
{Negative}
{Square}
{Square root}
}{Changing the contrast.
}{fig-section4-square}


%%%%%%%%%%%%%%%%%%%%%%%%%%%%%%%%%%%%%%%%%%%%%%%%%% %%%%%%%%%%%%
\subsection{Manipulations of Color Contrast}

In order to manipulate the contrast of a color image, it is important to respect the color tones as much as possible. It is therefore simpler to manipulate only the luminance component $a = (r + v + b) / 3$, while maintaining the residue $(r-a, v-a, b-a)$ constant. For example, a change in contrast can be defined by raising the luminance $a$ to the $\ ga> 0$ in order to obtain
\[
	\tilde a = 255 \times \pa{ \frac{a}{255} }^{\ga} = 255 \times \text{exp}\pa{  \ga \times \text{ln}
		\pa{ \frac{a}{255} }  },
\]
(with Convention $\tilde a = 0$ when $a = 0$). It is noted that for $\ga = 1/2$ (respectively $\ga = 2$) the contrast change is found by squaring (respectively square root) introduced in the previous section. And of course, for $\ga = 1$, the luminance is unchanged.

% As $\ga$ is not necessarily an integer, it is important to use the exponential and the logarithm to define this change.

This change in contrast is then reflected on the color image by defining three channels $(\tilde r, \tilde v, \tilde b)$ of a new image by
\[
	\choice{
	\tilde r = \max(0, \min(255, r + \tilde a - a)),\\ 
	\tilde v = \max(0, \min(255, v + \tilde a - a)),\\
	\tilde b = \max(0, \min(255, b + \tilde a - a)).
	}	
\]
It is important to take the maximum with 0 and the minimum with 255 so that the result remains in the range $[0.255]$, and is displayed correctly. Figure \ref{fig-contrast-color} shows the result for different values of $\ga$. For $\ga <1$, the image looks clearer, while for $\ga> 1$, the image is darkened.

\newcommand{\imgsix}[1]{\includegraphics[width=0.16\linewidth]{#1}}

\myfigure{
\begin{tabular}{@{}c@{\hspace{1mm}}c@{\hspace{1mm}}c@{\hspace{1mm}}c@{\hspace{1mm}}c@{\hspace{1mm}}c@{}}
\imgsix{section4-constrast-1}&
\imgsix{section4-constrast-2}&
\imgsix{section4-constrast-3}&
\imgsix{section4-constrast-4}&
\imgsix{section4-constrast-5}&
\imgsix{section4-constrast-6}\\
$\ga=0,5$& 
$\ga=0,75$&
$\ga=1$&
$\ga=1,5$&
$\ga=2$&
$\ga=3$
\end{tabular}
}{Changing the contrast of a color image.
}{fig-contrast-color}





%%%%%%%%%%%%%%%%%%%%%%%%%%%%%%%%%%%%%%%%%%%%%%%%%%%%%%%%%%%%%%
%%%%%%%%%%%%%%%%%%%%%%%%%%%%%%%%%%%%%%%%%%%%%%%%%%%%%%%%%%%%%%
%%%%%%%%%%%%%%%%%%%%%%%%%%%%%%%%%%%%%%%%%%%%%%%%%%%%%%%%%%%%%%
\section{Images and Matrices}

%%%%%%%%%%%%%%%%%%%%%%%%%%%%%%%%%%%%%%%%%%%%%%%%%% %%%%%%%%%%%%
\subsection{Symmetry and Rotation}

An image is an array of numbers, with $n$ lines and $p$
columns. It is therefore easy to perform
some
\lien{https://en.wikipedia.org/wiki/Geometric_transformation}{geometric transformations}
on the image. The values of the pixels that make up this table (denoted $A$) can be
represented as $A = (a_{i,j})_{i,j}$
or index $i$ describes the set of numbers $\{1,\dots, n\}$
(the integers between 1 and n) and the index
$j$ the numbers $\{1,\dots, p\}$.
$a_{i,j}$ is said to be the value of the pixel at position $(i,j)$.

The array of pixels thus indexed is represented as
\[
A = 
\begin{pmatrix}
a_{1,1} &           &           &   & a_{1,p}\\
       &           &  \vdots   &   &  \\
	   &           & a_{i-1,j} &   & \\
\ldots & a_{i,j-1} & a_{i,j}   & a_{i,j+1} & \ldots\\
	   &           & a_{i+1,j} &   & \\
       &           &  \vdots   &   &  \\
a_{n,1} &           &           &   & a_{n,p}\\
\end{pmatrix},
\]
This corresponds to the representation of the image in the form of a matrix. Transposing this matrix corresponds to symmetry with respect to the main diagonal. This transposition is carried out on each of the three color components (see figure \ref{fig-geometric}, on the left).


\myfigure{
\imageTri
{section10-original}
{section10-transpose}
{section10-rotation}
{Matrice $A$}
{Matrix $B$ (transposed)}
{Matrix $C$ (rotation)}
}{Transpose and rotate.
}{fig-geometric}

It is also possible to carry out a rotation by
a quarter turn clockwise on the image. This is obtained by defining a matrix $C = (c_{i,j})_{j, i}$ of
$p$ lines and $n$
columns by $c_{j, i} = a_{n-i+1,j} $.
Figure \ref{fig-geometric}, right, shows the action of this rotation on an image.



%%%%%%%%%%%%%%%%%%%%%%%%%%%%%%%%%%%%%%%%%%%%%%%%%%%%%%%%%%%%%%
\subsection{Interpolation Between Two Images}

It is possible to carry out a
\lien{https://en.wikipedia.org/wiki/Dissolve_(filmmaking)}{transition between two images}
$A$ and $B$. It is therefore assumed that the two images have the same number $n$ of lines
and the same number $p$ of columns. $A = (a_{i,j})_{i,j}$ the pixels of image $A$ and
$B = (b_{i,j})_{i,j}$ the pixels of the image $B$.

For a value $t$ set between $0$ and $1$, the image
$C = (c_{i,j})_{i,j}$ is defined as
\[c_{i,j} = (1-t) a_{i,j} + t b_{i,j} .\]
It is the formula of a
\lien{https://en.wikipedia.org/wiki/Linear_interpolation}{linear interpolation}
between the two images. For a color image, this formula is applied to each of the
channels R, V and B.


It can be seen that for $t = 0$, the image $C$ is equal to the image
$A$. For $t = 1$, the image $C$ is equal to the image
$B$. When the value $t$ increases from 0 to 1, one thus obtains a fading
effect, since the image, which at first is close to image $A$
resembles more and more the image $B$. Figure \ref{fig-interp} shows the result obtained for 6 values of $t$ distributed between 0 and 1.

\myfigure{
\begin{tabular}{@{}c@{\hspace{1mm}}c@{\hspace{1mm}}c@{\hspace{1mm}}c@{\hspace{1mm}}c@{\hspace{1mm}}c@{}}
\imgsix{section11-interp-1}&
\imgsix{section11-interp-2}&
\imgsix{section11-interp-3}&
\imgsix{section11-interp-4}&
\imgsix{section11-interp-5}&
\imgsix{section11-interp-6}\\
Image $A$, t=0 & 
t=0.2 &
t=0.4 &
t=0.6 &
t=0.8 &
Image $B$, t=1
\end{tabular}
}{Linear interpolation.
}{fig-interp}



%%%%%%%%%%%%%%%%%%%%%%%%%%%%%%%%%%%%%%%%%%%%%%%%%% %%%%%%%%%%%%
%%%%%%%%%%%%%%%%%%%%%%%%%%%%%%%%%%%%%%%%%%%%%%%%%% %%%%%%%%%%%%
%%%%%%%%%%%%%%%%%%%%%%%%%%%%%%%%%%%%%%%%%%%%%%%%%% %%%%%%%%%%%%
\section*{Conclusion}

Mathematical processing of images is a very active field, where the theoretical advances are obtained using fast computational algorithms. These algorithms have important applications for the manipulation of digital contents. This article, however, only scratched the surface of the immense list of treatments that can be subjected to an image. We refer to the website \textit{A Numerical Tour of Signal Processing}\FootLink{http://www.numerical-tours.com/} for many more examples of image processing and links to other resources available online.


%%%%%%%%%%%%%%%%%%%%%%%%%%%%%%%%%%%%%%%%%%%%%%%%%%%%%%%%%%%%%%
%%%%%%%%%%%%%%%%%%%%%%%%%%%%%%%%%%%%%%%%%%%%%%%%%%%%%%%%%%%%%%
%%%%%%%%%%%%%%%%%%%%%%%%%%%%%%%%%%%%%%%%%%%%%%%%%%%%%%%%%%%%%%

\section*{Glossary}

\newcommand{\gloss}[1]{\item \textbf{#1}}

\begin{rs}
\gloss{random}: unpredictable value, such as noise disturbing images of bad qualities.
\gloss{Bit}: a basic unit for storing information in the form of 0 and 1 in a computer.
\gloss{Channel}: one of three elementary images that make up a color image.
\gloss{Edges}: the area of an image where the values of the pixels change rapidely, which corresponds to the contours of the objects that make up the image.
\gloss{Noise}: small perturbations that degrade the quality of an image.
\gloss{Square}:  the square $b$ of a value $a$ is $a \times a$. It is noted $a^2$.
\gloss{Contrast}: informal quantity that indicates how much difference there is between light and dark areas of an image.
\gloss{Image compression}: a method to reduce the amount of memory required to store an image on the hard disk.
\gloss{Binary coding}: writing of numeric values using only 0 and 1.
\gloss{Blur}: degradation of an image that makes the contours of objects unclear, and therefore difficult to locate precisely.
\gloss{Fade}: linear interpolation between two images.
\gloss{Color image}: a set of three grayscale images, which can be displayed on a color screen.
\gloss{Digital image}: an array of values that can be displayed on the screen by assigning a gray level to each value.
\gloss{Inverse}: operation that returns an image to its original state.
\gloss{JPEG-2000}: recent image compression method that uses a wavelet transform.
\gloss{Luminance}: average of the different channels in an image, which indicates the light output of the pixel.
\gloss{Matrix}: array of values, represented as $(a_{i,j})_{i,j}$.
\gloss{Median}: central value when sorting a set of values.
\gloss{Average}: the average of a set of values is their sum divided by their number.
\gloss{Grayscale}: grayscale used to display a digital image on the screen.
\gloss{integers}: numbers 0, 1, 2, 3, 4 ...
\gloss{Byte}: set of eight consecutive bits.
\gloss{Wavelets}: image transformation that is used by image compression JPEG-2000 method.
\gloss{Ascending order}: classifying a set of values from the smallest to the largest.
\gloss{Pixel}: a single element in the array of values that corresponds to a digital image.
\gloss{Quantization}: a method to reduce the set of possible values of a digital image.
\gloss{square root}: the square root $b$ of a positive value $a$ is the positive value $b$ such that $a = b \times b$. It is denoted $\sqrt{a}$.
\gloss{Resolution}: the size of an image (number of pixels).
\gloss{Exposed}: photograph of a scene too dark for which the photographic lens did not stay open long enough.
\gloss{Additive synthesis}: rule to construct any color from the three colors red, green and blue. This is the rule that governs the mixing of the colors of light beams when  illuminating a white wall.
\gloss{Subtractive synthesis}: rule to construct any color from the three cyan, magenta and yellow colors. This is the rule that governs the mixing of colors in paint.
\end{rs}
 


% !TEX root = ../IntroImaging.tex


\chapter{Sparsity, Inverse Problems and Compressed Sensing}
\label{chap-sparsity}

Current standards for compressing music, image or video (MP3, JPG, or MPEG) all use methods derived from non-linear approximation. These methods compute an approximation of the initial data using a linear combination of a small number of elementary functions (such as sinusoids or wavelets).
%
These methods, initially used for approximation, denoising or compression, have been applied more recently to more difficult problems, such as increasing the resolution or inversion of operators in medical imaging. These extensions require the resolution of large-scale optimization problems, and are the subject of intense research activity.
%
One of the most recent advances in this field, compressed sampling, uses the theory of random matrices in order to obtain theoretical guarantees for the performance of these techniques. Compressed sampling allows Claude Shannon's theory of sampling and compression to be considered from a new angle. The compressibility of the data allows for simultaneous sampling and compression.

This chapter presents the key mathematical concepts that have allowed evolution from classical Shannon sampling to compressed sampling. The notion of sparse decomposition, which makes it possible to formalize the idea of compressibility of information, is the main thread.


%%%%%%%%%%%%%%%%%%%%%%%%%%%%%%%%%%%%
\section{Traditional Sampling}
\label{sec-sampling}

In the digital world, most data (sound, image, video, etc.) are discretized in order to store, transmit and modify them.
%
From an analog signal, which is represented by a continuous function $s \mapsto \tilde f(s)$, the measurement device calculates a set of discretized values $f = (f_q)_{q=1}^Q \in \RR^Q$.
%
Thus, $Q$ is the number of time samples for an audio piece or the number of pixels for an image.
%
Figure~\ref{fig-samples} shows examples of discretized data.
%
In the case of an image, $\tilde f(s)$ represents the amount of light arriving at a point $s \in \RR^2$ of the camera's focal plane, and $f_q = \int_{c_q} \tilde f(s) \text{d} s$ is the total amount of light illuminating the $c_q$ surface of a CCD sensor indexed by $q$.
%
For simplicity, here we assume scalar data (eg mono sound, grayscale image, or video), but the techniques described here may extend to vector data (stereo sound, color image ).


\begin{figure} \centering
\begin{tabular}{@{}c@{\hspace{5mm}}c@{}}
\includegraphics[width=.45\linewidth]{discrete/signal} &
\includegraphics[width=.45\linewidth]{discrete/image}
\end{tabular}
\caption{\label{fig-samples} Examples of a sound signal (1D data) and an image (2D data) discretized.}
\end{figure}

It is the theory developed by Claude Shannon~\cite{Shannon1948} that laid the foundation for sampling (the use of a discrete vector $f$ to faithfully represent a continuous function $\tilde f$) but also those of lossless compression.
%
We will see how current research has made it possible to build on these foundations lossy compression methods (i.e with a slight degradation of the quality), as well as to revisit the conventional sampling to give rise to the idea of compressed sampling.


%%%%%%%%%%%%%%%%%%%%%%%%%%%%%%%%%%%%
\section{Nonlinear Approximation and Compression}


%%%%%
\subsection{Nonlinear Approximation}

The size $Q$ of these data is generally very large (of the order of million for an image, of the billion for a video) and it is necessary to calculate a more economical representation in order to be able to store $f$ or to transmit it on a network.
%
All modern lossy compression methods (MP3, JPEG, MPEG, etc.) use sparse decompositions (that is composed of few non-zero coefficients) in a dictionary $\Psi = (\psi_n)_{n=1}^N$ composed of elemental atoms $\psi_n \in \RR^Q$.
%
It is thus sought to approach $f$ with the aid of a linear combination
\eq{
	f \approx  \Psi x \eqdef \sum_{n=1}^N x_n \psi_n \in \RR^Q
}
where the $x = (x_n)_{n=1}^N \in \RR^N$ are the coefficients that will be stored or transmitted. In order for this representation to be economical, and for storage to take up little space, it is necessary that a maximum of coefficients $x_n$ be zero, so that only the non-zero coefficients have to be stored. Given a budget $M>0$ of non-zero coefficients, the best possible combination is sought in order to approximate $\ell^2$ the initial data. The aim is to solve the optimization problem
\eql{\label{eq-pbm-approx}
	x^\star \in \uargmin{x \in \RR^N} \enscond{  \norm{f - \Psi x}_2  }{ \norm{x}_0 \leq M }
	\qwhereq
	\norm{f}_2^2 \eqdef \sum_{q=1}^Q |f_q|^2.
}
Here we have noted $\norm{x}_0 \eqdef \sharp\enscond{n}{x_n \neq 0}$ the number of non-zero coefficients of $x$, which is a counting measure often referred to by language abuse as the \guill{pseudo-norm} $\ell^0$ (which is not a standard!). This abuse of language will be explained in section~\ref{sec-pb-inv}, see in particular figure~\ref{fig-boules}.

The problem~\eqref{eq-pbm-approx} is in general impossible to solve: it is a combinatorial problem, which, without further hypothesis on $\Psi$, requires the exploration of all combinations of $M$ coefficients non-zero. It has been proved that this problem is indeed NP-difficult~\cite{Natarajan95}.


%%%%%
\subsection{Approximation in an orthonormal basis}

There is however a simple case, which is very useful for compression: this is the case where $\Psi$ is an orthonormal basis of $\RR^Q$, ie $Q = N$ and
\eq{
	\dotp{\psi_n}{\psi_{n'}} = \choice{
		1 \qsiq n = n', \\
		0 \quad\text{sinon.}
	}
	\qwhereq
	\dotp{f}{g} \eqdef \sum_{q=1}^Q f_q g_q.
}
This case is the one most often encountered for data compression, using for example discrete Fourier orthogonal bases, local cosines (used for MP3, JPG and MPG) and wavelets (used for JPEG2000), see the book~\cite{mallat2009a-wav}.
%
In this case, we have the identity of Parseval which corresponds to the decomposition of $f$ in an orthonormal basis
\eql{\label{eq-expansion-bon}
	f = \sum_{n=1}^N \dotp{f}{\psi_n} \psi_n 
	\qetq
	\norm{f - \Psi x}_2^2 = \sum_{n=1}^N | \dotp{f}{\psi_n} - x_n |^2.
}
These formulas show that the solution of~\eqref{eq-pbm-approx} is very simple to calculate.
%
Indeed, to minimize $\norm{f - \Psi x}_2$, for each non-zero $x_n$, one should choose $x_n = \dotp{f}{\psi_n}$.
%
And since we set a maximum budget of $M$ non-zero coefficients, we must choose the $M$ largest coefficients $|\dotp{f}{\psi_n}|$ in the formula~\eqref{eq-expansion-bon}. Mathematically, if we note $|\dotp{f}{\psi_{n_1}}| \geq |\dotp{f}{\psi_{n_2}}| \geq \ldots$ a sorting of the coefficients in descending order, then a $x^\star$ solution of~\eqref{eq-pbm-approx} is given by
\eql{\label{eq-formule-thresh}
x^\star_n = \choice{
\dotp{f}{\psi_{n}} \qsiq \in \{n_1,\ldots, n_M\}, \\
0 \quad\text{otherwise}
}
}

\newcommand{\myPic}[1]{\includegraphics[trim=50 50 30 30,clip,width=.24\linewidth]{approx/#1}}
\begin{figure}\centering
\begin{tabular}{@{}c@{\hspace{1mm}}c@{\hspace{1mm}}c@{\hspace{1mm}}c@{}}
\myPic{cameraman} &
\myPic{cameraman-4} &
\myPic{cameraman-8} &
\myPic{cameraman-16} \\
$f$ & 
$\Psi x^\star, M=N/4$ & 
$\Psi x^\star, M=N/8$ &  
$\Psi x^\star, M=N/16$ 
\end{tabular}
\caption{\label{fig-approx} Approximate Examples $f \approx \Psi x^\star$ with $M = \norm{x^\star}_0$ which varies, for a $f \in \RR^N$ image of $N = 256^2$ pixels.}
\end{figure}


The figure~\ref{fig-approx} shows approximations $f \approx \Psi x^\star$, with a variable number $M = \norm{x^\star}_0$  of coefficients.
%
These approximations are performed using an orthogonal base of wavelets $\Psi$, called the Daubechies 4 base, which are similar to the functions used in the JPEG2000 image compression standard, and are popular because there is a fast algorithm for calculate the scalar products $( \dotp{f}{\psi_{n}} )_n$ with a computation time proportional to $Q$ (see the book~\cite[Chapter 7]{mallat2009a-wav} for a complete description of the theory of wavelets).
%
It can be seen that the quality of the reconstructed image $\Psi x^\star$ degrades when $M$ decreases, but one can still considerably reduce the amount of information to be stored (the $M/Q$ compression ratio is small), while maintaining an acceptable visual quality.
%
This fundamental observation corresponds to the fact (observed in practice) that natural images are very well approximated by a \guill{sparse} linear combination of the form $\Psi x^\star$ with $\norm{x^\star}_0 \leq M$.
%
It is important to note that, although the calculation of $\Psi x^\star$ from $x^\star$ is a linear formula, the calculation of $x^\star$ from $f$ is \textit{non-linear}, as can be seen in the formula~\eqref{eq-formule-thresh}. The transition from $f$ to its approximation $\Psi x^\star$ is called a non-linear approximation.
%
The theoretical justification of this observation is the object of the study of nonlinear approximation theory, which seeks to prove that $\norm{f-\Psi x^\star}$ decreases rapidly when $M$ increases under certain assumptions of regularity over $f$, for instance assuming that the image is piecewise smooth, see~\cite[Chap. 9]{mallat2009a-wav}.

In order to obtain an complete compression algorithm, it is then necessary to use a technique making it possible to convert the $M$ coefficients $(x_{n_1}, \dots, x_{n_2})$ into binary writing and also to store the non-zero indices $(n_1,\ldots, n_M)$. This is done simply using techniques derived from information theory, in particular entropy coding methods, see~\cite[Chap. 10]{mallat2009a-wav}.

%%%%%%%%%%%%%%%%%%%%%%%%%%%%%%%%%%%%%%%%%%%%%
\section{Inverse Problems and Sparsity}
\label{sec-pb-inv}

%%%%%
\subsection{Inverse Problems}

Before $f$ data can be stored, it is most of the time necessary to carry out a preliminary restoration step, which consists in improving the quality of the data from observations of low quality, that is to say --9 of low resolution, possibly blurred , entangled with errors and noisy. In order to take into account the whole chain of data formation, we model mathematically the acquisition process in the form
\eql{\label{eq-fwd-model}
	y = \Phi f + w \in \RR^P
}
where $y \in \RR^P$ are the $P$ observations measured by the device, $w \in \RR^P$ is a measurement noise (unknown), $f \in \RR^Q$ is the image (unknown) that one wishes to recover, and $\Phi : \RR^Q \rightarrow \RR^P$ is an operator modeling the acquisition apparatus, and which is assumed to be ``linear". This means that $\Phi$ may be considered as a (gigantic) matrix $\Phi \in \RR^ (P \times Q) $. It is important to note that most of the time this matrix $\Phi$ is never explicitly stored, it is manipulated implicitly by means of fast operations (convolution, masking, etc.).


\begin{figure} \centering
\begin{tabular}{@{}c@{\hspace{4mm}}c@{\hspace{4mm}}c@{}}
\includegraphics[width=.25\linewidth]{operators/lena-original} &
\includegraphics[width=.25\linewidth]{operators/lena-blurring} &
\includegraphics[width=.25\linewidth]{operators/lena-inpainting} \\
Image originale $f$ & $\Phi f$ (flou) & $\Phi f$ (masquage)  
\end{tabular}
\caption{Observation (noiseless, $w=0$) $y=\Phi f$ in the case of a convolution ($\Phi f = \phi \star f$ is a convolution against a low pass filter $\phi$) and missing data  ($\Phi=\diag(\mu_q)_{q=1}^Q$ is a masking operator).  \label{fig-exemple-ip} }
\end{figure}


This model, which may seem rather restrictive (in particular the hypothesis of linearity) makes it possible to model a surprising quantity of situations that one meets in practice. For example,
\begin{itemize}
	\item Denoising: $\Phi = \Id_{\RR^Q} $, $P = Q$ and one is in the (simplest) situation in which one only seeks to remove the noise $w$ ;
	 \item Deconvolution: (see Figure~\eqref{fig-exemple-ip}, center) $\Phi f = \phi \star f$ is a convolution by a filter $\phi$ modeling for example the blur of a camera (either a blur of shake or a blur due to development) ;
	\item Missing data: (see Figure~\eqref{fig-exemple-ip}, right)  $\Phi=\diag(\mu_q)_{q=1}^Q$  is a diagonal masking operator, such as $\mu_q = 1$ if the data indexed by $q$ (for example, one pixel) is observed, and $\mu_q = 0$ if the data is missing;
	 \item tomographic imagery: $\Phi$ is a more complex linear operator, calculating integrals along straight lines (the Radon transform), see \cite[Sect. 2.4]{mallat2009a-wav}.
\end{itemize}
There are many other examples (in medical imaging, seismic, astrophysics, etc.), and in each case, calculating a good approximation of $f$ from $y$ is very difficult. Indeed, with the exception of denoising (ie $\Phi = \Id_{\RR^Q} $), the formula $\Phi^{-1} y = f + \Phi^{-1} w$ can not be used either because $\Phi$ is not invertible (for example for missing data), or because $\Phi$ has very small eigenvalues (for deconvolution or tomography), so that $\Phi^{-1} w$ is going to be very large, and thus $\Phi^{-1} y$ is a very bad approximation of $f$.


%%%%%
\subsection{Sparse Regularization}

To remedy this problem, we need to replace $\Phi^{-1}$ with an approximate \guill{inverse} which takes into account additional assumptions about the $f$ signal we are looking for. Recent methods, which give the best results on complex data, use an approximate inverse which is nonlinear. This may seem contradictory because $\Phi$ is linear, but the use of nonlinear methods is crucial to take advantage of realistic assumptions about complex data such as images.
%
Based on the approximation and compression techniques discussed in the previous section, current methods seek to exploit the fact that one can approach $f$ with a sparse approximation $\Psi x$ with $\norm{x}_0 \leq M$. Given a parameter $M> 0$, we will look to approximate $f$ by $f^\star = \Psi x^\star$ where $x^\star$ is a solution of
\eql{\label{eq-pbm-l0}
	x^\star \in \uargmin{x \in \RR^N} \enscond{\norm{y - \Phi \Psi x}_2}{\norm{x}_0 \leq M}
}
We see that~\eqref{eq-pbm-l0} is quasi-identical to~\eqref{eq-pbm-approx}, except that $f \in \RR^Q$ (unknown) has been replaced by $y\in \RR^P$, and that matrix $\Psi \in \RR^{Q \times N}$ by matrix product $\Phi \Psi \in \RR^{P \times N} $. In the particular case of denoising, $\Phi = \Id_{\RR^Q} $, the problems~\eqref{eq-expansion-bon} and~\eqref{eq-pbm-l0} are equivalent and have the same solution, so that the nonlinear approximation solves the denoising problem .

In the case of any operator $\Phi$, the problem~\eqref{eq-pbm-l0} is however an optimization problem extremely difficult to solve. Indeed, even if $\Psi$ is an orthonormal basis, in general (except in the case of denoising $\Phi = \Id_{\RR^Q}$), the matrix $\Phi \Psi$ is not orthogonal, so that the formula~\eqref{eq-formule-thresh} is not applicable, and~\eqref{eq-pbm-l0} is a NP-difficult combinatorial search problem.


%%%%%
\subsection{$\ell^1$ Regularization}

The approximation of the solutions of the problem~\eqref{eq-pbm-l0} using efficient methods is one of the most active subjects of research in data processing (and more generally in applied mathematics, imagery, statistics and machine learning). There are many methods, including greedy algorithms (see, for example,~\cite{MallatMP}) and convex relaxation methods. We will focus on this second class of methods.
%
One way (heuristic) to introduce these techniques is to replace $\norm{\cdot}_0$ in the~\eqref{eq-pbm-l0} problem with the function $\norm{\cdot}_\al^\al$, which is set to $\al>0$ by
\eq{
	\norm{x}_{\al}^\al \eqdef \sum_{n=1}^N |x_n|^\al.
}
Figure \ref{fig-boules} shows in the (unrealistic but convenient to draw) case of $N = 2$ coefficients, the balls of the $B_\al \eqdef \enscond{x}{\norm{x}_\al \leq 1}$ units associated with these functional $\norm{\cdot}_\al$. It can thus be seen that $B_\al$ \guill{tend} to the \guill{unit ball} associated with the counting measure $\norm{\cdot}_0$ as $\al$ tends to $0$,
\eq{
	B_\al \overset{\al \rightarrow 0}{\longrightarrow} B_0 \eqdef \enscond{ x \in [-1,1]^N }{ \norm{x}_0 \leq 1 },
}
the convergence of these sets (which is well visualized in the figure) being in the sense for example of the Hausdorff distance.
%
The limiting ball $B_0$ is composed of extremely sparse vectors, since they are composed of a single non-zero component.

\begin{figure} \centering
\begin{tabular}{@{}c@{\hspace{1mm}}c@{\hspace{1mm}}c@{\hspace{1mm}}c@{\hspace{1mm}}c@{}}
\includegraphics[width=.19\linewidth]{balls/l0} &
\includegraphics[width=.19\linewidth]{balls/l12} &
\includegraphics[width=.19\linewidth]{balls/l1} &
\includegraphics[width=.19\linewidth]{balls/l32} &
\includegraphics[width=.19\linewidth]{balls/l2} \\
$\al=0$ & $\al=1/2$ & $\al=1$ & $\al=3/2$ & $\al=2$
\end{tabular}
\caption{\label{fig-boules} Balls $B_\al$ for different values of $\al$.}
\end{figure}


One then has to take into account two conflicting points to choose a value of $\al$:
\begin{itemize}
	\item In order to have a functional enforcing the sparsity of vectors, we want to use a value of $\al$ as low as possible to replace $\norm{\cdot}_0$ by $\norm{\cdot}_\al$.
	\item In order to calculate the solution of~\eqref{eq-pbm-l0} with $\norm{\cdot}_\al$ instead of $\norm{\cdot}_0$, it is important that the $\norm{\cdot}_\al$ be \textit{convex}. The convexity is indeed essential in order to obtain a problem that is not NP-difficile and to be able to benefit from fast calculation algorithms. These algorithms find an exact solution $x^\star$ in polynomial time or quickly converge to this solution.
\end{itemize}
The convexity constraint of $\norm{\cdot}_\al$ requires that the set $B_\al$ be convex, which equivalently means that $\norm{\cdot}_\al$ must be a \textit{norm}. This imposes that $\al \geq 1$. Taking these two constraints into account leads naturally to the choice  $\al = 1$, so that we consider the convex optimization problem (that is, seeks to minimize a convex function on a convex set)
\eql{\label{eq-pbm-l1}
	x^\star \in \uargmin{x \in \RR^N} \enscond{  \norm{y - \Phi \Psi x}_2  }{ \norm{x}_1 = \sum_{n=1}^N |x_n| \leq \tau }, 
}
so that the retrieved image is defined as $f^\star = \Psi x^\star$.
%
It may be noted that a parameter $\tau>0$ was used here, which plays a role similar to parameter $M$ which appears in~\eqref{eq-pbm-l0}.
%
The question of choosing this parameter $\tau$ is crucial. If the noise $w$ is small, then we want that $\Phi f^\star = \Phi\Psi x^\star$ be close to $y$, and so we will choose $\tau$ big. On the contrary, if the noise $w$ is important, in order to obtain a greater denoising effect, the value of $\tau$ is reduced. The choice of a $\tau$ \guill{optimal} is a difficult search problem, and there is no universal response, the existing strategies strongly depend on the $\Phi$ operator as well as the atoms family~$\Psi$ .

The problem~\eqref{eq-pbm-l1} was initially proposed by engineers in the fields of seismic imaging (see for example~\cite{santosa1986linear}), and it was introduced jointly in signal processing under the name \guill{basis pursuit}~\cite{chen1999atomi} and in statistics under the name \guill{Lasso}~\cite{tibshirani1996regre}.


\begin{figure} \centering
\begin{tabular}{@{}c@{\hspace{4mm}}c@{\hspace{4mm}}c@{}}
\includegraphics[width=.25\linewidth]{operators/lena-original} &
\includegraphics[width=.25\linewidth]{operators/lena-blurring} &
\includegraphics[width=.25\linewidth]{operators/lena-inpainting} \\
$f$ Original & Observations $y$ & Reconstruction $f^\star$
\end{tabular}
\caption{Examples of reconstruction with missing data, $\Phi = \diag(\mu_q)_{q = 1}^Q$ with $\mu_q \in \{0,1\}$ and a number of observed data $\sharp\enscond{q}{\mu_q = 1}=10\%$. \label{fig-inpainting}}
\end{figure}

The problem~\eqref{eq-pbm-l1}, although convex, remains a difficult problem to solve because of the nondifferentiability of $\norm{\cdot}_1$ and large data size ($N$ is very large). This is the price to pay for getting good quality results. As will be explained in the following paragraph, it is in fact the nondifferentiability of $\norm{\cdot}_1$ which makes it possible to obtain sparsity. The development of efficient algorithms to solve~\eqref{eq-pbm-l1} is a very active field of research, and we refer to~\cite[section 6]{2014-vaiter-ps-review} for a review of these methods. Figure~\ref{fig-inpainting} shows an example of missing data interpolation performed by solving~\eqref{eq-pbm-l1} in a $\Psi$ family of translationally invariant wavelets.


%%%%
\subsection{From Intuition to Theory}


The figure~\ref{fig-l1-vs-l2} shows intuitively why the $x^\star$ solution calculated by replacing $\norm{\cdot}_0$ by $\norm{\cdot}_\al$ in~\eqref{eq-pbm-l0} is better (in the sense that it is more sparse) if we choose $\al=1$ (that is, if we solve~\eqref{eq-pbm-l1}) than if we choose $\al = 2$ (a similar conclusion is obtained for other values of $\al> 1$).
%
The figure is made in the (very simple) case of $N = 2$ coefficients and $P = 1$ observations. The crucial point, which makes the solution of~\eqref{eq-pbm-l1} sparse, is that the ball $B_1$ associated with the $\ell^1$ standard is \guill{pointed} so that the $x^\star$ solution is located along the axes. This is not the case for ball $B_2$ associated with standard $\ell^2$, which gives a $x^\star$ solution that is not along the axes, and therefore is not sparing.
%
This phenomenon, already visible in dimension 2, is actually accentuated when the dimension increases, so that the approximation obtained by replacing $\norm{\cdot}_0$ by $\norm{\cdot}_1$ becomes better in large dimension.
%
This phenomenon is called the \guill{blessing of dimensionality} by David Donoho~\cite{DonohoCurse}: although the data become very expensive and complex to treat, there are effective analysis and processing techniques when they are sufficiently sparse.
%
To make this intuition rigorous, however, is difficult, and this is the object of research still in progress for operators $\Phi$ such as convolutions~\cite{candes-towards2013,2015-duval-focm}. The analysis in the case of the operators that one meets for example in medical imaging is an open mathematical problem.


\begin{figure} \centering
\begin{tabular}{@{}c@{\hspace{4mm}}c@{}}
\includegraphics[width=.35\linewidth]{l1-vs-l2/l1} &
\includegraphics[width=.35\linewidth]{l1-vs-l2/l2} \\
$\ell^1$ minimisation &  $\ell^2$ minimisation
\end{tabular}
\caption{Comparison of the minimization with constraints of type $\norm{x}_\al \leq \tau$ for $\al \in \{1,2\}$.
%
A $x^\star$ solution is obtained when a tube $\enscond{x}{\norm{\Phi x-y} \leq \epsilon}$ is sufficiently large (ie gradually growing $\epsilon$) as it is tangent in $x^\star$ to the ball $\enscond{x}{\norm{x}_\al \leq \tau}$. \label{fig-l1-vs-l2}}
\end{figure}



%%%%%%%%%%%%%%%%%%%%%%%%%%%%%%%%%%%%%%%%%%%%%
\section{Compressed sampling}

There exists a particular class of operators $\Phi$ for which it is possible to analyze very precisely the performances obtained when we solve~\eqref{eq-pbm-l1}. This is the case where $\Phi$ is drawn randomly according to some distributions of random matrices. Using random matrices may seem strange, because the operators mentioned above (convolution, tomography, etc.) are not at all.
%
In fact, this choice is motivated by a concrete application proposed jointly by Cand�s, Tao and Romberg~\cite{candes2006stable} as well as Donoho~\cite{donoho2006compressed}, and which is commonly called \guill{compressed sensing}.


%%%%
\subsection{Single Pixel Camera}

In order to illustrate the exposition, we will discuss the \guill{single pixel camera} prototype developed at Rice University~\cite{DuarteSinglePixel}, and which is illustrated by the figure~\ref{fig-single-pixel} (left).
%
It is an important research problem of developing a new class of cameras allowing to obtain both the sampling and the compression of the image. Instead of first sampling very finely (ie with very large $Q$) the analog signal $\tilde f$ to obtain a $f \in \RR^Q$ image then compressing enormously (ie with $M$ small) using~\eqref{eq-formule-thresh}, we would like to dispose directly of an economic representation $y \in \RR^P$ of the image, with a budget $P$ as close to $M$ and such that one is able to \guill{decompress} $y$ to obtain a good approximation of the image $f$.

The \guill{single-pixel} hardware performs the compressed sampling of an observed scene $\tilde f$ (the letter \guill{R} in Figure~\ref{fig-single-pixel}), which is a continuous function indicating the amount of light $\tilde f(s)$ reaching each point $s \in \RR^2$ of the focal plane of the camera.
%
To do this, the light is focused against a set of $Q$ micro-mirrors aligned on the focal plane. These micro-mirrors are not sensors. Unlike conventional sampling (described in Section~\ref{sec-sampling}), they do not record any information, but they can each be positioned to reflect or absorb light.
%
To obtain the complete sampling/compression process, one very quickly changes $P$ times the configurations of the micro-mirrors. For $p = 1,\dots, P$, one sets $\Phi_{p, q} \in \{0,1\}$, depending on whether the micromirror at position $q$ has been placed in the absorbing (0) or reflective (value 1) position at step $p$ of the acquisition.
%
The total light reflected at step $p$ is then accumulated into a single sensor (hence the name \guill{single pixel}, in fact it is rather a \guill{single sensor}), which achieves a linear sum of the reflected intensities to obtain the recorded $y_p \in \RR$ value.
%
In the end, if the light intensity arriving on the surface $c_q$ of the mirror indexed by $f_q = \int_{c_q} \tilde f(s) \text{d} s$ is denoted (as in the~\ref{sec-sampling} section) as $q$, the equation that links the discrete image $f \in \RR^Q$ \guill{seen through the mirrors} to the $P$ measures $y \in \RR^P$ is
\eq{
	\foralls p = 1,\ldots,P, \quad
	y_p = \sum_q \Phi_{p,n} \int_{c_n} \tilde f(s) \text{d} s = (\Phi f)_p, 
}
which corresponds exactly to~\eqref{eq-fwd-model}.
%
It is important to note that the mirrors do not record anything, so in particular the $f$ discrete image is never calculated or recorded, since the device directly calculates the compressed representation $y$ from the analog signal $\tilde f$.
%
The term $w$ models here the acquisition imperfections (measurement noise). The compressed sampling therefore corresponds to the transition from the observed scene $\tilde f$ to the compressed vector $y$. The \guill{decompression} corresponds to the resolution of an inverse problem, whose goal is to find a good approximation of $f$ (the discrete image \guill{ideal} as seen by the micro-mirrors) from $y$.

\begin{figure} \centering
\begin{tabular}{@{}c@{\hspace{1mm}}c@{\hspace{1mm}}c@{}}
\includegraphics[width=.45\linewidth]{single-pixel/single-pixel-schema}&
\includegraphics[width=.25\linewidth]{single-pixel/reconstruction-1}&
\includegraphics[width=.25\linewidth]{single-pixel/reconstruction-6}\\
Diagram of the device & $f$ & $f^\star$, $P/Q = 6$
\end{tabular}
\caption{Left: diagram of the single-pixel acquisition method.
%
Center: image $f \in \RR^Q$ \guill{ideal} observed in the focal plane of the micro-mirrors.
%
Right: image $f^\star = \Psi x^\star$ reconstructed from observation $y \in \RR^P$ with a compression factor $P / Q = 6$.
\label{fig-single-pixel}}
\end{figure}




%%%%
\subsection{Theoretical Guarantees}

An important feature of this inverse problem is that one can choose, as desired, the configurations of the micro-mirrors, which amounts to saying that one can choose freely the matrix $\Phi \in \{0,1\}^{P \times Q} $. The question is therefore to make the best choice, so that the inverse problem can be solved effectively. If we make the hypothesis that the signal $f$ to be reconstructed is compressible in an orthonormal basis $\Psi$ (that is to say that $f \approx \Psi x_0$ with $M \eqdef \norm{x_0}_0$ small) then several recent works, starting with~\cite{candes2006stable,donoho2006compressed}, showed that method~\eqref{eq-pbm-l1} is effective if $\Phi$ is chosen as a realization of some random matrices. For the single-pixel camera, each $\Phi_{p, n}$ can then be randomly drawn with a probability of $1/2$ for the values $0$ and $1$.
%
In practice, a pseudo-random generator is used, so that both the person who compresses the data and the person who is going to decompress them knows the matrix $\Phi$ perfectly (because they can communicate the seed of the generator).
%
The figure~\ref{fig-single-pixel} (right) shows an example of reconstruction obtained for the case of the single-pixel apparatus with such a random choice of matrix $\Phi$, with $\Psi$ a translation-invariant family of wavelets (see~\cite[Section 5.2]{mallat2009a-wav} for a description of this family).

It has been shown by~\cite{candes2006stable,donoho2006compressed} that there exists a constant $C$ such that if $f = \Psi x_0$ where $x_0$ are the coefficients of the image to be retrieved, where $\Psi$ is an orthogonal basis therefore in particular $Q = N$), and if the number $P$ of measurements satisfies
\eql{\label{eq-cs-contrainte}
\frac{P}{M} \geq C \log\pa{\frac{N}{M}} \qwhereq M \eqdef \norm{x_0}_0
}
then a solution $f^\star = \Psi x^\star$ computed by~\eqref{eq-pbm-l1} tends to $f$ when the $w$ noise tends to $0$ and $\tau$ tends to$ +\infty$. This result is true \guill{with high probability} on the random drawing of the matrix $\Phi$, that is to say a probability tending rapidly towards 1 when $N$ increases. In particular, if there is no noise, $w = 0$, taking $\tau \rightarrow + \infty$, the method makes it possible to find exactly $f$ if $P$ satisfies~\eqref{eq-cs-contrainte}.
%
This theory also allows to take account of \guill{compressible} data, ie if we only assume that $f$ is close to (but not necessarily equal to) $\Psi x_0$ with $M \eqdef \norm{x_0}_0$ small.

Intuitively, this theoretical result means that compressed sampling can do almost \guill{as well} by calculating $\Psi x^\star$ from $y$ (solving~\eqref{eq-pbm-l1}) than a usual compression method (MP3, JPEG , JPEG2000, MPEG, etc.) that would know exactly the $f$ signal and calculate the best approximation $\Psi x_0$ with $M \eqdef \norm{x_0}_0$ coefficients (solving~\eqref{eq-pbm-approx} via the formula~\eqref{eq-expansion-bon}).
%
The precise meaning of the qualifier \guill{equally} corresponds to the $C \log(N/M)$ multiplying factor, which bounds $P/M$. This factor corresponds to the \guill{extra cost} of the compressed sampling method (which calculates $P$ measurements) compared to a usual compression method (which calculates $M$ coefficients).
%
Despite this additional cost, the compressed sampling method has many advantages: saving time and energy (at the same time sampling and compression), \guill{democratic} coding (all $y_n$ coefficients play the same role, and therefore none has a dominant role, unlike the coding of the coefficients of $x_0$ which have an importance proportional to their amplitude), coding automatically encrypted (if $\Phi$ is not known, $f$ can not be found from $y$ ). The value of the $C$ constant depends on the meaning given to \guill{with high probability}. If this probability bears only $\Phi$, but must be true for all $x_0$ (worst case analysis), then it is very large (see~\cite{dossal-laa-09}). If, on the other hand, if the high probability is both on $\Phi$ and $x_0$ (so that the theoretical result is true for almost all the signals) then it can be shown that for example, for $N/P = 4$, we have $C \log (N / M) \sim 4$ (see~\cite{chandrasekaran2012convex}), which remains a significant overhead but is acceptable for some applications.

The \guill{single pixel} camera is a particular application of the compressed sampling technique. Applications to photography are limited because the CCD sensors of cameras are powerful and inexpensive. Compressed sampling is likely to have an impact on applications where measurements are difficult to acquire or costly. Another source of potential applications is medical imaging, for example by magnetic resonance imaging (MRI). In these fields, however, it is impossible to obtain totally random matrices, so that the theory of compressed sampling can not be applied directly. Encouraging results on these applications have however been obtained, see for example~\cite{AdcockBreaking,Chauffert14}.


%%%%%%%%%%%%%%%%%%%%%%%%%%%%%%%%%%%%%%%%%%%%%
\section*{Conclusion}

Recent advances in data analysis have made it possible to extend the scope of compression in order to deal with difficult inverse problems in imaging, but also in other fields (recommendation system, network analysis, etc.). These advances have been made possible by the use of a very broad spectrum of techniques in applied mathematics, covering both harmonic analysis, nonlinear approximation, non-smooth optimization and probability, but also analysis and PDEs (which were not mentioned in this article). Sparse methods associated with $\ell^1$ regularization are only the tip of the iceberg, and more advanced regularizations make it possible to obtain better results by taking into account the complex geometric structures of the data. For more details on these latest advances, we recommend reading the article~\cite{2014-vaiter-ps-review}, as well as visiting the web site \guill{Numerical Tours of Signal Processing}~\cite{2011-peyre-cise}, which features many computer codes to carry out the numerical experiments presented here, as well as many others.

%%%%%%%%%%%%%%%%%%%%%%%%%%%%%%%%%%%%%%%%%%%%%
\section*{Acknowledgments}

I would like to thank Charles Dossal, Jalal Fadili, Samuel Vaiter, St�phane Seuret and the anonymous reviewer for their invaluable help.



% !TEX root = ../IntroImaging-FR.tex

\newcommand{\iC}{{\color{red}i}}
\newcommand{\jC}{{\color{blue}j}}

\chapter{Le Transport Optimal et ses Applications}
\label{chap-ot}

Livres sur le transport optimal~\cite{Villani03}

%%%%%%%%%%%%%%%%%%%%%%%%%%%%%%%%%%%%%%%%%%%%%%
\section{Le Transport Optimal de Monge}

Gaspard Monge, en plus d'�tre un grand math�maticien, a particip� activement � la r�volution Fran�aise, et a cr�er l'\'Ecole Polytechnique ainsi que l'\'Ecole Normale Sup�rieure. Motiv� par des applications militaire, il a formul� en 1781 le probl�me du transport optimal~\cite{Monge1781}. Il s'est pos� la question du calcul de la fa�on la plus �conomique de transporter de la terre entre deux endroits pour faire des remblais. Dans son texte original, il a fait l'hypoth�se que le co�t du d�placement d'une unit� de masse est �gal � la distance parcourue, mais l'on peut utiliser n'importe quel co�t adapt� au probl�me � r�soudre. 

Pour illustrer le probl�me et sa formulation math�matique, int�ressons-nous � la fa�on optimale de distribuer les croissantes depuis les boulangeries vers le caf�, le matin dans Paris. Pour simplifier, nous allons supposer qu'il y a uniquement 6 boulangeries et caf�s, que l'on peut voir � la figure~\ref{fig:image-cafe} (les boulangeries sont en rouge et les caf�s en bleu). Le co�t � minimis� est le temps de trajets, et l'on note $C_{\iC,\jC}$ le temps entre la boulangerie $\iC \in \{1,\ldots,6\}$  et le caf� $\jC \in \{1,\ldots,6\}$. Par exemple, on a $C_{{\color{red}3},{\color{blue}4}}=10$, ce qui signifie qu'il y a 10 minutes de trajet entre la boulangerie num�ro ${\color{red}3}$ et le caf� num�ro ${\color{blue}4}$. 

\begin{figure}\centering
    \begin{tabular}{@{}c@{\hspace{1mm}}c@{\hspace{4mm}}c@{\hspace{1mm}}c@{}}
        \includegraphics[width=.2\linewidth]{transport/cafe-paris/map-paris-0-couts} 
        \includegraphics[width=.24\linewidth]{transport/cafe-paris/map-paris-0} 
        \includegraphics[width=.2\linewidth]{transport/cafe-paris/map-paris-1-couts} 
        \includegraphics[width=.24\linewidth]{transport/cafe-paris/map-paris-1} 
    \end{tabular}
    \caption{\label{fig:image-cafe} Matrice de co�t et connexions associ�es. Gauche: une ligne de la matrice co�t. Droite: un choix de permutation valide. } 
\end{figure}

Afin de satisfaire la contrainte d'approvisionnement (que l'on appelle aussi la conservation de la masse), il faut que chaque boulangerie soit connect�e � un et un seul caf�. Comme il y a le m�me nombre de boulangeries que de caf�, ceci implique que chaque caf� est �galement connect� � une et une seule boulangerie. On va noter 
\eq{    
    \si : \iC \in \{1,\ldots,6\} \longmapsto \jC \in \{1,\ldots,6\}
}
un tel choix de connexion. La figure~\ref{fig:image-cafe} illustre au centre et � droite l'exemple
\eql{\label{eq-bijection-exmp}
    \si({\color{red}1})={\color{blue}5}, \;
    \si({\color{red}2})={\color{blue}2}, \;
    \si({\color{red}3})={\color{blue}6}, \;
    \si({\color{red}4})={\color{blue}1}, \;
    \si({\color{red}5})={\color{blue}3}, \;
    \si({\color{red}6})={\color{blue}4}.
}  
La contrainte de conservation de masse signifie que $\si$ est une bijection de l'ensemble $\{1,\ldots,6\}$ dans lui-m�me. On dit ainsi que $\si$ est une permutation. 

Le co�t de transport associ� � une telle bijection est la somme des co�t $C_{\iC,\si(\iC)}$  s�lectionn�s par la permutation $\si$, c'est-�-dire 
\eq{
    \text{Co�t}(\si) \eqdef 
        C_{{\color{red}1},\si({\color{red}1})} + 
        C_{{\color{red}2},\si({\color{red}2})} + 
        C_{{\color{red}3},\si({\color{red}3})} + 
        C_{{\color{red}4},\si({\color{red}4})} + 
        C_{{\color{red}5},\si({\color{red}5})} + 
        C_{{\color{red}6},\si({\color{red}6})}. 
}
Par exemple, pour la bijection~\eqref{eq-bijection-exmp} montr�e � la figure~\ref{fig:image-cafe}, on obtient comme co�t
\eq{
    C_{{\color{red}1},{\color{blue}5}} + 
    C_{{\color{red}2},{\color{blue}2}} + 
    C_{{\color{red}3},{\color{blue}6}} + 
    C_{{\color{red}4},{\color{blue}1}} + 
    C_{{\color{red}5},{\color{blue}3}} + 
    C_{{\color{red}6},{\color{blue}4}} = 
    10 + 7 + 15 + 10 + 14 + 9 = 65. 
}

Le probl�me de Monge consiste � chercher la permutation $\si$ qui a le co�t minimum, c'est-�-dire r�soudre le probl�me d'optimisation
\eq{
    \umin{\si} \enscond{
        \text{Co�t}(\si)
    }{
        \si \in \text{Perm}(\{1,\ldots,6\})
    }
}
o� l'on a not� $\text{Perm}(\{1,\ldots,6\})$ l'ensemble des permutations de l'ensemble  $\{1,\ldots,6\})$.


\begin{figure}\centering
    \begin{tabular}{@{}c@{\hspace{1mm}}c@{\hspace{1mm}}c@{\hspace{1mm}}c@{}}
        \includegraphics[width=.22\linewidth]{transport/cafe-paris/ordre-croissant-64}&
        \includegraphics[width=.22\linewidth]{transport/cafe-paris/ordre-croissant-65}&
        \includegraphics[width=.22\linewidth]{transport/cafe-paris/ordre-croissant-66}&
        \includegraphics[width=.22\linewidth]{transport/cafe-paris/ordre-croissant-152}\\
        Co�t=64 &  
        Co�t=65 &  
        Co�t=66 &  
        Co�t=152
    \end{tabular}
    \caption{\label{fig:ordre-croissant} Choix de permutation avec diff�rent co�ts. } 
\end{figure}

La figure~\ref{fig:ordre-croissant} montre que la permutation~\eqref{eq-bijection-exmp} n'est pas la meilleure : il existe par exemple une autre permutation qui a un co�t de 64. Mais est-ce la meilleure ? Il se trouve que oui, on peut en effet tester sur un ordinateur toutes les permutations de  $\{1,\ldots,6\}$ et calculer leur co�t. Combien y a-t-il de permutation au total ? Il s'agit  


\begin{figure}\centering
    \includegraphics[width=.7\linewidth]{transport/metro/plan-metro}
    \caption{\label{fig:metro} Le transport optimal en 1D. } 
\end{figure}


\begin{figure}\centering
    \begin{tabular}{@{}c@{\hspace{1mm}}c@{\hspace{1mm}}c@{\hspace{1mm}}c@{}}
        \includegraphics[width=.22\linewidth]{transport/monge-2d/article-monge}&
        \includegraphics[width=.22\linewidth]{transport/monge-2d/decroisement}&
        \includegraphics[width=.22\linewidth]{transport/monge-2d/example-10}&
        \includegraphics[width=.22\linewidth]{transport/monge-2d/example-70}
    \end{tabular}
    \caption{\label{fig:ot2d} Le transport optimal en 2D. Gauche: extrait de l'article de Monge~\cite{Monge1781}. } 
\end{figure}


%%%%%%%%%%%%%%%%%%%%%%%%%%%%%%%%%%%%%%%%%%%%%%
\section{Le Transport Optimal de Kantorovitch}



\begin{figure}\centering
    \begin{tabular}{@{}c@{\hspace{1mm}}c@{\hspace{1mm}}c@{\hspace{1mm}}c@{}}
        \includegraphics[width=.22\linewidth]{transport/kantorovitch/coupling-array}&
        \includegraphics[width=.22\linewidth]{transport/kantorovitch/coupling-squares}&
        \includegraphics[width=.22\linewidth]{transport/kantorovitch/coupling-map}&
        \includegraphics[width=.22\linewidth]{transport/kantorovitch/coupling-bipartite}
    \end{tabular}
    \caption{\label{fig:coupling-visu} Diff�rentes fa�ons de repr�senter une matrice de couplage. } 
\end{figure}





\cite{Kantorovich42}

\cite{Dantzig51}

Th�or�me de birkhoff-von Neumann. 

\cite{birkhoff,von1953certain}

%%%%%%%%%%%%%%%%%%%%%%%%%%%%%%%%%%%%%%%%%%%%%%
\section{Les Applications}


\begin{figure}\centering
\begin{tabular}{@{}c@{\hspace{1mm}}c@{\hspace{1mm}}c@{}}
    \includegraphics[width=.24\linewidth]{transport/applis/painting-1} &
    \includegraphics[width=.24\linewidth]{transport/applis/painting-2} &
    \includegraphics[width=.24\linewidth]{transport/applis/painting-2-equalized} \\
    \includegraphics[width=.24\linewidth]{transport/applis/painting-1-histo}&
    \includegraphics[width=.24\linewidth]{transport/applis/painting-2-histo}&
    \includegraphics[width=.24\linewidth]{transport/applis/painting-1-histo}\\
    Toto & tata & tutu 
\end{tabular}
\caption{\label{fig:image-eq} Exemple de transfert d'histogrammes � l'aide du transport optimal. }
\end{figure}



\begin{figure}\centering
        \includegraphics[width=.6\linewidth]{transport/applis/shapes-3d}
    \caption{\label{fig:barycenters} Exemple d'interpolation barycentrique entre des formes 3D.  }
\end{figure}


\begin{figure}\centering
    \includegraphics[width=.35\linewidth]{transport/applis/bag-word-1}
    \qquad
    \includegraphics[width=.35\linewidth]{transport/applis/bag-word-2}
\caption{\label{fig:bagwords} Exemple d'interpolation barycentrique entre des formes 3D.  }
\end{figure}


\bibliographystyle{plain}
\bibliography{biblio-sparsity,biblio-shannon,biblio-ot}


\end{document}

