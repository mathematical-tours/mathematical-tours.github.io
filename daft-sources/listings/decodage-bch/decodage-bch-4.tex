\begin{maplegroup}
\begin{flushleft}
\textbf{{\large 2e partie : }}Résolution des équations pour 
\mapleinline{inert}{2d}{i = 0;}{%
$i=0$%
} ... 
\mapleinline{inert}{2d}{n-2*t-1;}{%
$n - 2\,t - 1$%
} pour trouver 
\mapleinline{inert}{2d}{epsilon;}{%
$\varepsilon $%
}[0], 
\mapleinline{inert}{2d}{epsilon;}{%
$\varepsilon $%
}[2t+1] ... 
\mapleinline{inert}{2d}{epsilon;}{%
$\varepsilon $%
}[n-1]
\end{flushleft}

\end{maplegroup}
\begin{maplegroup}
\begin{flushleft}
Calcule les équations pour 
\mapleinline{inert}{2d}{i = 0;}{%
$i=0$%
} ... 
\mapleinline{inert}{2d}{n-2*t-1;}{%
$n - 2\,t - 1$%
}, puis les évalue :
\end{flushleft}

\end{maplegroup}
\begin{maplegroup}
\begin{mapleinput}
\mapleinline{active}{1d}{list_eqn2 := \{seq( coeff(eqn,Z,i), i=0..n-2*t-1 )\}:
eqn_eval2 := eval(list_eqn2, epsilon_connu):
eqn_eval2 := eval(eqn_eval2, sigma_connu);}{%
}
\end{mapleinput}

\mapleresult
\begin{maplelatex}
\mapleinline{inert}{2d}{eqn_eval2 :=
\{(alpha^3+alpha^2+alpha)*epsilon[14]+epsilon[12]+(alpha^3+1)*epsilon[
13]+(alpha^2+1)*epsilon[0],
epsilon[14]+(alpha^3+1)*epsilon[0]+(alpha^2+1)*(alpha^3+alpha^2)+(alph
a^3+alpha^2+alpha)*alpha^3,
epsilon[13]+(alpha^3+1)*epsilon[14]+(alpha^3+alpha^2+alpha)*epsilon[0]
+(alpha^2+1)*alpha^3,
epsilon[0]+(alpha^3+alpha^2+alpha)*(alpha^3+alpha^2)+(alpha^3+alpha+1)
*(alpha^2+1)+(alpha^3+1)*alpha^3,
epsilon[11]+(alpha^2+1)*epsilon[14]+(alpha^3+alpha^2+alpha)*epsilon[13
]+(alpha^3+1)*epsilon[12],
epsilon[10]+(alpha^3+1)*epsilon[11]+(alpha^2+1)*epsilon[13]+(alpha^3+a
lpha^2+alpha)*epsilon[12],
epsilon[8]+(alpha^3+1)*epsilon[9]+(alpha^3+alpha^2+alpha)*epsilon[10]+
(alpha^2+1)*epsilon[11],
(alpha^2+1)*epsilon[12]+(alpha^3+alpha^2+alpha)*epsilon[11]+(alpha^3+1
)*epsilon[10]+epsilon[9],
(alpha^2+1)*epsilon[10]+epsilon[7]+(alpha^3+alpha^2+alpha)*epsilon[9]+
(alpha^3+1)*epsilon[8]\};}{%
\maplemultiline{
\mathit{eqn\_eval2} := \{\mathrm{\%1}\,{\varepsilon _{14}} + {
\varepsilon _{12}} + (\alpha ^{3} + 1)\,{\varepsilon _{13}} + (
\alpha ^{2} + 1)\,{\varepsilon _{0}},  \\
{\varepsilon _{14}} + (\alpha ^{3} + 1)\,{\varepsilon _{0}} + (
\alpha ^{2} + 1)\,(\alpha ^{3} + \alpha ^{2}) + \mathrm{\%1}\,
\alpha ^{3},  \\
{\varepsilon _{13}} + (\alpha ^{3} + 1)\,{\varepsilon _{14}} + 
\mathrm{\%1}\,{\varepsilon _{0}} + (\alpha ^{2} + 1)\,\alpha ^{3}
,  \\
{\varepsilon _{0}} + \mathrm{\%1}\,(\alpha ^{3} + \alpha ^{2}) + 
(\alpha ^{3} + \alpha  + 1)\,(\alpha ^{2} + 1) + (\alpha ^{3} + 1
)\,\alpha ^{3},  \\
{\varepsilon _{11}} + (\alpha ^{2} + 1)\,{\varepsilon _{14}} + 
\mathrm{\%1}\,{\varepsilon _{13}} + (\alpha ^{3} + 1)\,{
\varepsilon _{12}}, \,{\varepsilon _{10}} + (\alpha ^{3} + 1)\,{
\varepsilon _{11}} + (\alpha ^{2} + 1)\,{\varepsilon _{13}} + 
\mathrm{\%1}\,{\varepsilon _{12}},  \\
{\varepsilon _{8}} + (\alpha ^{3} + 1)\,{\varepsilon _{9}} + 
\mathrm{\%1}\,{\varepsilon _{10}} + (\alpha ^{2} + 1)\,{
\varepsilon _{11}}, \,(\alpha ^{2} + 1)\,{\varepsilon _{12}} + 
\mathrm{\%1}\,{\varepsilon _{11}} + (\alpha ^{3} + 1)\,{
\varepsilon _{10}} + {\varepsilon _{9}},  \\
(\alpha ^{2} + 1)\,{\varepsilon _{10}} + {\varepsilon _{7}} + 
\mathrm{\%1}\,{\varepsilon _{9}} + (\alpha ^{3} + 1)\,{
\varepsilon _{8}}\} \\
\mathrm{\%1} := \alpha ^{3} + \alpha ^{2} + \alpha  }
%
}
\end{maplelatex}

\end{maplegroup}
\begin{maplegroup}
\begin{flushleft}
Met sous forme matricielle les équations :
\end{flushleft}

\end{maplegroup}
\begin{maplegroup}
\begin{mapleinput}
\mapleinline{active}{1d}{# les indices de epsilon a calculer
epsilon_indices := [0,seq(i, i=2*t+1..n-1)]: 
m2 := matrix(n-2*t,n-2*t):
b2 := vector(n-2*t):
i := 1:
\pfor eq \pin eqn_eval2 \pdo
\quad j:= 1:
\quad \pfor index \pin epsilon_indices \pdo
\quad \quad m2[i,j] := coeff(eq,epsilon[index],1):
\quad \quad j := j+1;
\quad \pend \pdo:
\quad b2[i]:=eval(eq,[epsilon[0]=0,seq(epsilon[k]=0,k=2*t+1..n-1)]);
\quad i := i+1:
\pend \pdo:}{%
}
\end{mapleinput}

\end{maplegroup}
\begin{maplegroup}
\begin{flushleft}
Calcule les valeurs de 
\mapleinline{inert}{2d}{epsilon;}{%
$\varepsilon $%
}[0], 
\mapleinline{inert}{2d}{epsilon;}{%
$\varepsilon $%
}[2t+1] ... 
\mapleinline{inert}{2d}{epsilon;}{%
$\varepsilon $%
}[n-1], puis regroupe toutes les valeurs : 
\end{flushleft}

\end{maplegroup}
\begin{maplegroup}
\begin{mapleinput}
\mapleinline{active}{1d}{epsilon_val := Linsolve(m2,b2) mod 2:
epsilon_val := [epsilon_val[1], seq(Syndi(p_recu,it),it=1..2*t),
seq(epsilon_val[it],it=2..n-2*t)];}{%
}
\end{mapleinput}

\mapleresult
\begin{maplelatex}
\mapleinline{inert}{2d}{epsilon_val := [1, alpha^3, alpha^3+alpha^2, alpha^3+alpha+1,
alpha^3+alpha^2+alpha+1, 1, alpha^3+1, alpha^3+alpha+1, alpha^3+alpha,
alpha^3+alpha^2+alpha, 1, alpha^3+alpha^2+alpha, alpha^3+alpha^2+1,
alpha^3+alpha^2+1, alpha^3+1];}{%
\maplemultiline{
\mathit{epsilon\_val} := [1, \,\alpha ^{3}, \,\alpha ^{3} + 
\alpha ^{2}, \,\alpha ^{3} + \alpha  + 1, \,\alpha ^{3} + \alpha 
^{2} + \alpha  + 1, \,1, \,\alpha ^{3} + 1, \,\alpha ^{3} + 
\alpha  + 1, \,\alpha ^{3} + \alpha ,  \\
\alpha ^{3} + \alpha ^{2} + \alpha , \,1, \,\alpha ^{3} + \alpha 
^{2} + \alpha , \,\alpha ^{3} + \alpha ^{2} + 1, \,\alpha ^{3} + 
\alpha ^{2} + 1, \,\alpha ^{3} + 1] }
%
}
\end{maplelatex}

\end{maplegroup}
\begin{maplegroup}
\begin{flushleft}
On peut maintenant déterminer l'erreur par transformée de Fourier
inverse :
\end{flushleft}

\end{maplegroup}
\begin{maplegroup}
\begin{mapleinput}
\mapleinline{active}{1d}{erreurs := Normal( TFD(epsilon_val,-1) ) mod 2;
mot_corrige := mot_transmis - erreurs mod 2:
evalb( mot_corrige = mot_code );}{%
}
\end{mapleinput}

\mapleresult
\begin{maplelatex}
\mapleinline{inert}{2d}{erreurs := [0, 0, 0, 0, 0, 0, 0, 0, 0, 0, 0, 1, 1, 0, 1];}{%
\[
\mathit{erreurs} := [0, \,0, \,0, \,0, \,0, \,0, \,0, \,0, \,0, 
\,0, \,0, \,1, \,1, \,0, \,1]
\]
%
}
\end{maplelatex}

\begin{maplelatex}
\mapleinline{inert}{2d}{true;}{%
\[
\mathit{true}
\]
%
}
\end{maplelatex}

\end{maplegroup}