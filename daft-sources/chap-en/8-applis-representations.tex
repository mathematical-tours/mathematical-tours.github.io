 
\chapter{Applications of linear representations}
\label{chap-applications-linear-representations}
 
 
Linear representations have many applications, mainly in theoretical algebra. Even in the simple framework of finite groups, this theory allows to demonstrate difficult results. Without going very far in this direction, the second paragraph shows how, starting from the knowledge of the characters of a group (that is to say of information on how our group acts on external objects), we can deduce information about the subgroups that compose it. First of all, and to provide some study material, the first paragraph studies some important finite groups. Finally, the last paragraph, which closes this book, transposes the problem of data analysis within the framework of non-commutative groups.
 
% ------------------------------------------------- -----
% ------------------------------------------------- -----
% ------------------------------------------------- -----
% section - Representation of classic groups                           
% ------------------------------------------------- -----
% ------------------------------------------------- -----
% ------------------------------------------------- -----
\section{Representation of classic groups}
% \addcontentsline{toc}{section}{Representation of classic groups}
 
The practical application of the theory developed in this chapter involves the study of elementary groups but which intervene constantly in theoretical physics or crystallography as well as in mathematics. We are therefore going to determine the list of irreducible representations of these groups, their characters, by trying to find the different geometric meanings of our groups (groups of isometries of a figure, action on faces, edges, etc.).
% ------------------------------------------------- -----
% ------------------------------------------------- -----
% sub-section - Character map                           
% ------------------------------------------------- -----
% ------------------------------------------------- -----
\subsection{Character table}
 
\index{Table!of characters} As the characters are constant on the conjugation classes $ C_1, \ldots, \, C_p $ of $ G $, we just need to draw up an array of the values of the characters $ (\chi_i)_{i = 1}^p $ on these classes. We will therefore consider the quantities $ \chi_i (g_j) $, where $ g_j $ is a representative of the class $ C_j $. In the following, we always place in first position the trivial representation, so that $ \chi_1 = 1 $. For convenience, we also indicate the cardinals $ k_j $ of the different classes $ C_j $. Finally, we use the fact that $ \chi_i (1) = n_i $, to establish a table, which is a square matrix of size $ p $:
\begin{equation*}
\begin{array}{c | cccc} & 1 & k_2 & \ldots & k_p \\& 1 & g_2 & \ldots & g_p \\\hline \chi_1 & 1 & 1 & \ldots & 1 \\\chi_2 & n_2 & \chi_2 (g_2) & \ldots & \chi_2 (g_p) \\\vdots & \vdots & \vdots & \ddots & \vdots \\\chi_p & n_p & \chi_p (g_2) & \ldots & \chi_p (g_p) \end{array}
\end{equation*}
We saw, in Paragraph~\ref{sect2-space-central-functions}, that the characters form an orthonormal basis of the space of central-functions. This results, on the character table, by orthogonality relations between the rows of the table, taking care to affect each column $ j $ by the weight $ k_j $. In fact, we also have similar relations on the columns of the matrix as specified in the following proposition:
 
\begin{prop}[Orthogonality of columns]
\label{prop-orthogonalite-columns}
If we denote by $ \chi_{\lambda} (C_1) $ the value of the character $ \chi_\lambda $ on the conjugation class $ C_1 $, we have
\begin{equation*}
\sum_{\lambda \in \wh{G}}{\chi_{\lambda} (C_1) \chi_{\lambda} (C_2)} = \left\{\begin{array}{ll} \frac{|G|}{| C_1 |} & \text{si} C_1 = C_2 \\0 & \text{otherwise} \end{array} \right. .
\end{equation*}
\end{prop}
\begin{proof}
Let $ C $ be a conjugation class. We recall that we denote by $ f_C $ the characteristic function of this class (cf. equation \eqref{eq-function-char-class}). Let us calculate its Fourier coefficients using the defining formula \eqref{defn-coef-fourier-grpe-non-commutative}:
\begin{equation*}
\forall \lambda \in \wh{G}, \quad c_{f_C} (\lambda) = \frac{1}{|G|} \sum_{g \in G}{f_C (g) \ol{\chi_{\lambda} (g)}} = \frac{| C |}{|G|} \ol{\chi_{\lambda} (C)}.
\end{equation*}
By taking successively $ C = C_1 $ then $ C = C_2 $ in this formula, then by using the formula of \textit{Plancherel}, equation \eqref{prop-formula-floorel-representation}, we obtain the desired result.
\end{proof}
 
% ------------------------------------------------- -----
% ------------------------------------------------- -----
% sub-section - Cyclic groups                           
% ------------------------------------------------- -----
% ------------------------------------------------- -----
\subsection{Cyclic groups}
 
\index{Cyclic!Group} \index{Vandermonde!Matrix} A cyclic group being commutative, according to the corollary \ref{cor-representations-commutative-grpes}, it has only representations of dimension 1, c'that is, characters in the primary sense of the term (morphisms of $ G $ in the multiplicative group $ \CC^* $). Let $ G = \{1, \, g_0, \, g_0^2, \ldots, \, g_0^{n-1}\} $ be a finite cyclic group of cardinal $ n $ and generator $ g_0 $. Let $ \omega_n = e^{\frac{2 \imath \pi}{n}} $. We have already seen that all the elements of $ \wh{G} $ are then of the form, for $ i \in \{0, \ldots, \, n-1\} $,
\begin{equation*}
\chi_i: \func{G}{\CC^*}{g = g_0^k}{(\omega_n^i)^k = e^{\frac{2 \imath \pi ik}{n}}}.
\end{equation*}
In particular, we have $ G \simeq \wh{G} $. We can therefore write the table of $ \wh{\ZZ/n \ZZ} $, which is a matrix of \textit{Vandermonde}:
\begin{equation*}
\begin{array}{c | cccc} & 1 & k_2 = 1 & \ldots & k_n = 1 \\& g_1 = 0 & g_2 = 1 & \ldots & g_n = n-1 \\\hline \chi_1 & 1 & 1 & \ldots & 1 \\\chi_2 & 1 & \omega_n & \ldots & \omega_n^{n-1} \\\chi_3 & 1 & \omega_n^2 & \ldots & \omega_n^{2 (n-1)} \\\vdots & \vdots & \vdots & \ddots & \vdots \\\chi_n & 1 & \omega_n^{n-1} & \ldots & \omega_n^{(n-1) (n-1)} \\\end{array}
\end{equation*}
 
% ------------------------------------------------- -----
% ------------------------------------------------- -----
% sub-section - Dihedral groups                           
% ------------------------------------------------- -----
% ------------------------------------------------- -----
\subsection{Dihedral groups}
 
 
\index{Group!dihedral}
 
\begin{defn}[Dihedral group]
\index{Rotation} \index{Symmetry} We call dihedral group $ D_n $ the group of isometries of the plane which keep a regular polygon with $ n $ sides. It contains $ n $ angle rotations $ \frac{k \pi}{n}, \; k = 0, \ldots, \, n-1 $ which form a subgroup isomorphic to $ C_n $, as well as $ n $ symmetries. If we denote by $ r $ the rotation of angle $ \frac{2 \pi}{n} $ and $ s $ one of the symmetries, then we have the relations
\begin{equation*}
r^n = 1 \quad \quad s^2 = 1 \quad \quad (sr)^2 = 1.
\end{equation*}
Depending on whether or not an element of $ D_n $ belongs to $ C_n $ or not, an element of $ D_n $ is uniquely written as $ s^ir^k $ with $ k = 0, \ldots, \, n-1 $ and $ i = 0, \, 1 $. In addition we have $ x \in C_n \Leftrightarrow i = 0 $.
\end{defn}
 
 
 
Note first of all that we only need to give the values of the different representations and the different characters for the two generators $ r $ and $ s $.
 
 
\textbf{Case where n is even:} \\A representation $ \rho $ of degree one (or its character, since it is the same thing) must verify $ \psi (s)^2 = 1 $, c' that is, $ \psi (s) = \pm 1 $. It must also check $ \psi (sr)^2 = 1 $, so $ \psi (r) = \pm 1 $ and $ \psi (r)^n = 1 $. Since $ n $ is even, the condition on $ r $ is written $ \psi (r) = \pm 1 $. In the end, we obtain the following 4 representations:
\begin{equation*}
\begin{array}{c | cc} & n & n \\& r^k & sr^k \\\hline \psi_1 & 1 & 1 \\\psi_2 & 1 & -1 \\\psi_3 & (-1)^k & (-1 )^k \\\psi_4 & (-1)^k & (-1)^{k + 1} \end{array}
\end{equation*}
For representations of degree two, let $ \omega_n = e^{\frac{2 \imath \pi}{n}} $. We will define for $ h \in \NN $ a representation on $ D_n $ by the formulas
\begin{equation*}
\rho_h (r^k) = \begin{pmatrix} \omega_n^{hk} & 0 \\0 & \omega_n^{- hk} \end{pmatrix} \quad \quad \rho_h (s^k) = \begin{pmatrix} 0 & \omega_n^{- hk} \\\omega_n^{hk} & 0 \end{pmatrix}.
\end{equation*}
We check that these formulas do indeed define a representation. Moreover, we can take $ h \in \{0, \ldots, \, n-1\} $, and the representations $ \rho_h $ and $ \rho_{nh} $ are isomorphic, since
\begin{equation*}
\forall g \in G, \quad \rho_h (g) = \begin{pmatrix} 0 & 1 \\1 & 0 \end{pmatrix} \rho_{nh} (g) \begin{pmatrix} 0 & 1 \\1 & 0 \end{pmatrix}^{-1}.
\end{equation*}
We therefore come to consider only the representations $ \rho_h $ for $ h = 0, \ldots, \, n / 2 $. The representation corresponding to the case $ h = 0 $ is reducible, since the lines $ \CC (e_1 + e_2) $ and $ \CC (e_1 - e_2) $ are stable. It is the same for the case $ h = n / 2 $. We can also see that $ \chi_{\rho_0} = \psi_1 + \psi_2 $ and that $ \chi_{\rho_{n / 2}} = \psi_3 + \psi_4 $, which proves that the representations $ \rho_0 $ and $ \rho_{n / 2} $ are reducible, and allows to know their decomposition. For the other values of $ h $, the representation $ \rho_h $ is indeed irreducible. Indeed, if $ \rho_h $ admitted a non-trivial under-representation, it would be a straight line, and we see that a straight line stable by $ \rho_h (r) $ is necessarily a coordinate axis, which n is not left stable by $ \rho_h (sr) $. We can calculate the characters of these $ n / 2-1 $ irreducible representations:
\begin{equation*}
\begin{array}{c | cc} & r^k & sr^k \\\hline \chi_h & 2 \cos \left(\frac{2 \pi hk}{n} \right) & 0 \end{array}
\end{equation*}
We can therefore see that these representations are not isomorphic (because their characters are different). To check that all the representations have been constructed in this way, it suffices to calculate the sum of the squares of the degrees of the representations. In total, we get $ 4 \times 1 + (n / 2 - 1) \times 4 = 2 n = | D_n | $.
 
 
\textbf{Case where n is odd:} \\This time, we can only have two representations of degree one:
\begin{equation*}
\begin{array}{c | cc} & n & n \\& r^k & sr^k \\\hline \psi_1 & 1 & 1 \\\psi_2 & 1 & -1 \\\end{array}
\end{equation*}
We define the representations $ \rho_h $ as in the case where $ n $ is even. For $ 1 \leq h \leq (n-1) / 2 $, these representations are irreducible and two by two not isomorphic. Their characters have already been calculated in the $ n $ even case. By calculating the sum of the squares of the degrees, we get $ 2 \times 1 + (n-1) / 2 \times 4 = 2 n = | D_n | $. We have thus enumerated all the irreducible representations.
% ------------------------------------------------- -----
% ------------------------------------------------- -----
% sub-section - The group $ \mathfrak{S}_4 $                           
% ------------------------------------------------- -----
% ------------------------------------------------- -----
\subsection{The group $ \mathfrak{S}_4 $}
\label{sect2-group-s4}
 
 
\index{Group!of the cube $ \mathfrak{S}_4 $} The first thing to do is to determine the conjugation classes of $ \mathfrak{S}_4 $, the group of permutations of a set with 4 elements, identified at $ \{1,2,3,4\} $. To do this, we use the lemma \ref{lem-classes-conjugation-sn}, and we thus obtain \begin{rs}
\item the class of the identity, which corresponds to the decomposition $ 4 = 1 + 1 + 1 + 1 $, that is to say in four 1-rings. It has 1 element.
\item the class of transpositions, for example of the element $ (12) $, which corresponds to the decomposition $ 4 = 2 + 1 + 1 $. It has $ 6 $ elements (choice of 2 elements out of 4 without order, which makes $ C_4^2 $).
\item the class of the three cycles, for example of the element $ (123) $, which corresponds to the decomposition $ 4 = 3 + 1 $. It comprises $ 8 $ elements (4 possible choices of 3 elements among 4, and 2 possible cycles by choice).
\item the class of the four cycles, for example of the element $ (1234) $, which corresponds to the decomposition $ 4 = 4 $. It comprises $ 6 $ elements (24 permutations which are grouped together in packets of 4 identical 4-cycles).
\item the class of couples of disjoint 2-cycles, for example of the element $ (12) (34) $, which corresponds to the decomposition $ 4 = 2 + 2 $. It comprises $ 3 $ elements (6 possible choices for the first transposition, and the choice of the second divides the number of possibilities by two).
\end{rs} By the corollary \ref{cor-classes-conjugaisons-representations}, we know that $ \mathfrak{S}_4 $ admits, up to isomorphism, 5 irreducible representations. We have already determined a certain number of representations in Paragraph~\ref{sect2-symmetric-group}: \begin{rs}
\item the trivial representation, on a space $ U $ (of dimension 1), of character $ \chi_1 = (1,1,1,1,1) $ (we thus note the corresponding line in the table of characters. indexes columns in the same order as that used for conjugation classes).
\item the alternate representation, on a space $ V $ (of dimension 1), which corresponds to the signature and has for character $ \chi_{\epsilon} = (1, -1,1, -1,1) $.
\item The standard representation, on a space $ V_s $ (of dimension 3), whose character $ \chi_s $, according to the decomposition found in Paragraph~\ref{sect2-symmetric-group}, verifies $ \chi_p = \chi_{s} + \chi_1 $ (we noted $ \chi_{p} $ the character of the representation by permutation of the elements of a base). However, the value $ \chi_p (\sigma) $ corresponds to the number of elements left fixed by $ \sigma $, which gives $ \chi_p = (4,2,1,0,0) $. In the end, we therefore have $ \chi_s = (3,1,0, -1, -1) $. We notice that we have
\begin{align*}
|G| \dotp{\chi_s}{\chi_s} & = 1 \chi_s (\Id)^2 + 6 \chi_s ((12))^2 \\
& + 8 \chi_s ((123))^2 + 6 \chi_s ((1234))^2 + 3 \chi_s ((12) (34))^2 = 24.
\end{align*}
Hence $ \dotp{\chi_s}{\chi_s} = 1 $, so according to the corollary \ref{cor-representation-isomorphes-caracteres}, the standard representation of $ \mathfrak{S}_4 $ is irreducible .
\end{rs} For the moment, we obtain a partial character table:
\begin{equation*}
\begin{array}{c | ccccc} & 1 & 6 & 8 & 6 & 3 \\& \Id & (12) & (123) & (1234) & (12) (34) \\\hline \chi_1 & 1 & 1 & 1 & 1 & 1 \\\chi_{\epsilon} & 1 & -1 & 1 & -1 & 1 \\\chi_s & 3 & 1 & 0 & -1 & -1 \end{array}
\end{equation*}
There are still two representations to be determined, and using the relation (i) of the corollary \ref{cor-enombrement}, we have $ n_4^2 + n_5^2 = 13 $, where we have noted $ n_4 $ and $ n_5 $ the degrees of the two representations. We therefore necessarily have a representation of degree 3 and the other of degree 2. The first representation can be obtained through the representation of morphisms on $ W \eqdef \Ll (V_s, V_\epsilon) $ of representations standard and alternating. It is of degree 3, and its character is $ \chi_{\Ll (W, V)} = \chi_W \ol{\chi_V} = (3, -1,0,1, -1) $. We notice that it is very different from the characters already determined, and that $ \dotp{\chi_{\Ll (W, V)}}{\chi_{\Ll (W, V)}} = 1 $, so this representation is indeed irreducible (see the exercise \oldref{exo-irred-representation-degre-1} for generalization). To determine the last representation, on a space denoted $ W'$ (of dimension 2), we use the relation (ii) of the corollary \ref{cor-enombrement}, and we find $ \chi_{W'} = (2 , 0, -1.0.2) $. In the end, we get the character table:
\begin{equation*}
\begin{array}{c | ccccc} & 1 & 6 & 8 & 6 & 3 \\& \Id & (12) & (123) & (1234) & (12) (34) \\\hline \chi_1 & 1 & 1 & 1 & 1 & 1 \\\chi_{\epsilon} & 1 & -1 & 1 & -1 & 1 \\\chi_s & 3 & 1 & 0 & -1 & -1 \\\chi_{W} & 3 & -1 & 0 & 1 & -1 \\\chi_{W'} & 2 & 0 & -1 & 0 & 2 \end{array}
\end{equation*}
 
 
 
One of the concrete realizations of the group $ \mathfrak{S}_4 $ is the group of direct isometries keeping a cube. We can see this realization by the action of the group on the four large diagonals of the cube. Consequently, the group also acts by permuting the faces of the same cube, which gives rise to a representation by permutation of the group $ \mathfrak{S}_4 $, i.e. $ \rho_E: \mathfrak{S }_4 \rightarrow GL (E) $, where $ E $ is a vector space of dimension $ 6 $. As for any representation by permutation, the value of $ \chi_E (\sigma) $, for $ \sigma \in \mathfrak{S}_4 $ is equal to the number of faces fixed by the action of $ \sigma $. Let's identify the different values of this character: \begin{rs}
\item \index{Rotation} a rotation of 180 $^\circ $ on an axis connecting the midpoints of two opposite sides: this permutation exchanges only two diagonals. It corresponds to the class of $ (12) $. No face is attached.
\item a rotation of 120 $^\circ $ along a large diagonal: only the diagonal in question is invariant, the others permuting circularly. It corresponds to the class of $ (123) $. No face is attached.
\item a rotation of 90 $^\circ $ along a coordinate axis: permute the four diagonals in a circle. It corresponds to the class of $ (1234) $. Two faces are fixed.
\item a rotation of 180 $^\circ $ along a coordinate axis: swaps the diagonals two by two. It corresponds to the class of $ (12) (34) $. Two faces are fixed.
\end{rs} The character of our representation is therefore given by
\begin{equation*}
\begin{array}{c | ccccc} & \Id & (12) & (123) & (1234) & (12) (34) \\\hline \chi_E & 6 & 0 & 0 & 2 & 2 \end{array}
\end{equation*}
We have $ \dotp{\chi_{\rho}}{\chi_{\rho}} = 3 $, so our representation is written as the sum of three irreducible representations. To calculate the decomposition of this representation, it suffices to calculate the different scalar products:
\begin{equation*}
\begin{array}{lll} \dotp{\chi_E}{\chi_1} = 1, & \dotp{\chi_E}{\chi_{\epsilon}} = 0, & \dotp{\chi_E}{\chi_s} = 0, \\\dotp{\chi_E}{\chi_{W}} = 1, & \dotp{\chi_E}{\chi_{W'}} = 1. & \end{array}.
\end{equation*}
We thus obtain the decomposition $ E = \CC \oplus W \oplus W'$, as the sum of $ G $ -modules.
% ------------------------------------------------- -----
% ------------------------------------------------- -----
% ------------------------------------------------- -----
% section - The question of simplicity                           
% ------------------------------------------------- -----
% ------------------------------------------------- -----
% ------------------------------------------------- -----
\section{The question of simplicity}
% \addcontentsline{toc}{section}{The question of simplicity}
 
\index{Subgroup!distinguished} \index{Group!simple} In this paragraph, we will use character theory to obtain information about the structure of our group. We are going to focus on finding distinguished subgroups.
% ------------------------------------------------- -----
% ------------------------------------------------- -----
% sub-section - Character core                           
% ------------------------------------------------- -----
% ------------------------------------------------- -----
\subsection{Character core}
 
\index{Kernel!of a character} \index{Character!kernel} Let us start with a proposition, which will allow us to characterize the kernel of representations.
 
\begin{prop}
\label{kernel-prop-characters}
Let $ G $ be a finite group, and $ \rho: G \rightarrow GL (V) $ a representation, of character $ \chi_V $ on a space $ V $ of dimension $ d $. We denote by $ g \in G $ an element of order $ k $. So: \begin{itemize}
\item [{\upshape (i)}] \index{Diagonalization} $ \rho (g) $ is diagonalizable.
\item [{\upshape (ii)}] $ \chi_V $ is sum of $ \chi_V (1) = \dim (V) = d $ roots \ordin{k}{iths} of the unit.
\item [{\upshape (iii)}] $ | \chi_V (g) | \leq \chi_V (1) = d $.
\item [{\upshape (iv)}] \index{Subgroup!distinguished} $ K_{\chi_V} \eqdef \enscond{x \in G}{\chi_V (x) = \chi_V (1)} $ is a distinguished subgroup of $ G $. We call it the \textit{kernel} of the representation.
\end{itemize}
\end{prop}
\begin{proofnoqed}
\begin{itemize}
\item [{\upshape (i)}] As $ g^k = 1 $, we have $ \rho (g)^k = \Id $. So the minimal polynomial of $ \rho (g) $ divides $ X^k - 1 $, which is split to single root.
\item [{\upshape (ii)}] Let $ \omega_1, \ldots, \, \omega_d $ be the eigenvalues of $ \rho (g) $, which are \ordin{k}{iès} roots of l'unit. We have $ \chi_V (g) = \omega_1 + \cdots + \omega_d $.
\item [{\upshape (iii)}] $ | \chi_V (g) | \leq | \omega_1 | + \cdots + | \omega_d | = d $.
\item [{\upshape (iv)}] If $ | \chi_V (g) | = d $, we have equality in the previous triangular inequality. This means that the complex numbers $ \omega_i $ are positively related on $ \RR $. Since they are of module 1, they are all equal. If $ \chi_V (g) = d $, we necessarily have $ \omega_i = 1 $, so $ \rho (g) = \Id $. So $ K_{\chi_V} = \Ker (\rho) $, is indeed a distinguished subgroup.{\raggedright \qed}
\end{itemize}
\end{proofnoqed}
In the following, we will need the following lemma.
 
\begin{lem}
\label{lem-extension-repr-grpe-distignue}
Let $ N \lhd G $ be a distinguished subgroup of $ G $. Let $ \rho_U $ be a representation of $ G / N $ on a vector space $ U $. Then there exists a canonical representation of $ G $ on $ U $ such that the sub-representations of $ U $ under the action of $ G / N $ are exactly those of $ U $ under the action of $ G $.
\end{lem}
\begin{proof}
Just ask
\begin{equation*}
\forall g \in G, \quad \wt{\rho}_U (g) \eqdef \rho_U \circ \pi (g),
\end{equation*}
where $ \pi: G \rightarrow G / N $ is the canonical projection. $ \wt{\rho}_U $ defines well the sought representation.
\end{proof}
 
% ------------------------------------------------- -----
% ------------------------------------------------- -----
% sub-section - Using the character table                           
% ------------------------------------------------- -----
% ------------------------------------------------- -----
\subsection{Use of the character table}
 
\index{Table!of characters} Let $ G $ be a finite group. We denote by $ \wh{G} = \{\rho_1, \ldots, \, \rho_r\} $ its dual, formed by representatives of non-isomorphic irreducible representations. Here is the result which will allow us to determine the set of distinguished subgroups of a given group.
 
\begin{prop}
The distinguished subgroups of $ G $ are exactly of the type
\begin{equation*}
\bigcap_{i \in I}{K_{\chi_i}} \quad \quad \text{where} \quad I \subset \{1, \ldots, \, r\}.
\end{equation*}
\end{prop}
\begin{proof}
Let $ N \lhd G $ be a distinguished subgroup. We denote by $ \rho_U $ the regular representation of $ G / N $. This therefore means that $ U $ is a vector space of dimension equal to $ | G / N | = |G| / | N | $, base $ \{e_g\}_{g \in G / N} $, and we have $ \rho_U (h) (e_g) = e_{hg} $. \\We have already seen in the proposition \ref{prop-repr-regular-fidele}, that the regular representation is faithful, therefore $ \rho_U $ is injective. Using the \ref{lem-extension-repr-grpe-distignue} lemma, we extend this representation to a $ \wt{\rho}_U: G \rightarrow U $ representation. Denote by $ \chi $ the character of the representation $ \wt{\rho}_U $. We then have the equality $ \Ker (\wt{\rho}_U) = \Ker (\rho_U \circ \pi) = N $, hence $ N = K_{\chi} $. \\It does not all that remains is to decompose the representation $ \wt{\rho}_U $ according to the irreducible representations, to obtain
\begin{equation*}
\chi = a_1 \chi_1 + \cdots + a_r \chi_r.
\end{equation*}
We therefore have, using point (iii) of the proposition \ref{kernel-prop-characters},
\begin{equation*}
\forall g \in G, \quad | \chi (g) | \leq \sum_{i = 1}^r{a_i | \chi_i (g) |} \leq \sum_{i = 1}^r{a_i | \chi_i (1) |} = \chi (1).
\end{equation*}
We therefore have the equality $ \chi (g) = \chi (1) $ (i.e. $ g \in K_\chi $) if and only if we have an equality in the previous triangular inequality . It follows that $ \chi (g) = \chi (1) $ if and only if $ \forall i, \; a_i \chi_i (g) = a_i \chi_i (1) $. This is ultimately equivalent to
\begin{equation*}
\forall i, \quad a_i> 0 \Longrightarrow g \in K_{\chi_i}.
\end{equation*}
We therefore have the desired result, with $ I \eqdef \enscond{i}{a_i> 0} $. \\Finally, the converse is obvious: indeed, as the $ K_{\chi_i} $ are distinguished, all subgroup of the type $ \cap_{i \in I}{K_{\chi_i}} $ is also.
\end{proof}
 
 
\begin{cor}
\index{Group!simple} $ G $ is simple if and only if $ \forall i \neq 1, \; \forall g \in G, \; \chi_i (g) \neq \chi_i (1) $.
\end{cor}
\begin{proof}
If we assume that there exists $ g \in G $, with $ g \neq 1 $, such that $ \chi_i (g) = \chi_i (1) $, then $ K_i \subset G $ is a subgroup distinguished non-trivial, so $ G $ is not simple. \\Conversely, if $ G $ is non-simple, then there exists $ g \neq 1 $ in some distinguished subgroup $ N \lhd G $ non-trivial . With the previous proposition, $ N = \cap_{i \in I}{K_i} $, so $ g \in K_i $ for $ i \in I \subset \{2, \ldots, \, r\} $. This does mean that $ \chi_i (g) = \chi_i (1) $.
\end{proof}
Thanks to the character table, we are therefore able to draw up the list of all the distinguished subgroups of a given $ G $ group, and even to determine the inclusion relations between these subgroups. For example, we can consider the group $ \mathfrak{S}_4 $, whose table was established in Paragraph~\ref{sect2-group-s4}. We see that it has two distinguished non-trivial subgroups: $ \Ker (\chi_{\epsilon}) = \mathfrak{A}_4 $ as well as $ \Ker (\chi_{W'}) = \langle ( 12) (34) \rangle $ (the class of the permutation $ (12) (34) $). Furthermore, we see that $ \Ker (\chi_W') \subset \Ker (\chi_\epsilon) $.
% ------------------------------------------------- -----
% ------------------------------------------------- -----
% ------------------------------------------------- -----
% section - Spectral analysis                           
% ------------------------------------------------- -----
% ------------------------------------------------- -----
% ------------------------------------------------- -----
\section{Spectral analysis}
% \addcontentsline{toc}{section}{Spectral analysis}
 
We saw in the previous chapter that the family of characters of a finite group constitutes an orthogonal basis of the space of central functions. The fundamental result of this paragraph is the generalization of this result to the space of functions of $ G $ in $ \CC $ as a whole. Of course, it will be necessary to consider another family of functions, which intervenes in a natural way when one tries to calculate in a matrix way the Fourier transform. This method for finding orthonormal bases of a functional space is the basis of spectral analysis on any finite group, which has many applications, especially in statistics.
% ------------------------------------------------- -----
% ------------------------------------------------- -----
% sub-section - Orthogonality of coordinate functions                           
% ------------------------------------------------- -----
% ------------------------------------------------- -----
\subsection{Orthogonality of coordinate functions}
 
 
The characters are above all theoretical objects for the search for the representations of a group $ G $ (thanks to the orthogonality relations of the rows and columns of the table of characters), and for the study of the group $ G $ lui even (study of its simplicity, resolubility, etc.). In a practical way, the fact that they form a basis only of the space of central functions makes them of little use to analyze a function from $ G $ in $ \CC $. To solve this difficulty, we will prefer to use the Fourier transform as defined in the previous paragraph. We will even see that, thanks to a certain matrix formulation, this transform also corresponds to the calculation of a decomposition in an orthogonal base.
 
 
\index{Endomorphism!unitary} We consider as usual a finite group $ G $, and we denote by $ \wh{G} = \{\rho_1, \ldots, \, \rho_p\} $ the representatives of the classes of irreducible representations. Each representation $ \rho_k $ is linked to a space $ V_k $ of dimension $ n_k $, and these different representations are of course two by two non-isomorphic. We have seen, with the proposition \ref{unit-representation-prop}, that we could, for each representation $ \rho_k $, find a basis of $ V_k $ in which the matrices $ R_k (g) $ of endomorphisms $ \rho_k (g) $ are unitary. We will denote these matrices in the form $ R_k (g) = \{r_{ij}^k (g)\} $. We thus obtain a series of applications:
\begin{equation*}
\forall k \in \{1, \ldots, \, p\}, \; \forall (i, \, j) \in \{1, \ldots, \, n_k\}^2, \quad r_{ij}^k: G \rightarrow \CC.
\end{equation*}
More precisely, we thus obtain $ \sum_{k = 1}^{p}{n_k^2} = n $ elements of $ \CC [G] $. The following proposition, which is the heart of the developments which will follow, tells us that these elements are not arbitrary.
 
\begin{thm}[Orthogonality of coordinate functions]
The $ r_{ij}^k $ for $ k \in \{1, \ldots, \, p\} $ and for $ (i, j) \in \{1, \ldots, \, n_k\}^2 $, form an orthogonal basis of $ \CC [G] $. More precisely, we have
\begin{equation*}
\forall (k, \, l) \in \{1, \ldots, \, p\}^2, \; \forall (a, \, b, \, c, \, d) \in \{1, \ldots, \, n_k\}^4, \quad \dotp{r_{ab}^k}{r_{cd }^l} = \delta_a^c \delta_b^d \delta_k^l \frac{1}{n_k}.
\end{equation*}
\end{thm}
\begin{proofnoqed}
\index{Operator!interlacing} It is in fact a question of reformulating the result of Paragraph~\ref{sect2-application-average}. Let $ \rho_k $ and $ \rho_l $ be two irreducible representations. We know that for $ f \in \Ll (U_k, \, U_l) $, the application $ \wt{f} \eqdef R_G (f) \in \Ll (U_k, \, U_l) $ is an operator interlacing. According to Schur's lemma, it is either a homothety of ratio $ \frac{\tr (f)}{n_k} $ (if $ k = l $), or the null morphism (if $ k \neq l $). \\In the bases we have chosen for $ U_k $ and $ U_l $, the morphism $ f $ is written in matrix form $ \{x_{ij}\}_{i = 1 \ldots d_l}^{j = 1 \ldots n_k} $. Similarly, we write the matrix of $ \wt{f} $ in the form $ \{\wt{x}_{ij}\}_{i = 1 \ldots d_l}^{j = 1 \ldots n_k} $. We can explicitly calculate the value of $ \wt{x}_{ij} $:
\begin{equation}
\label{eq-orth-functions-coord-interm}
\wt{x}_{i_2 i_1} \eqdef \frac{1}{|G|} \sum_{j_1, \, j_2, \, g \in G}{r_{i_2 j_2}^l (g) x_{j_2 j_1} r_{j_1 i_1}^k (g^{-1})}.
\end{equation}
Let's start with the case where the representations are not isomorphic, i.e. $ k \neq l $. The fact that $ \wt{f} = 0 $ is equivalent to $ \wt{x}_{i_2 i_1} = 0 $, and this whatever the $ x_{j_2 j_1} $. The expression for $ \wt{x}_{i_2 i_1} $ defines a linear form in $ x_{j_2 j_1} $, which is zero. This therefore means that its coefficients are zero. By noting that $ r_{i_1 j_1} (g^{-1}) = \ol{r_{j_1 i_1} (g)} $, we thus obtain, in the case where $ k \neq l $,
\begin{equation*}
\begin{split}
& \forall (i_1, \, j_1) \in \{0, \ldots, \, n_k\}^2, \; \forall (i_2, \, j_2) \in \{0, \ldots, \, n_l\}^2, \\
& \frac{1}{|G|} \sum_{g \in G}{\ol{r_{j_1 i_1}^k (g)} r_{j_2 i_2}^l (g)} \eqdef \dotp{ r_{j_1 i_1}^k}{r_{j_2 i_2}^l} = 0.
\end{split}
\end{equation*}
There now remains the case where $ k = l $. This time we have $ \wt{f} = \frac{\tr (f)}{n_i} \Id $, hence
\begin{equation*}
\forall (i_1, \, j_1) \in \{0, \ldots, \, n_k\}^2, \; \forall (i_2, \, j_2) \in \{0, \ldots, \, n_l\}^2 \quad \wt{x}_{i_2 i_1} \eqdef \frac{1}{d_i} \left( \sum_{j_1, \, j_2}{\delta_{j_1}^{j_2} x_{j_2 j_1}} \right) \delta_{i_1}^{i_2}.
\end{equation*}
By reusing the expression of $ \wt{x}_{i_2 i_1} $ obtained from the equation \eqref{eq-orth-functions-coord-interm}, and by equaling the coefficients of the linear form obtained, we have the formula
\begin{equation*}
\frac{1}{|G|} \sum_{g \in G}{\ol{r_{j_1 i_1}^k (g)} r_{j_2 i_2}^l (g)} \eqdef \dotp{r_{j_1 i_1}^k}{r_{j_2 i_2}^l} = \frac{1}{n_i} \delta_{i_1}^{i_2} \delta_{j_1}^{j_2}. \tag *{\qed }
\end{equation*}
\end{proofnoqed}
 
 
\begin{rem}
As the characters of the irreducible representations are sums of different coordinate functions, this result simultaneously asserts the orthogonality of the characters, which we have already demonstrated in the theorem \ref{thm-orthogonalite-character-group-non-commutative}.
\end{rem}
 
 
 
We denote by $ I \eqdef \enscond{(k, i, j)}{k = 1, \ldots, \, p \; \text{and} \; i, \, j = 1, \ldots, \, n_k} $. The result we have just demonstrated affirms the existence of an orthonormal basis of the space $ \CC [G] $, which we denote in the form $ \{\Delta_{kij}\}_{(k , i, j) \in I} $. We notice that we have of course $ | I | = |G| $, which is the dimension of $ \CC [G] $. These functions are defined as follows:
\begin{equation*}
\forall (k, \, i, \, j) \in I, \quad \Delta_{(k, i, j)} \eqdef \sqrt{n_k} r_{ij}^k.
\end{equation*}
 
% ------------------------------------------------- -----
% ------------------------------------------------- -----
% sub-section - Generalized Fourier series                           
% ------------------------------------------------- -----
% ------------------------------------------------- -----
\subsection{Generalized Fourier series}
 
 
\index{Endomorphism!unitary} The fundamental result of the previous paragraph therefore provides us with an orthogonal basis of the space of functions from $ G $ in $ \CC $. We can not help but make the comparison with the result already obtained thanks to the theory of characters to the theorem \ref{thm-orthogonalite-character-group-non-commutative}. However, it is important to understand that these two constructions have absolutely nothing to do. Characters are canonically defined. They do not depend on the choice of any matrix writing of our representations. It is above all a theoretical tool for obtaining information on representations (for example knowing if a representation is irreducible) or on the group itself(to determine the distinguished sub-groups). On the other hand, one can construct a quantity of orthonormal bases of $ \CC [G] $ thanks to the coordinate functions. It suffices to apply to the matrices of the different unit representations a change of unit basis. It is therefore a calculating tool. The only case where these two constructions coincide is that of commutative finite groups. Indeed, the irreducible representations of such a group are of dimension 1, and the unique entry of the corresponding matrices is equal (except for a constant) to the characters of the representation. We see moreover that in this particular case, the construction of the coordinated functions, not canonical in the general case, becomes canonical.
 
 
We now want to apply the construction we just performed to the analysis of a function $ f \in \CC [G] $. We therefore suppose that we have an orthonormal basis $ \{\Delta_{kij}\}_{(k, i, j) \in I} $. The Fourier coefficients are then defined with respect to this base.
 
\begin{defn}[Fourier coefficients]
\index{Fourier coefficient} \index{Fourier!series} \label{notation-97} For $ f \in \CC [G] $, we call \textit{Fourier coefficients} with respect to the base $ \{\Delta_{kij}\}_{(k, i, j) \in I} $, and we denote by $ c_{f} (k, \, i, \, j) $ the quantities
\begin{equation*}
\forall (k, \, i, \, j) \in I, \quad c_{f} (k, \, i, \, j) \eqdef \dotp{f}{\Delta_{kij}}.
\end{equation*}
\end{defn}
We therefore have the following Fourier development:
\begin{equation*}
f = \sum_{(k, i, j) \in I}{c_{f} (k, \, i, \, j) \Delta_{kij}}.
\end{equation*}
We can then ask ourselves what link there is between the Fourier coefficients that we have just introduced, and the Fourier transform defined in \ref{defn-coef-fourier-grpe-non-commutative}. The calculation of the Fourier transform of a function $ f \in \CC [G] $ is equivalent to the calculation, for any irreducible representation $ \rho_k $, of each coefficient of $ \Ff(f) (\rho_k) $ , that is to say
\begin{equation}
\label{eq-lien-calcul-tfnoncommutative-coeff-fourier}
\forall (k, \, i \, j) \in I, \quad \Ff(f) (\rho_k)_{ij} = \sum_{g \in G}{f(s) \left(\rho_k (g) \right)_{ij}} = c_{h} (k, \, i, \, j),
\end{equation}
where we noted $ h \eqdef \frac{1}{\sqrt{n_k}} \ol{f} $. It can therefore be seen that the calculation of the Fourier coefficients is totally equivalent to that of the calculation of the Fourier transform. By continuing to exploit the analogies between these two concepts, we can also say that the calculation of the transform is similar to a calculation of change of bases. We notice indeed that on condition of replacing $ f $ by its conjugate, then to normalize the result (by multiplying it by $ \sqrt{n_i} $), the calculation of the Fourier transform (in matrix form) in fact amounts to going from the canonical basis of $ \delta_g $ to the orthonormal basis of $ \Delta_{kij} $.
 
 
One of the questions is to know if we have, like the FFT algorithm on abelian groups, a fast calculation algorithm for the Fourier transform on a noncommutative group. We can indeed note that a naive implementation of the equations \eqref{eq-lien-calcul-tfnoncommutative-coeff-fourier} requires $ \grdo (|G|^2) $ operations. The review article by \nompropre{Rockmore} \cite{rockmore-generalized} explains that there are such algorithms for large classes of groups, including the symmetric groups discussed in the next paragraph.
% ------------------------------------------------- -----
% ------------------------------------------------- -----
% sub-section - The signal representation                           
% ------------------------------------------------- -----
% ------------------------------------------------- -----
\subsection{The signal representation}
 
 
\index{Signal} The fundamental problem of signal processing is that of the representation (in the first sense of the term) of the data studied. The language of linear algebra allows this problem to be formalized in a concise and elegant way. The signals that we want to analyze can in fact be seen as functions $ f: D \rightarrow \CC $ where $ D $ is any a priori domain (for example a square in the case of an image). In the context of computer processing, we have to consider finite $ D $ domains. The problem of the representation of a finite signal can then be summed up in the search for a \guill{good} basis of the finite dimensional vector space formed by the functions of $ D $ in $ \CC $. From a practical point of view, the quality of our database will be measured by its ability to simplify our way of understanding the data to be analyzed. In particular, the representation of the data in the new base will have to be simpler, more \textit{hollow} than in the original base.
 
 
\index{Symmetry} The first important property that we want for the base sought is to be orthonormal. This makes it possible to have simple analysis and reconstruction formulas, and more robust from a numerical point of view. This is exactly what we did during the various calculations of Fourier transforms already encountered. Secondly, the search for a good basis requires exploiting the symmetries of the domain $ D $. Even if this point may seem unrelated to the efficiency of the basis (a priori, there is no reason for the studied signals to follow the symmetries of the domain), the exploitation of the symmetries is essential to obtain algorithms of quick calculations. For example, if the FFT algorithm is so fast, it is because it fully exploits the symmetry (periodicity) of the set $ \ZZ/n \ZZ $, which makes it possible to avoid any superfluous calculation as much as possible. . In practice, this property of respect for symmetry is in fact also important for the representation of functions, because most of the \guill{natural} signals respect the regularities of the original domain. The most striking example is the study of stationary musical signals by Fourier series decomposition. We observe, after a few fundamental harmonics, coefficients which decrease very rapidly: the frequency representation of such a signal is much more compact than its temporal representation.
 
 
To try to exploit the ideas developed in the previous paragraph, it seems natural to want to provide $ D $ with a finite group structure. This usually leaves a great deal of latitude for the choice of an orthonormal basis. On the one hand, there is a multitude of structures, which may be non-isomorphic, and even if two structures are isomorphic, one may be better suited than the other to the signal being studied. On the other hand, we have already explained that the choice of different bases for the computation of the matrices of the irreducible representations gave rise to different orthonormal bases. Thus, the exercise \oldref{exo-grpe-quaternionique} proposes to use representation theory to find an orthonormal basis of the space of functions of $ \{0, \, 1\}^n $ in $ \CC $. This echoes the exercises \oldref{exo-boolean-functions} and \oldref{exo-learning-boolean-functions} which use the Walsh basis (ie abelian characters) to study Boolean functions. We will now see on a concrete example how to make these choices of structures and bases, and use them to analyze a set of data.
 
 
The example we will mention now is taken from the book by \nompropre{Diaconis} \cite{diaconis-representations-statistics}, who was the first to apply representation theory to statistics. For a complete panorama of fast computation algorithms in representation theory, we can refer to the article by \nompropre{Rockmore} \cite{rockmore-generalized}. We consider the result of a survey where we asked a significant number of people to rank in order of preference the following three places of residence: \textit{city} (proposition 1), \textit{suburb} (proposition 2), \textit{countryside} (proposition 3). Each person answers the survey by giving a permutation of the three propositions. For example, the permutation $ (2, \, 3, \, 1) $ corresponds to the classification suburb, then countryside, then city. Here are the results of the survey:
\begin{equation*}
\begin{array}{c | c | c | c} \text{city} & \text{suburb} & \text{countryside} & \text{result} \\\hline 1 & 2 & 3 & 242 \\2 & 1 & 3 & 170 \\3 & 2 & 1 & 359 \\1 & 3 & 2 & 28 \\2 & 3 & 1 & 12 \\3 & 1 & 2 & 628 \end{array}
\end{equation*}
The set of results can thus be seen as a function $ f: \mathfrak{S}_3 \rightarrow \NN $, which at each permutation of $ (1, \, 2, \, 3) $ assigns the number of people having given this permutation as an answer. The problem which arises now is that of the analysis of these results. The permutation with the highest result (in this case $ (3, \, 1, \, 2) $) gives us some information about the preferences of the respondents. But to analyze the interactions between the different permutations, a more detailed analysis must be used.
 
 
We are therefore going to perform a change of base, and calculate the way in which $ f $ decomposes in an orthogonal base obtained thanks to the irreducible representations of the group $ \mathfrak{S}_3 $. Besides the $ \rho_1 $, trivial, and $ \rho_2 $, alternating representations, there is an irreducible representation of dimension $ 2 $, the standard representation $ \rho_3 $. The exercise \oldref{exo-representation-s3} proposes a geometric method to find the associated orthogonal matrices. Here we offer another basic choice. In this case, if we denote by $ \{e_1, \, e_2, \, e_3\} $ the canonical basis of $ \CC^3 $, we choose $ \{(e_1-e_2) / \sqrt{2} , \, (e_1 + e_2-2 e_3) / \sqrt{6}\} $ for the orthonormal basis of the orthogonal of $ e_1 + e_2 + e_3 $. The matrices of the representation $ \rho_3 $ are written in this base:
\begin{align*}
\rho_3 ((1, \, 2, \, 3)) & = \begin{pmatrix} 1 & 0 \\0 & 1 \end{pmatrix}, \quad \quad & \rho_3 ((2, \, 1, \, 3)) & = \begin{pmatrix} -1 & 0 \\0 & 1 \end{pmatrix}, \\
\rho_3 ((3, \, 2, \, 1)) & = \frac{1}{2} \begin{pmatrix} 1 & - \sqrt{3} \\- \sqrt{3} & -1 \end{pmatrix}, \quad \quad & \rho_3 ((1, \, 3, \, 2)) & = \frac{1}{2} \begin{pmatrix} 1 & \sqrt{3} \\\sqrt{3} & -1 \end{pmatrix}, \\
\rho_3 ((2, \, 3, \, 1)) & = \frac{1}{2} \begin{pmatrix} -1 & \sqrt{3} \\- \sqrt{3} & -1 \end{pmatrix}, \quad \quad & \rho_3 ((3, \, 1, \, 2)) & = \frac{1}{2} \begin{pmatrix} -1 & - \sqrt{3} \\\sqrt{3} & -1 \end{pmatrix}.
\end{align*}
By calculating the dot products between the function $ f $ and the $ 6 $ coordinate functions of these representations, we can decompose the functions $ f $ as follows:
\begin{equation*}
f = \frac{1}{6} \left(1439 \rho_1 + 325 \rho_2 - 109 \rho_{311} - 1640.2 \rho_{312} + 493.6 \rho_{321} - 203 \rho_{322} \right),
\end{equation*}
where we have denoted $ \rho_{3ij} $ the coordinate function $ (i, \, j) $ of the matrix representation $ \rho_3 $. The first and most important coefficient corresponds to the mean value of the function. It is therefore not very informative. On the other hand, we note that the coefficient of $ \rho_{312} $ is clearly greater than all the others. It corresponds to the component of $ f $ on the function
\begin{equation*}
\rho_{321} = (0, \, 0, \, - 0.87, \, 0.87, \, 0.87, \, - 0.87),
\end{equation*}
where we listed the values of $ \rho_{312} $ on the elements of $ \mathfrak{S}_3 $ in the same order as that of the poll results. We see that this function actually performs a grouping of the responses in $ 3 $ packets of $ 2 $ permutations (depending on whether the value is $ -0.87 $, $ 0 $ or $ 0.87 $), each packet being characterized by the choice of the place classified last . The best estimator after the average therefore corresponds here to the choice of the least appreciated place of residence.
% ------------------------------------------------- -----
% ------------------------------------------------- -----
% ------------------------------------------------- -----
% section - Exercises                           
% ------------------------------------------------- -----
% ------------------------------------------------- -----
% ------------------------------------------------- -----
\section{Exercises}
% \addcontentsline{toc}{section}{Exercises}
\label{sect1-chap6-exercises}
 
 
 
\begin{exo}[Orthogonality of characters]
\label{exo-orthogonalite-caracteres}
 
\index{Table!of characters} \index{Unit!matrix} We write $ \Phi \eqdef \{\chi_i (C_j)\}_{1 \leq i, j \leq p} $ the table of characters. We denote by $ K = \diag (k_1, \ldots, \, k_p) $ the diagonal matrix whose inputs are the cardinals of the conjugation classes. Show that we have $ \Phi K \Phi^{*} = |G| \Id $, i.e. the matrix $ \frac{1}{\sqrt{|G|}} K^{1/2} \Phi $ is unitary. Deduce from it another proof of the column orthogonality formula, proposition \ref{prop-orthogonalite-columns}.
\end{exo}
 
 
\begin{exo}[Representation of the dihedral group]
\label{exo-repr-groupe-diedral}
 
\index{Group!dihedral} We consider the dihedral group $ D_n $. Show that it can be realized geometrically as the group formed from the following transformations: \begin{rs}
\item the rotations around the axis $ Oz $ and angles $ \frac{2 k \pi}{n} $, for $ k = 0, \ldots, \, n-1 $.
\item \index{Symmetry} the symmetries with respect to the lines of the plane $ Oxy $ forming angles $ \frac{k \pi}{n} $ with the axis $ Ox $, for $ k = 0, \ldots, \, n-1 $.
\end{rs} We thus obtain a representation $ \rho: D_n \rightarrow O_3 (\RR) $. Is it irreducible? Calculate its character, and deduce the decomposition of this representation.
\end{exo}
 
 
\begin{exo}[Representations of $ \mbold{\mathfrak{S}_3} $]
\label{exo-representation-s3}
 
We consider the triangle whose three vertices have for affixes $ 1, \, e^{\frac{2 \imath \pi}{3}}, \, e^{- \frac{2 \imath \pi}{3} } $, and we fix the base $ \{1, \, \imath\} $ of the complex plane. The group $ \mathfrak{S}_3 $ acts on the vertices of the triangle by permuting them. Calculate the matrices of two generators of this group (for example $ (12) $ and $ (123) $). Deduce the character table of the group $ \mathfrak{S}_3 $.
\end{exo}
 
 
\begin{exo}[Action on the faces of a cube]
\label{exo-action-face-cube}
 
As indicated in Paragraph~\ref{sect2-group-s4}, the group $ \mathfrak{S}_4 $ can be considered as the group of direct isometries of the cube. It therefore acts by permuting the set of 8 elements formed by the vertices of the cube, which gives rise to a representation of dimension 8. Calculate the character of this representation. Using the character table of $ \mathfrak{S}_4 $, deduce a decomposition of this representation. Do the same with the permutation of the edges.
\end{exo}
 
 
\begin{exo}[Character from $ \mbold{\mathfrak{S}_4} $]
\label{exo-caracteres-s4}
 
\index{Symmetric group} \index{Symmetric group!$ \mathfrak{S}_4 $} \index{Symmetric group!$ \mathfrak{S}_3 $} We consider the character $ \chi_{W'} $ of $ \mathfrak{S}_4 $ whose table is given by
\begin{equation*}
\begin{array}{c | ccccc} & \Id & (12) & (123) & (1234) & (12) (34) \\\hline \chi_{W'} & 2 & 0 & -1 & 0 & 2 \end{array}
\end{equation*}
\begin{enumerate}
\item We denote by $ \rho_{W'} $ the associated representation. Show that $ \rho_{W'} ((12) (34)) = \Id $.
\item \index{Quotient} \index{Subgroup!distinguished} Show that if $ H \subset G $ is a distinguished subgroup, then a representation $ \rho: G \rightarrow GL (V) $ is trivial over $ H $ if and only if it is factored by $ G / H $ in $ \wh{\rho} $:
\begin{equation*}
G \xrightarrow{\pi} G / H \xrightarrow{\wh{\rho}} GL (V)
\end{equation*}
that is, we can identify the representations of $ G / H $ with the trivial representations on $ H $.
\item We denote by $ H $ the subgroup of $ \mathfrak{S}_4 $ generated by $ (12) (34) $. Show that $ \mathfrak{S}_4 / H \simeq \mathfrak{S}_3 $. For example, we can consider the action of $ \mathfrak{S}_4 $ on the opposite faces of a cube.
\item \index{Standard!Representation} Conclude by showing that $ \rho_{W'} $ is in fact the standard representation of $ \mathfrak{S}_3 $.
\end{enumerate}
\end{exo}
 
 
\begin{exo}[Representation of a simple group]
\label{exo-representation-gpe-simple}
 
\index{Group!simple} Let $ G $ be a finite simple non-abelian group. We want to show that $ G $ does not have an irreducible representation of dimension 2. \begin{enumerate}
\item \index{Lemma!de Cauchy} \index{Cauchy@\nompropreindex{Cauchy}} \index{Equation!to classes} Start by showing Cauchy's lemma: if $ p $ is a prime number which divides $ |G| $, then $ G $ has an element of order $ p $. To do this, we can consider the set $ X = G^p $, as well as the action of the group $ \ZZ/p \ZZ $ on $ X $:
\begin{equation*}
\func{(\ZZ/p \ZZ, \, G)}{X}{(k, \, (x_{\ol{0}}, \ldots, \, x_{\ol{p-1}} ))}{(x_{\ol{k}}, \ldots, \, x_{\ol{p-1 + k}})},
\end{equation*}
to which the equation will be applied to the classes (see the book by \nompropre{Perrin}{\upshape \cite{perrin}}).
\item We assume that $ G $ has an irreducible representation $ \rho: G \rightarrow GL_2 (\CC) $. Assuming the result of the exercise \oldref{exo-representation-theory-numbers}, deduce that $ G $ has an element $ t $ of order 2.
\item Show that $ \rho $ is in fact valued in $ SL_2 (\CC) $. Then show that $ \rho (t) \in SL_2 (\CC) $ must be equal to $ - \Id $. Deduce that $ t $ is in the center of $ G $. Conclude.
\end{enumerate}
\end{exo}
 
 
\begin{exo}[Quaternionic group]
\label{exo-grpe-quaternionique}
\index{Quaternionic!group} \index{Quaternion} \index{Algebra!of quaternions} \index{Learning} \index{Function!boolean} We denote by $ H_8 $ the quaternionic group, which is formed by $ 8 $ elements $ \{\pm 1, \, \pm i, \, \pm j, \, \pm k\} $ whose multiplications are given by the sign rule and the formulas
\begin{equation*}
i^2 = j^2 = k^2 = -1, \quad jk = - kj = i, \quad ki = - ik = j, \quad ij = -ji = k.
\end{equation*}
We call $ H = \RR [H_8] $ the quaternion algebra. For more information on quaternions, see{\upshape \cite{perrin}}. \begin{enumerate}
\item We denote by $ q = a 1 + bi + cj + dk $ a generic element of $ H $. Show that the application:
\begin{equation*}
q \mapsto \begin{pmatrix} a & -b & -c & -d \\b & a & -d & c \\c & d & a & -b \\d & -c & b & a \end{pmatrix}
\end{equation*}
allows us to identify $ H $ to a subalgebra of $ M_4 (\RR) $. Deduce that $ H $ is indeed an associative algebra. Also deduce a representation of $ H_8 $.
\item \index{Unitary Endomorphism} Show that the application
\begin{equation*}
\varphi: q \mapsto M (a + \imath \, b, \, c \, - \imath d) \quad \text{with} \quad M (\alpha, \, \beta) \eqdef \begin{pmatrix} \alpha & - \ol{\beta} \\\beta & \ol{\alpha} \end{pmatrix}
\end{equation*}
allows us to identify $ H $ to a subalgebra of $ M_2 (\CC) $. Deduce a representation of dimension $ 2 $ of $ H_8 $ on the field of the complexes. Is it unitary?
\item Calculate the $ 4 $ irreducible representations of dimension 1 of $ H_8 $. Show that with the representation obtained in the previous question, we have all the irreducible representations. Then give the corresponding orthonormal basis of the space of functions of $ H_8 $ in $ \CC $.
\item Explain how $ H_8 $ allows to define a non-commutative group structure on the space $ \{0, \, 1\}^3 $. Using the result of the exercise \oldref{exo-representation-product} describe the representations of the group $ \{0, \, 1\}^{3n} $ seen as the product of the group $ \{0, \, 1\}^3$. Deduce an orthonormal basis from the space of functions of $ \{0, \, 1\}^{3n} $ in $ \CC $.
\end{enumerate} We can compare this construction to that of the Walsh base encountered in Section~\ref{sect1-transforme-walsh}, which consisted in using the abelian additive group structure of $ \{0, \, 1\}^n $. We have, in a way, refined the construction to use a non-commutative structure. There are important applications of this type of construction, for example such orthonormal bases make it possible to generalize the technique of learning Boolean functions presented in the exercise \oldref{exo-learning-boolean-functions}. It was \nompropre{Boneh}, in~{\upshape \cite{boneh-learning-boolean}} who first introduced this process.
\end{exo}
 
 
\begin{exo}[Ring of invariants]
\label{exo-ring-invariants}
 
\index{Group action!on polynomials} \index{Representation!on polynomials} We consider $ G $ the group of direct isometries of $ \RR^3 $ keeping a cube centered at the origin and whose edges are aligned with the coordinate axes. This exercise does not assume that the isomorphism between $ G $ and $ \mathfrak{S}_4 $ is known. We keep the notations of the exercise \oldref{exo-action-polynomials}, and we want to geometrically determine elements of $ K [X, \, Y, \, Z]^G $. \begin{enumerate}
\item Explain why $ X^2 + Y^2 + Z^2 \in K [X, \, Y, \, Z]^G $.
\item Show that, if we denote by $ f \eqdef XYZ $, then
\begin{equation*}
\forall A \in G, \quad f(A \cdot (X, Y, Z)) = af(X, \, Y, \, Z), \quad \text{pour} a \in \RR.
\end{equation*}
Then show that we necessarily have $ a = \pm 1 $. Conclude that $ (XYZ)^2 \in K [X, \, Y, \, Z]^G $.
\item Similarly, show that the polynomials
\begin{align*}
f & = (X + Y + Z) (X + YZ) (XYZ) (XYZ) \\
\text{and} \quad \quad g & = (X^2-Y^2) (X^2-Z^2) (Y^2-Z^2)
\end{align*}
are squared invariant under $ G $.
\end{enumerate}
\end{exo}
 
 
\begin{exo}[Auto-dual correction codes]
\label{exo-codes-autoduaux}
 
\index{Identity!of MacWilliams} \index{MacWilliams@\nompropreindex{MacWilliams}} \index{Auto-dual} \index{Code!auto-dual} \index{Weight of a word} Let $ \Cc $ un linear code on $ \FF_2 $ of size $ n $ and dimension $ k $. We denote by $ \Cc^\bot $ its dual code and $ W_\Cc (X, \, Y) $ the weight enumerator polynomial of $ \Cc $. Assume that $ \Cc $ is self-dual, i.e. $ \Cc = \Cc^\bot $. \begin{enumerate}
\item Which relation must $ n $ and $ k $ verify?
\item We denote by $ A $ the matrix $ 2 \times 2 $ defined by
\begin{equation*}
A \eqdef \frac{1}{\sqrt{2}} \begin{pmatrix} 1 & 1 \\1 & -1 \end{pmatrix}.
\end{equation*}
Using the identities of MacWilliams \ref{thm-id-mac-williams-correctors-codes}, show that $ W_\Cc (X, \, Y) $ is invariant under the action of $ A $ (as defined in Paragraph~\ref{sect3-action-polynomials}).
\item We denote by $ G_1 $ the group generated by $ A $. Write the elements that compose it. Explain why $ W_\Cc (X, \, Y) \in K [X, \, Y]^{G_1} $. Using the result of the exercise \oldref{exo-action-polynomials}, show that $ K [X, \, Y]^{G_1} $ is generated, in the sense of \eqref{eq-defn-generation-ring-invariants}, by polynomials
\begin{equation*}
X + (\sqrt{2} -1) Y \quad \text{and} \quad Y (XY).
\end{equation*}
 
\item Show that for all $ x \in \Cc $, $ w (x) $ is even. \\Deduce that $ W_\Cc (X, \, Y) \in K [X, \, Y]^{G_2} $, where we noted $ G_2 $ the group generated by $ A $ and $ - \Id_2 $.
\item Write a program \Maple{} which computes generators of the ring of invariants under the action of a given group. Use this program to calculate generators for $ K [X, \, Y]^{G_2} $. What are the problems with this method?
\end{enumerate} \index{Base!of Gröbner} There are more efficient methods to calculate the ring of invariants of a given group. The book of \nompropre{Cox}~\cite{cox} presents the Gröbner bases, and their applications to the theory of invariants.
\end{exo}
 
% flush page must be blank
%\clearpage{\pagestyle{empty} \cleardoublepage}
 
 
 

